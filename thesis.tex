\documentclass[a4paper, BCOR=1cm, DIV=12,
	headsepline=true, numbers=noenddot]{scrreprt}

% Standardpakete
\usepackage[ngerman]{babel}
\usepackage[utf8]{inputenc}
\usepackage[T1]{fontenc}

% Mathematik
\usepackage{amsmath,amssymb,amsthm}

% Sonstiges
\usepackage{tikz}
\usepackage[shortcuts]{extdash}

% Literaturverzeichnis
\usepackage[backend=biber, bibstyle=authoryear, citestyle=authoryear,
	natbib=false, abbreviate=false]{biblatex}
\bibliography{thesis}

% KOMA-Script Optionen
% Kopf- und Fußzeile
\pagestyle{headings}
\setkomafont{pageheadfoot}{\normalfont\scshape}
\deffootnote[1em]{0.5em}{1em}{\textsuperscript{\thefootnotemark}}
\setkomafont{captionlabel}{\sffamily\bfseries}
% Erlaube Zeilenumbrüche in AMS-Umgebungen
\allowdisplaybreaks[2]

% Theoreme etc.
\newtheorem*{defin}{Definition}
\newtheorem{theorem}{Satz}[chapter]
\newtheorem{lemma}[theorem]{Lemma}
\newtheorem{fact}[theorem]{Fakt}
\newtheorem*{prop}{Proposition}
\newtheorem{coroll}[theorem]{Korollar}
\newtheorem*{remarks}{Bemerkung}

% Allgemein
\def\note#1?{\textbf{#1?}\marginpar{$\leftarrow$}}
\def\todo#1.{\noindent\textsc{Todo}: \textbf{#1}.\marginpar{$\longleftarrow$}\par}
\def\wunderbrace#1_#2{\underbrace{#1}_\text{\makebox[0pt]{#2}}}
\def\overrel#1^#2{\mathrel{\mathop{#1}\limits^{#2}}}

% Mathematik
\def\to{\rightarrow}
\def\mat#1{\mathbf{#1}}
\def\aff#1{\mathbb A^{#1}}
\def\proj#1{\mathbb P^{#1}}

\def\orth#1{\mathbb O(#1)}
\def\GL#1#2{\mathrm{GL}(#1,#2)}
\def\PGL#1#2{\mathrm{PGL}(#1,#2)}
\def\PGaL#1#2{\mathrm{P\Gamma L}(#1,#2)}
\def\grass#1/#2{\mathrm{Gr}(#1,#2)}
\def\Zmod#1Z{\mathbb Z/#1\mathbb Z}
\def\Lcl(#1){L^\text{(#1)}}

\DeclareMathOperator{\sign}{sign}
\DeclareMathOperator{\ord}{ord}
\DeclareMathOperator{\id}{id}
\DeclareMathOperator{\Stab}{Stab}
\DeclareMathOperator{\Char}{char}
\DeclareMathOperator{\Aut}{Aut}
\DeclareMathOperator{\Sym}{Sym}

\begin{document}
% Titel
\tableofcontents

% Kapitel als einzelne Dateien inkludieren
\chapter{Grundlagen} \label{chap:prelim}

\chapter{Allgemeine Untersuchungen} \label{chap:general}
\section{Die \textsc{Grassmann}-Mannigfaltigkeit} \label{sec:grassmann}
Die \textsc{Grassmann}-Mannigfaltigkeit~$\grass r/n$ ist die Mannigfaltigkeit der $r$\-/dimensionalen Untervektorräume eines $n$\-/dimensionalen Vektorraums über einem Körper~$K$. Ihre Struktur ergibt sich wie folgt: die Gruppe der allgemeinen linearen Transformationen~$\GL nK$ operiert auf den $r$-dimensionalen Unterräumen von~$K^n$ transitiv. Der Stabilisator jedes Unterraums ist isomorph zu $\GL rK \times K^{r(n-r)} \times \GL{n-r}K$, damit ergibt sich
\begin{equation}
\grass r/n = \GL nK / (\GL rK \times K^{r(n-r)} \times \GL{n-r}K)
\end{equation}
und wegen $\dim \GL nK = n^2$ ist $\dim \grass r/n = r(n-r)$: das folgt aus dem Satz über die Faserdimension für die kanonische Projektion $\GL nK \to \grass r/n$.

Man kann die Grassmann-Mannigfaltigkeit als projektive Varietät im $r$\-/fachen äußeren Produkt $\mathbb P(\bigwedge^r K^n)$ auffassen: wir haben eine Abbildung $(K^n)^r \to \bigwedge^r K^n \to \mathbb P(\bigwedge^r K^n)$, die die Basis $(v_1, \dots, v_r)$ eines Unterraums auf $K^*(v_1 \wedge \dots \wedge v_r)$ schickt.\footcite[siehe hierzu auch][S.~42]{Shafarevich} Man überlegt sich leicht, dass das Bild eines Unterraums unabhängig von der Wahl der Basis ist: ist $(w_1, \dots, w_r)$ eine andere Basis gegeben durch $w_j = \sum_i a_{ij} v_i$ mit einer geeigneten Matrix $\mat A = (a_{ij}) \in \GL nK$, dann ist
\begin{align*}
\bigwedge_j w_j &= \bigwedge_j \sum_i a_{ij} v_i = \sum_{\sigma: \{1,\dots,r\} \rightarrow \{1,\dots,r\}} \bigwedge_j a_{\sigma(j)j} v_{\sigma(j)} \\
	&= \sum_{\sigma: \{1,\dots,r\} \rightarrow \{1,\dots,r\}} \prod_j a_{\sigma(j)j} \wunderbrace{\bigwedge_j v_{\sigma(j)}}_{$= 0$, falls $\sigma$ nicht injektiv} = \sum_{\sigma \in S_r} \prod_j a_{\sigma(j)j} \bigwedge_j v_{\sigma(j)} \\
	&= \sum_{\sigma \in S_r} \sign \sigma \prod_j a_{\sigma(j)j} \bigwedge_j v_j = \det A \bigwedge_j v_j
\end{align*}
Das induziert eine Einbettung $\grass r/n \into \mathbb P(\bigwedge^r K^n)$: seien $U, W \subset K^n$ verschiedene Unterräume, ihre Basen seien $(u_1, \dots, u_r)$ bzw.~$(w_1, \dots, w_r)$. Mit~$x \in U \setminus W$ gilt dann
\begin{equation*}
u_1 \wedge \dots \wedge u_r \wedge x = 0, \qquad w_1 \wedge \dots \wedge w_r \wedge x \neq 0.
\end{equation*}
Also sind $\bigwedge_j u_j$ und $\bigwedge_j w_j$ in $\mathbb P(\bigwedge^r K^n)$ auch verschieden.

Sei $(e_1, \dots, e_n)$ die Standardbasis des $K^n$, dann ist $(e_{i_1} \wedge \dots \wedge e_{i_r})_{i_1 < \dots < i_r}$ eine Basis von~$\bigwedge^r K^n$. Die Koordinaten~$(p_{i_1 \dots i_r})$ eines Punktes nennt man dann \textsc{Plücker}\-/Koordinaten.

Die Abbildung ist aber nicht surjektiv, nicht alle Elemente des äußeren Produkts entstehen durch eine Basis eines $r$\-/dimensionalen Teilraums. Das Bild ist vielmehr eine Untervarietät gegeben durch die Gleichungen\footcite[siehe][S.~42]{Shafarevich}
\begin{equation} \label{eq:grcond}
\sum_{t=1}^{r+1} (-1)^t p_{i_1 \dots i_{r-1} j_t} p_{j_1 \dots \hat{j_t} \dots j_{r+1}}, \quad\text{für } i_1 < \dots < i_{r-1}, j_1 < \dots < j_{r+1}.
\end{equation}
Wir wollen nun zeigen, dass die \textsc{Grassmann}-Mannigfaltigkeit diesen Namen verdient. Dazu zeigen wir zunächst ein Lemma.

\begin{lemma}
Für jede natürliche Zahl $n$ und jeden Körper $K$ ist die quasiprojektive Varietät~$\GL nK$ glatt und irreduzibel.
\end{lemma}
\begin{proof}
Glattheit ist offensichtlich, da $\GL nK$ algebraische Gruppe ist. Für die andere Aussage zeigen wir zunächst, dass $\SL nK$ und $K^*$ irreduzibel sind: $K^*$ ist birational isomorph zu der irreduziblen Varietät $\{ XY = 1 \} \subset \aff 2$ durch die Projektion $(x,y) \mapsto x$ mit Inversem $x \mapsto (x,1/x)$.

Zur Irreduzibilität von $\SL nK = \{\det \mat A = 1\}$: sei $(\mathrm{det}-1)=fg \in K[(X_{ij})]$ eine Faktorisierung, die konstanten Koeffizienten von $f$ und $g$ sind $\neq 0$. Treten in beiden Polynomen Monome höherer Grade auf, so treten in ihrem Produkt Monome mindestens dreier verschiedener Grade auf. Da alle Monome in der Determinante aber den gleichen Grad haben, kann das nicht sein, also ist $f$ oder $g$ konstant.

Nun haben wir eine reguläre Abbildung $\GL nK \to \SL nK$ definiert durch $\mat A \mapsto \mat (\!\det \mat A)^{-1} A$. Diese Abbildung ist surjektiv mit irreduziblem Bild, die Fasern sind isomorph zu $K^*$ und damit auch irreduzibel. Nach Theorem~8 aus \cite[S.~77]{Shafarevich} ist damit auch $\GL nK$ irreduzibel.
\end{proof}

\begin{fact}
Die Grassmann-Mannigfaltigkeit ist glatt und irreduzibel, also tatsächlich eine Mannigfaltigkeit.
\end{fact}
\begin{proof}
Wie oben erwähnt, operiert $\GL nK$ transitiv auf $\grass r/n$. Wäre $\grass r/n$ reduzibel, würde daher auch $\GL nK$ in mehrere Komponenten zerfallen---als Urbilder unter $\GL nK \to \grass r/n$.

Glattheit folgt ähnlich: da eine Gruppe transitiv operiert, muss $\grass r/n$ überall glatt sein.
\end{proof}

\section{Affine Unterräume auf projektiven Hyperflächen} \label{sec:linesproj}
Wir wollen nun untersuchen, wann auf Hyperflächen projektive Unterräume liegen. Dazu verallgemeinern wir die Vorgehensweise für Geraden auf kubische Flächen.\footcite[siehe][S.~78ff]{Shafarevich} Als Hyperfläche vom Grad $d$ bezeichnen wir die Nullstellenmenge eines homogenen Polynoms vom Grad $d$. Alle Komponenten einer solchen Hyperfläche haben Codimension 1, wie aus der Literatur bekannt.\footcite[siehe][S.~74, Theorem~4]{Shafarevich}

Sei im Folgenden $n$ die Dimension des Raums, $r$ die Dimension der gesuchten Unterräume. Die homogenen Formen vom Grad $d$ in $n+1$ Variablen bilden einen Raum $H_{n+1}^d$ der Dimension $\binom{n+d}{n}$: sind $X_0, \dots, X_n$ die Variablen, so bilden $X_0^{d_0} \dots X_n^{d_n}$ mit $d_0 + \dots + d_n = d$, $d_i \geq 0$ eine Basis. Die projektiven Hyperflächen vom Grad $d$ in einem projektiven Raum der Dimension $n$ bilden also eine Varietät isomorph zu $\proj{\binom{n+d}{n} - 1}$.

Projektive Unterräume der Dimension $r$ in $\proj n$ korrespondieren zu Untervektorräumen der Dimension $r+1$ in $\aff{n+1}$. Also bilden sie eine Varietät isomorph zu $\grass r+1/{n+1}$.

Wir definieren nun eine Teilmenge $\Gamma_{r,n}^d$ im Produkt $\mathbb P(H_{n+1}^d) \times \grass r+1/{n+1}$:
\begin{equation}
\Gamma_{r,n}^d = \{(F,L) \in \mathbb P(H_{n+1}^d) \times \grass r+1/{n+1} \colon \text{$L$ liegt auf der durch $F$ definierten Varietät} \}
\end{equation}

\begin{fact} \label{fact:gammaproj}
Die Menge $\Gamma_{r,n}^d$ ist eine projektive Varietät.
\end{fact}
\begin{proof}
Es sind also zunächst zur verbalen Beschreibung äquivalente algebraische Gleichungen zu finden. Betrachten wir dazu einen $(r+1)$-dimensionalen Unterraum~$W$ in~$\aff{n+1}$ mit Basis $(v_i)_{0 \leq i \leq r}$, die einzelnen Komponenten eines Basisvektors~$v_i$ mögen $v_{ij}$, $0 \leq j \leq n$, heißen. Dann gilt für die Plückerkoordinaten des Unterraums:
\begin{equation}
p_{i_0 \dots i_r} = \sum_{\sigma \in S_{r+1}} \sign \sigma \cdot v_{0i_{\sigma(0)}} \dots v_{ri_{\sigma(r)}}
\end{equation}
wobei $S_{r+1}$ die Permutationsgruppe von $\{0,\dots,r\}$ ist. Wie lässt sich nun der Unterraum aus den Plückerkoordinaten rekonstruieren? Zunächst macht man sich klar, dass sich $W$ schreiben lässt als
\begin{equation}
W = \left\{ \sum\limits_{\sigma \in S_{r+1}} \sign \sigma \cdot \phi(v_{\sigma(1)}, \dots, v_{\sigma(r)}) v_{\sigma(0)} \;\middle|\; \phi \colon (K^{n+1})^r \to K \text{ multilinear} \right\}.
\end{equation}
Eine solche Multilinearform $\phi \colon (K^{n+1})^r \to K$ wiederum hat die Form
\begin{equation}
\phi(x_1, x_2, \dots, x_r) = \sum_{\iota: \{1, \dots, r\} \to \{0, \dots, n\}} \alpha_{\iota(1) \dots \iota(r)} \langle x_1, e_{\iota(1)} \rangle \dots \langle x_r, e_{\iota(r)} \rangle
\end{equation}
mit Koeffizienten $\alpha_{i_1 \dots i_r} \in K$. Sei nun $w \in W$, Einsetzen liefert:
\begin{align*}
w &= \sum\limits_{\sigma \in S_{r+1}} \sign \sigma \sum_{\iota: \{1, \dots, r\} \to \{0, \dots, n\}} \alpha_{\iota(1) \dots \iota(r)} \langle v_{\sigma(1)}, e_{\iota(1)} \rangle \dots \langle v_{\sigma(r)}, e_{\iota(r)} \rangle v_{\sigma(0)} \\
	&= \sum\limits_{\sigma \in S_{r+1}} \sign \sigma \sum_{\iota: \{1, \dots, r\} \to \{0, \dots, n\}} \alpha_{\iota(1) \dots \iota(r)} v_{\sigma(1)\iota(1)} \dots v_{\sigma(r)\iota(r)} \sum_{j=0}^n v_{\sigma(0)j} e_j \\
	&= \sum_{j=0}^n \left(\sum\limits_{\sigma \in S_{r+1}} \sign \sigma \sum_{\iota: \{1, \dots, r\} \to \{0, \dots, n\}} \alpha_{\iota(1) \dots \iota(r)} v_{\sigma(1)\iota(1)} \dots v_{\sigma(r)\iota(r)} v_{\sigma(0)j}\right) e_j \\
	&= \sum_{j=0}^n \left(\sum\limits_{\sigma \in S_{r+1}} \sign \sigma \sum_{\substack{\iota: \{0, \dots, r\} \to \{0, \dots, n\} \\ \iota(0)=j }} \alpha_{\iota(1) \dots \iota(r)} v_{\sigma(0)\iota(0)} \dots v_{\sigma(r)\iota(r)}\right) e_j \\
	&= \sum_{j=0}^n \left(\sum_{\substack{\iota: \{0, \dots, r\} \to \{0, \dots, n\} \\ \iota(0)=j }} \alpha_{\iota(1) \dots \iota(r)} \sum\limits_{\sigma \in S_{r+1}} \sign \sigma \cdot v_{0\iota(\sigma^{-1}(0))} \dots v_{r\iota(\sigma^{-1}(r))}\right) e_j \\
\Rightarrow\quad w_j	&= \sum_{\substack{\iota: \{0, \dots, r\} \to \{0, \dots, n\} \\ \iota(0)=j }} \alpha_{\iota(1) \dots \iota(r)} \sum\limits_{\sigma \in S_{r+1}} \sign \sigma \cdot v_{0(\iota\circ\sigma)(0)} \dots v_{r(\iota\circ\sigma)(r)}
\end{align*}
Die innere Summe verschwindet für nicht injektive $\iota$: gilt $\iota(x) = \iota(y)$ für $x \neq y$, so ersetze man $\sigma$ durch $\sigma \circ \tau$ mit einer Transposition $\tau: x \leftrightarrow y$. Die Summanden für $\sigma$ und $\sigma \circ \tau$ heben sich dann genau auf.

Für die übrigen (also injektiven) $\iota$ ist diese Summe genau die Plückerkoordinate $\sign \tau \cdot p_{(\iota\circ\tau)(0)\dots(\iota\circ\tau)(r)}$, wobei $\tau \in S_{r+1}$ die Permutation ist, die $\iota \circ \tau$ streng monoton macht.

Ist nun~$F$ eine homogene Form über $X_0, \dots, X_n$, so setzen wir $X_j = w_j$ und erhalten eine algebraische Gleichung in den~$\alpha_{i_1 \dots i_r}$ und den~$p_{i_0 \dots i_r}$. Der entsprechende Unterraum zu den Plückerkoordinaten liegt genau dann auf der durch die Form beschriebene Fläche, wenn die Form in den~$\alpha_{i_1 \dots i_r}$ identisch erfüllt ist. Demnach erhalten wir die gesuchten Gleichungen durch Koeffizientenvergleich.
\end{proof}

Damit können wir nun untersuchen, auf welchen projektiven Varietäten eines vorgegebenen Grades Unterräume einer gewissen Dimension liegen. Insbesondere zeigt sich, dass es in bestimmten Fällen solche Unterräume immer gibt.

Wir haben Projektionen $\phi \colon \Gamma_{r,n}^d \to \mathbb P(H_{n+1}^d)$ und $\psi: \Gamma_{r,n}^d \to \grass r+1/{n+1}$. Beide sind natürlich regulär.

\begin{prop}
Die Abbildung $\psi$ ist surjektiv mit Fasern isomorph zu einem projektiven Raum der Dimension $\binom{n+d}d - \binom{r+d}d - 1$.
\end{prop}
\begin{proof}
Betrachte den Unterraum $L_0 = \{X_{r+1} = \dots = X_n = 0\} \subset \proj n$: dessen Urbild ist isomorph zu dem der anderen Unterräume, da auf $\grass r+1/{n+1}$ sowie $\mathbb P(H_{n+1}^d)$ die Gruppe $\GL{n+1}K$ operiert, und das auf der \textsc{Grassmann}-Mannigfaltigkeit transitiv geschieht.

Sei $a_{i_0 \dots i_n}$ der Koeffizient von $X_0^{i_0} \dots X_n^{i_n}$ für $i_0 + \dots + i_n = d$. Dann bilden die Fasern einen Unterraum in $\mathbb P(H_{n+1}^d)$ gegeben durch $a_{i_0 \dots i_r 0 \dots 0} = 0$: auf allen solchen Flächen liegt offenbar $L_0$. Ist hingegen einer der Koeffizienten $a_{i_0 \dots i_r 0 \dots 0}$ nicht null, dann erhalten wir nach Einsetzen von $X_{r+1} = \dots = X_n = 0$ eine nichttriviale Form in $K[X_0, \dots, X_r]$. Nach \textsc{Hilbert}s Nullstellensatz kann diese nicht identisch verschwinden auf $\aff{n-r}$, also liegt $L_0$ nicht auf der Fläche. Damit folgt
\begin{equation*}
\psi^{-1}(L_0) \cong \mathbb P(H_{n+1}^d / H_{r+1}^d) \cong \proj{\binom{n+d}d - \binom{r+d}d - 1}.
\end{equation*}
Insbesondere ist ein Unterraum irreduzibel.
\end{proof}

\begin{coroll}
Die Varietät $\Gamma_{r,n}^d$ ist irreduzibel und hat Dimension
\begin{equation}
(r+1)(n-r) + \binom{n+d}d - \binom{r+d}d - 1.
\end{equation}
\end{coroll}
\begin{proof}
Das folgt mit dem Satz über irreduzible Fasern, den wir oben schon verwendet haben: die Projektion auf $\grass r+1/{n+1}$ hat Fasern isomorph zu einem projektiven Unterraum von $\mathbb P(H_{n+1}^d)$, sind also irreduzibel.

Die Dimension ist $\dim \grass r+1/{n+1} + \dim \psi^{-1}(L) = (r+1)(n-r) + \binom{n+d}d - \binom{r+d}d - 1$ nach dem Satz über die Faserdimension.
\end{proof}

Betrachten wir nun die Projektion $\phi \colon \Gamma_{r,n}^d \to \mathbb P(H_{n+1}^d)$: Surjektivität bedeutet, dass auf jeder Fläche vom Grad $d$ in $\proj n$ ein Unterraum der Dimension $r$ liegt. Dafür notwendig ist
\begin{align*}
\dim \Gamma_{r,n}^d = (r+1)(n-r) + \binom{n+d}d - \binom{r+d}d - 1 &\geq \binom{n+d}d - 1 \\
\Leftrightarrow \qquad (r+1)(n-r) &\geq \binom{r+d}d
\end{align*}
Andernfalls ist der Menge der Flächen, auf denen solche Unterräume liegen, eine echte abgeschlossene Teilmenge.
\begin{theorem}
Liegt auf jeder Hyperfläche $\{F = 0\} \subset \proj n$ mit $\deg F = d$ einen $r$-dimensionaler Unterraum, so gilt
\begin{equation}
(r+1)(n-r) \geq \binom{r+d}d \qquad\text{bzw.}\qquad n \geq \frac{\binom{r+d}d}{r+1} + r.
\end{equation}
\end{theorem}

Uns interessiert in den folgenden Kapiteln die Situation $n=3$, $r=1$, also Geraden auf projektiven Flächen. Obige Bedingung wird hier zu $d \leq 3$, die Differenz $\dim \Gamma_{r,n}^d - \dim \mathbb P(H_{n+1}^d)$ ist $3-d$. Folgende Fälle ergeben sich:
\begin{itemize}
\item Für $d=1$ liegt eine Ebene vor, die Geraden auf einer Ebene bilden eine Varietät isomorph zu $\grass 2/3$, diese hat tatsächlich Dimension $3-d = 2$.
\item Wie in der Einleitung angedeutet, liegen auf jeder nichtentarteten Quadrik zwei Familien von Geraden, die durch $\proj 1$ parametrisiert werden. Die Fasern sind also im generischen Fall isomorph zu $\proj 1 \sqcup \proj 1$, das hat Dimension $3-d = 1$.
\item Auf jeder regulären kubischen Flächen liegen $27$ Geraden.\footcite[siehe etwa][]{Henderson} Mit obiger Methode erhalten wir zumindest die Dimension dieser Menge, nämlich $3-d = 0$.
\item Wie oben gesehen, ist $\phi$ für $d \geq 4$ nicht surjektiv. Wir können aber zumindest die Codimension des Bildes ausrechnen: im nächsten Kapitel werden wir eine spezielle Klasse von Flächen untersuchen, je eine für jeden Grad $d \geq 4$. Wir werden erhalten, dass auf jeder nur endlich viele Geraden liegen. Folglich ist die generische Faserdimension auf dem Bild $0$, dieses hat also Codimension $d-3$ in $\mathbb P(H_n^d)$.
\end{itemize}

\chapter{Geraden auf Fermat-Flächen} \label{chap:fermat}
Wir untersuchen nun einen wichtigen Spezialfall der allgemeinen Theorie, nämlich Geraden auf Fermat-Flächen. Im Jahre 1637 vermutete \textsc{Pierre de Fermat}, dass die Gleichung
\begin{equation*}
X^d + Y^d = Z^d
\end{equation*}
für $d \geq 3$ keine nichttrivialen ganzzahligen Lösungen hat. Dazu äquivalent ist, dass die Varietäten
\begin{equation*}
F_d^{(2)} = \{ X^d + Y^d = Z^d \} \subset \proj 2(\mathbb C)
\end{equation*}
keine rationalen Punkte haben. Diese nennen sich \emph{Fermat-Kurven}. Wir bezeichnen eine analoge Familie von Flächen im $\proj 3$ als \emph{Fermat-Flächen}:
\begin{equation}
F_d = F_d^{(3)} = \{ X_0^d + X_1^d + X_2^d + X_3^d = 0 \} \subset \proj 3(K).
\end{equation}
Dabei können wir o.\,E. $\Char K \nmid d$ annehmen: sei $\Char K = p$, $d = kp$, dann ist $X_0^d + X_1^d + X_2^d + X_3^d = (X_0^k + X_1^k + X_2^k + X_3^k)^p$. Mithin ist $F_{kp} = F_k$, nun iteriert man bis $p \nmid d$ gilt. Eine allgemeine Theorie sogenannter Fermat-Varietäten findet sich in \cite{Fermat}.

\section{Allgemeine Betrachtungen}
Wir wollen nun die Rechnungen aus dem Beweis von Fakt~\ref{fact:gammaproj} konkreter machen. Sei $W \in \grass 2/4$ ein Unterraum mit Basis $\{a = (a_0, \dots, a_3), b = (b_0, \dots, b_3)\}$. Dann sind die Plückerkoordinaten $p_{ij} = a_i b_j - a_j b_i$, wobei $p_{ij} + p_{ji} = 0$. Wir rekonstruieren nun den Unterraum aus den Plückerkoordinaten:
\begin{equation}
W = \{ \phi(a)b - \phi(b)a \colon \phi \in (K^4)^* \}.
\end{equation}
Ein $\phi \in (K^4)^*$ hat die Form $\phi(x) = \sum_{i=0}^3 \alpha_i \langle x, e_i \rangle$ mit geeigneten Koeffizienten $\alpha_i \in K$. Damit
\begin{align*}
\phi(a)b - \phi(b)a &= \sum_i \alpha_i \langle a, e_i \rangle b - \sum_i \alpha_i \langle b, e_i \rangle a \\
	&= \sum_i \alpha_i a_i \sum_j b_j e_j - \sum_i \alpha_i b_i \sum_j a_j e_j \\
	&= \sum_j \sum_i \left(\alpha_i a_i b_j - \alpha_i b_i a_j \right) e_j \\
\intertext{Dabei laufen die Summen jeweils über $\{0,\dots,3\}$. Für $i=j$ verschwinden die Summanden jeweils, also erhalten wir}
\phi(a)b - \phi(b)a &= \sum_j \left(\sum_{i \neq j} \alpha_i p_{ij} \right) e_j
\end{align*}

Das setzen wir nun in die Gleichung der Fermat-Fläche $F_d$ ein:
\begin{align*}
0 = \sum_{j=0}^3 X_j^d &= \sum_{j=0}^3 \left(\sum_{i \neq j} \alpha_i p_{ij} \right)^d \\
\text{(Multinomialtheorem)}\qquad &= \sum_{j=0}^3 \sum_{\substack{(d_0,\dots,d_3) \\ \sum d_i=d,\;d_j=0}} \binom d{d_0,\dots,d_3} \prod_{i=0}^3 \alpha_i^{d_i} p_{ij}^{d_i} \\
	&= \sum_{\substack{(d_0,\dots,d_3) \\ \sum d_i=d}} \binom d{d_0,\dots,d_3} \left(\sum_{\substack{j=0 \\ d_j=0}}^3 \prod_{i=0}^3 p_{ij}^{d_i} \right) \prod_{i=0}^3 \alpha_i^{d_i}
\end{align*}
Da die Gleichung in den $\alpha_i$ identisch gelten soll, können wir nach \textsc{Hilbert}s Nullstellensatz einen Koeffizientenvergleich machen. Ein Vergleich der Koeffizienten zu $\prod_{i=0}^3 \alpha_i^{d_i}$ für ein $(d_0,\dots,d_3)$ ergibt
\begin{equation}
\binom d{d_0,\dots,d_3} \sum_{\substack{j=0 \\ d_j=0}}^3 \prod_{i=0}^3 p_{ij}^{d_i} = 0.
\end{equation}
Das liefert uns den folgenden Fakt.

\begin{fact}
Eine projektive Gerade mit Plückerkoordinaten $(p_{ij})$ liegt genau dann auf der Fermat-Fläche vom Grad $d$, wenn für alle $(d_0,\dots,d_3)$ mit $d_0 + \dots + d_3 = d$ und $\binom d{d_0,\dots,d_3} \neq 0$ die folgende Gleichung gilt:
\begin{equation}
\sum_{\substack{j=0 \\ d_j=0}}^3 \prod_{i=0}^3 p_{ij}^{d_i} = 0.
\end{equation}
\end{fact}

\section{Reguläre Geraden}
Es ist bekannt, dass auf einer Fermat-Fläche vom Grad $d$ eine Familie von $3d^2$ Geraden liegt.\footcite[siehe u.\,a.][S.~5]{LinesOnFermat} Wir zeigen im Folgenden, dass es in den meisten Fällen auch nicht mehr als diese gibt.

Seien Indizes $i,j,k,l$ gewählt mit $\{i,j,k,l\} = \{0,1,2,3\}$. Unabhängig von $d$ ist dann $\binom d{d,0,0,0} = 1 \neq 0$ (und analog für Permutationen), damit haben wir
\begin{equation} \label{eq:powers}
p_{ij}^d + p_{ik}^d + p_{il}^d = 0.
\end{equation}
Weiterhin ist $\binom d{d-1,1,0,0} = d \neq 0$, daher erhalten wir Gleichungen der Form
\begin{equation} \label{eq:ratios}
p_{jk}^{d-1} p_{ik} + p_{jl}^{d-1} p_{il} = 0 \qquad\overrel\Longleftrightarrow^{p_{il}, p_{jk} \neq 0}\qquad \frac{p_{ik}}{p_{il}} = -\left(\frac{p_{jl}}{p_{jk}}\right)^{d-1}.
\end{equation}
Inwiefern weitere Gleichungen erfüllt sein müssen, beantwortet die folgende Proposition.
\begin{prop}
Sei $p \in \mathbb P$ und $d$ kein Vielfaches von $p$. Gilt weiterhin, dass $d-1$ keine $p$-Potenz ist, dann gibt es $d_0, d_1, d_2 > 0$ mit $p \nmid \binom d{d_0,d_1,d_2,0}$.
\end{prop}
\begin{proof}
Nach der $p$-adischen Stirlingformel ist $\ord_p n! = \frac{n - \sigma_p(n)}{p-1}$, wobei $\sigma_p$ die $p$-adische Quersumme ist.\footcite[Kap.~2, §8, Lemma~1, S.~171]{LieGroups} Damit folgt $(p-1)\ord_p \binom d{d_0,d_1,d_2,d_3} = \sigma_p(d) - \sum_i \sigma_p(d_i)$. Damit $p \nmid \binom d{d_0,d_1,d_2,d_3}$ gilt, darf also bei der Summe $\sum_i d_i$ kein $p$-adischer Übertrag stattfinden. Um die Proposition zu zeigen, genügt es daher, $d$ in drei nichtverschwindende Summanden zu zerlegen, sodass die Summation keinen Übertrag ergibt. Das ist offenbar für $\sigma_p(d) \geq 3$ immer möglich.

Die Fälle $\sigma_p(d) = 0, 1$ führen auf $d = 0, p^n$ mit $n \in \mathbb N$, was der Voraussetzung widerspricht. Für $\sigma_p(d) = 2$ hat $d$ wegen $p \nmid d$ die Form $d=p^n+1$, was ebenfalls ausgeschlossen ist. Also ist $\sigma_p(d) \geq 3$.
\end{proof}

\begin{fact}
Sei $K$ Körper der Charakteristik~$p \in \mathbb P \cup \{0\}$, $d \geq 3$ mit $p \nmid d$, und $d-1$ keine Potenz von~$p$. Dann liegen auf $F_d(K)$ genau die drei Familien von Geraden
\begin{equation} \label{eq:regular}
\begin{split}
\text{(I)}\qquad	&(1:\theta:0:0)(0:0:1:\eta) \\
\text{(II)}\qquad	&(1:0:\theta:0)(0:1:0:\eta) \\
\text{(III)}\qquad	&(1:0:0:\theta)(0:1:\eta:0)
\end{split} \qquad \theta, \eta \in \mu_{2d} \setminus \mu_d,
\end{equation}
wobei $\mu_d \subset K$ die Menge der $d$-ten Einheitswurzeln ist.
\end{fact}
\begin{proof}
Nach voriger Proposition ist $\binom d{d_0,d_1,d_2,0} \neq 0$ für geeignete $d_0, d_1, d_2 > 0$. Damit haben wir
\begin{equation} \label{eq:products}
p_{03}^{d_0} p_{13}^{d_1} p_{23}^{d_2} = 0
\end{equation}
und Varianten. Das bedeutet: für jedes $i$ verschwindet mindestens eines der $p_{ij}$. Wir machen uns zunächst klar, dass nicht mehr verschwinden können: sei o.\,E. $p_{01} = p_{02} = 0$, dann ist wegen \eqref{eq:powers} auch $p_{03} = 0$. Mit \eqref{eq:ratios} folgt daraus, dass $p_{13}^{d-1}p_{23} = p_{12}^{d-1}p_{23} = p_{12}^{d-1}p_{13} = 0$, also verschwinden mindestens zwei von der drei Koordinaten $p_{12}$, $p_{13}$, $p_{23}$. Dass nur eine Koordinate nicht verschwindet, geht aber wegen \eqref{eq:powers} nicht.

Also verschwindet jeweils einer der Summanden in Gleichung \eqref{eq:powers} und diese bekommen die Form $X^d + Y^d = 0$. Das ist äquivalent zu $(X/Y)^d = -1$ bzw. $X/Y \in \mu_{2d} \setminus \mu_d$. Schreiben wir nun die Koeffizienten in einer Tabelle auf:

{\vskip 2ex\hfil
\begin{tabular}{|c|c|c|c|} \hline
0 & $p_{01}$ & $p_{02}$ & $p_{03}$ \\ \hline
$-p_{01}$ & 0 & $p_{12}$ & $p_{13}$ \\ \hline
$-p_{02}$ & $-p_{12}$ & 0 & $p_{23}$ \\ \hline
$-p_{03}$ & $-p_{13}$ & $-p_{23}$ & 0 \\ \hline
\end{tabular}
\hfil\vskip 2ex}

In jeder Spalte und Zeile steht zusätzlich eine Null, insgesamt hat die Tabelle also acht Nulleinträge. Von den vier Nulleinträgen, die nicht auf der Diagonale liegen, sind jeweils zwei oberhalb und zwei unterhalb, also verschwinden zwei der sechs $p_{ij}$, seien dies $p_{ij}$ und $p_{kl}$, dabei gilt $\{i,j,k,l\} = \{0,1,2,3\}$. Wir können also ohne Einschränkung annehmen, dass $p_{01}$ und $p_{23}$ verschwinden.

Setzen wir nun $p_{02} = 1$, dann ergibt sich $p_{03} = \theta$ und $p_{12} = \eta$ mit $\theta, \eta \in \mu_{2d}$ und $\theta^d = \eta^d = -1$. Mit \eqref{eq:grcond} folgt $p_{13} = \theta\eta$. Die entsprechende projektive Gerade wird durch $(1,\theta,0,0)$ und $(0,0,1,\eta)$ aufgespannt. Man überzeugt sich leicht, dass diese tatsächlich auf der Fermat-Fläche liegt. Die anderen Klassen ergeben sich für $p_{02} = p_{13} = 0$ bzw.~$p_{03} = p_{12} = 0$.
\end{proof}

Im generischen Fall liegen also $3d^2$ Geraden auf eine Fermat-Fläche vom Grad $d$. Im nächsten Kapitel werden wir ihre Konfiguration untersuchen. Betrachten wir aber zunächst noch die Spezialfälle.

\section{Geistergeraden}
Die Existenz von zusätzlichen Geraden ist auch bekannt.\footcite[siehe][S.~14f]{LinesOnFermat} Wir wollen nun genau untersuchen, wann es sie gibt und wie viele davon.
\begin{prop}
Sei $p \in \mathbb P$ und $d = p^n+1$ mit $n \in \mathbb N$. Dann sind alle Multinomialkoeffizienten $\binom d{d_0,d_1,d_2,d_3}$ mit $d_0, d_1, d_2, d_3 \not\in \{d-1, d\}$ durch $p$ teilbar.
\end{prop}
\begin{proof}
Wir zählen, wie oft die Faktoren in Zähler und Nenner dieses Bruches verschiedene Potenzen von $p$ enthalten:
\begin{equation*}
\binom d{d_0,d_1,d_2,d_3} = \frac{d!}{d_0! d_1! d_2! d_3!}.
\end{equation*}
Für $0 < k < n$ enthalten im Zähler $p^{n-k}$ Faktoren einen Faktor $p^k$, im Nenner sind es $\sum_i \lfloor d_i/p^k \rfloor \leq \sum_i d_i/p^k = d/p^k$, also höchstens ebenso viele. Der Zähler enthält allerdings einen Faktor $p^n$, der Nenner wegen $d_i \not\in \{d-1, d\} = \{p^n, p^n+1\}$ nicht. Damit ist der Bruch durch $p$ teilbar.
\end{proof}

\begin{lemma}
Sei $K$ ein Körper der Charakteristik $p$. Die Gleichung $X+Y+Z=0$ mit $X,Y,Z \in \mu_{p^n-1}$ hat dann $(p^n-1)(p^n-2)$ Lösungen.
\end{lemma}
\begin{proof}
Offenbar gilt $\mu_{p^n-1} = \mathbb F_{p^n}^* \subset K$. Es sind also alle Lösungen von $X+Y+Z=0$ mit $X,Y,Z \in \mathbb F_{p^n} \setminus \{0\}$ zu finden. Das Folgende ist nun Kombinatorik: $X$ können wir aus $p^n-1$ verschiedenen Werten wählen. Haben wir $Y$ gewählt, ergibt sich $Z$ als $Z=-X-Y$. Damit $Z \neq 0$ ist, muss $X+Y \neq 0$ sein, also $Y \not\in \{0,-X\}$. Folglich gibt es für $Y$ genau $p^n-2$ mögliche Wahlen.
\end{proof}
\begin{coroll} \label{cor:projrootsum}
Sei $K$ ein Körper der Charakteristik $p$. Die Nullstellenmenge der Gleichung $X+Y+Z=0$ in $\proj 3(K)$ mit $X,Y,Z \in \mu_{p^n-1}$ besteht dann aus $p^n-2$ Elementen.
\end{coroll}

\begin{theorem}[Geistergeraden]
Für alle Charakteristiken $p > 2$ und Grade $d = p^n + 1$ mit $n \in \mathbb N$ liegen auf $F_d$ neben den Familien von Geraden (I)--(III) zusätzlich die Geraden
\begin{equation} \label{eq:ghost}
\text{(IV)}\qquad (0:\mu\eta i:1:\nu\theta i)(-\mu\eta i:0:\nu\lambda i/\theta:\lambda)
\end{equation}
mit den Parametern $\lambda \in \mu_d$, $\eta, \theta \in \mu_{2d}$ und $\mu, \nu \in \mu_{2(d-2)}$ und den Bedingungen $\mu^2 + 1 + \nu^2 = 0$, $(i\eta)^d = \mu^{d-2}$ und $(i\theta)^d = \nu^{d-2}$. Weitere Geraden gibt es nicht.
\end{theorem}
\begin{remarks}
Ändert man $\eta$ und $\mu$ oder $\theta$ und $\nu$ um den Faktor $-1$ ab, erhält man diesselbe Gerade. Die Zuordnung von Geraden zu Parametern ist also nicht eindeutig. Legt man sich allerdings für $\mu$ und $\nu$ auf eine Quadratwurzel von $\mu^2$ bzw.~$\nu^2$ fest, ist die Eindeutigkeit wiederhergestellt, wie sich aus dem Beweis ergibt.
\end{remarks}
Die Geraden \eqref{eq:regular} nennen wir \emph{reguläre} Geraden, die in \eqref{eq:ghost} \emph{Geistergeraden}. Die Anzahl der regulären Geraden ist $3d^2$, die der Geistergeraden $(d-3)d^3$.
\begin{proof}
Nach obiger Proposition haben wir neben \eqref{eq:grcond} genau die Gleichungen für $\binom{d}{d,0,0,0}=1$ und $\binom{d}{d-1,1,0,0}$ und Varianten. Der erste Fall führt auf die Gleichungen \eqref{eq:powers}, der zweite auf \eqref{eq:ratios}. Letztere in sich selbst eingesetzt liefern
\begin{equation*}
\frac{p_{12}}{p_{13}} = -\left(\frac{p_{03}}{p_{02}}\right)^{d-1} = -\left(-\left(\frac{p_{12}}{p_{13}}\right)^{d-1}\right)^{d-1} = \left(\frac{p_{12}}{p_{13}}\right)^{(d-1)^2}, \qquad\text{da $p$ ungerade,}
\end{equation*}
falls $p_{12}, p_{13} \neq 0$ und $p_{03} p_{02} \neq 0$. Verschwindet also keine der Plückerkoordinaten, so sind ihre Verhältnisse $k$-te Einheitswurzeln mit $k=(d-1)^2-1=d(d-2)$. Da die Plückerkoordinaten homogene Koordinaten sind, können wir annehmen, dass $p_{ij} \in \mu_{d(d-2)}$ für alle $i \neq j$.

Betrachten wir nun \eqref{eq:ratios} zur $d$-ten Potenz erhoben:
\begin{equation*}
\frac{p_{ik}^d}{p_{il}^d} = \left(\frac{p_{ik}}{p_{il}}\right)^d = \left(\frac{p_{jl}}{p_{jk}}\right)^{d(d-1)} = \left(\frac{p_{jl}}{p_{jk}}\right)^d = \frac{p_{jl}^d}{p_{jk}^d}.
\end{equation*}
Mit der Substitution $i \leftrightarrow k$, $j \leftrightarrow l$ erhalten wir
\begin{equation*}
\frac{p_{ik}^d}{p_{jk}^d} = \frac{p_{ki}^d}{p_{kj}^d} = \frac{p_{lj}^d}{p_{li}^d} = \frac{p_{jl}^d}{p_{il}^d}.
\end{equation*}
Durcheinander geteilt ergibt das
\begin{equation*}
\frac{p_{jk}^d}{p_{il}^d} = \frac{p_{il}^d}{p_{jk}^d} \qquad\text{bzw.}\qquad \frac{p_{il}^d}{p_{jk}^d} = \pm 1.
\end{equation*}

Setze $\mu = p_{01}^d$, $\eta = p_{02}^d$, $\nu = p_{03}^d$. Die Plückerkoordinaten in $d$-ter Potenz verhalten sich dann wie folgt:
{\vskip 2ex\hfil
\begin{tabular}{|c|c|c|c|} \hline
0 & $\mu$ & $\eta$ & $\nu$ \\ \hline
$\mu$ & 0 & $\pm \nu$ & $\pm \eta$ \\ \hline
$\eta$ & $\pm \nu$ & 0 & $\pm \mu$ \\ \hline
$\nu$ & $\pm \eta$ & $\pm \mu$ & 0 \\ \hline
\end{tabular}
\hfil\vskip 2ex}
Wegen \eqref{eq:powers} müssen die Summen über alle Zeilen und Spalten gleich sein. Eine leichte Überlegung ergibt dann, dass für alle $\pm$ nur $+$ infrage kommt. Steht an nur einer Stelle ein $-$, sei also beispielsweise $p_{12}=-\nu$, aber $p_{13}=\eta$. Dann ergibt eine Subtraktion der Gleichungen \eqref{eq:powers} für die ersten beiden Zeilen $\nu = -\nu$, also $\nu = 0$. Das ist ein Widerspruch. Steht an mindestens zwei Stellen ein $-$, sei also o.\,E. $p_{12}=-\nu$ und $p_{13}=-\eta$. Dann ergibt eine Addition der ersten beiden Zeilen, dass $2\mu = 0$, also $\mu = 0$. Auch das geht nicht.

Die Gleichung ist daher genau dann erfüllt, wenn $\mu+\eta+\nu = 0$ mit $\mu, \eta, \nu \in \mu_{d-2} = \mathbb F_{p^n}^*$. Ist $\zeta \in \mu_{d(d-2)}$ primitiv, so können wir $\mu = \zeta^{ad}$, $\eta = \zeta^{bd}$, $\nu = \zeta^{cd}$ schreiben. Nach dem vorigen Lemma gibt es genau $(p^n-1)(p^n-2)$ solche Tripel $(a,b,c) \in (\Zmod (d-2)Z)^3$. Wir liften sie nach $(\Zmod d(d-2)Z)^3$ und nennen sie $a_{01} = a_{23} \equiv a, a_{02} = a_{13} \equiv b$, $a_{03} = a_{12} \equiv c \pmod{d-2}$.

Damit haben die $p_{ij}$ die Form $\zeta^{a_{ij} + b_{ij}(d-2)}$ sind mit $b_{ij} \in \Zmod dZ$. Die Gleichungen \eqref{eq:ratios} werden dann zu
\begin{align*}
\frac{\zeta^{a_{ik} + b_{ik}(d-2)}}{\zeta^{a_{il} + b_{il}(d-2)}} = \frac{p_{ik}}{p_{il}} &= -\left(\frac{p_{jl}}{p_{jk}}\right)^{d-1} = -\left(\frac{\zeta^{a_{jl} + b_{jl}(d-2)}}{\zeta^{a_{jk} + b_{jk}(d-2)}}\right)^{d-1} \\
a_{ik} - a_{il} + (b_{ik} - b_{il})(d-2) &\equiv (a_{jl} - a_{jk} + (b_{jl} - b_{jk})(d-2))(d-1) + d(d-2)/2 &&\mod{d(d-2)} \\
(b_{ik} - b_{il} + b_{jl} - b_{jk})(d-2) &\equiv (a_{ik} - a_{il})(d-2) + d(d-2)/2 &&\mod{d(d-2)} \\
b_{ik} - b_{il} - b_{jk} + b_{jl} &\equiv a_{ik} - a_{il} + d/2 &&\mod d
\end{align*}
Man beachte dabei, dass $a_{ik} = a_{jl}$, $a_{il} = a_{jk}$. Weiterhin gilt $b_{ij} - b_{ji} \equiv d/2 \pmod d$ wegen $p_{ij} + p_{ji} = 0$. Wir müssen die Gleichung nicht für alle Permutationen testen, sondern wegen Symmetrie nur für die drei Quadrupel $(i,j,k,l) = (0,1,2,3), (0,2,1,3), (0,3,1,2)$.

Das ergibt ein inhomogenes Gleichungssystem in den $b_{ij} \in \Zmod dZ$, wir lösen sie aber zunächst im größeren $(\Zmod 2dZ)$-Modul $\frac 12 \Zmod dZ$. Dort können wir leicht eine Lösung angeben: für $i<j$ setze $b_{ij} = a_{ij}/2$ für $2 \mid i-j$ und $b_{ij} = a_{ij}/2 + d/4$ sonst. Es ist zu prüfen, ob für die drei Quadrupel die Gleichung erfüllt ist:
\begin{align*}
&(0,1,2,3): &(a_{02}/2) - (a_{03}/2+\tfrac d4) - (a_{12}/2+\tfrac d4) + (a_{13}/2) &\overrel{\equiv}^! a_{02} - a_{03} + d/2 \\
&(0,2,1,3): &(a_{01}/2+\tfrac d4) - (a_{03}/2+\tfrac d4) - (a_{12}/2-\tfrac d4) + (a_{23}/2+\tfrac d4) &\overrel{\equiv}^! a_{01} - a_{03} + d/2 \\
&(0,3,1,2): &(a_{01}/2+\tfrac d4) - (a_{02}/2) - (a_{13}/2+\tfrac d2) + (a_{23}/2-\tfrac d4) &\overrel{\equiv}^! a_{01} - a_{02} + d/2
\end{align*}

Nun zur Lösung des homogenen Gleichungssystems. Statt es über dem $\Zmod 2dZ$-Modul $\frac 12 \Zmod dZ$ zu lösen, kann man es auch als Gleichungssystem über dem Ring $\Zmod 2dZ$ selbst betrachten, indem man alle Koordinaten verdoppelt.
\begin{equation*}
\begin{pmatrix}
1 & 0 & -1 & -1 & 0 & 1 \\
0 & 1 & -1 & -1 & 1 & 0 \\
1 & -1 & 0 & 0 & -1 & 1
\end{pmatrix}
\begin{pmatrix}
b_{01} \\ b_{02} \\ b_{03} \\ b_{12} \\ b_{13} \\ b_{23}
\end{pmatrix}
= \underline{0}
\end{equation*}
Das ist äquivalent zu $b_{01} + b_{23} = b_{02} + b_{13} = b_{03} + b_{12} = k/2$, $k \in \Zmod 2dZ$. Die allgemeine Lösung mit Parametern $\alpha, \beta, \gamma, k \in \Zmod 2dZ$ ist daher:
{\vskip 2ex\hfil
\begin{tabular}{|c|c|c|c|} \hline
- & $(a+\alpha)/2+d/4$ & $(b+\beta)/2$ & $(c+\gamma)/2+d/4$ \\ \hline
$(a+\alpha)/2-d/4$ & - & $(c+k-\gamma)/2+d/4$ & $(b+k-\beta)/2$ \\ \hline
$(b+\beta)/2+d/2$ & $(c+k-\gamma)/2-d/4$ & - & $(a+k-\alpha)/2+d/4$ \\ \hline
$(c+\gamma)/2-d/4$ & $(b+k-\beta)/2+d/2$ & $(a+k-\alpha)/2-d/4$ & - \\ \hline
\end{tabular}
\hfil\vskip 2ex}
Um die Lösungen der Gleichung in $\Zmod dZ$ zu bekommen, schränken wir sie einfach ein. Das bedeutet $\alpha \equiv a+d/2 \equiv k-\alpha,\; \beta \equiv b \equiv k-\beta,\; \gamma \equiv c+d/2 \equiv k-\gamma \pmod 2$. Insbesondere gilt also $2 \mid k$, dies ist auch ausreichend für die Wahl von $k$. Die Parameter $\alpha$, $\beta$, $\gamma$ haben also die selbe Parität wie $a+d/2$, $b$, resp.~$c+d/2$. Die endgültigen Exponenten $a_{ij} + (d-2)b_{ij}$ sind damit: \note Oder wollen wir das untere Dreieck weglassen?
{\vskip 2ex\hskip 0pt plus 1fil minus 1.5cm
\begin{tabular}{|c|c|c|c|} \hline
- & $\frac{ad}2+\alpha\frac{d-2}2+\frac{d(d-2)}4$ & $\frac{bd}2+\beta\frac{d-2}2$ & $\frac{cd}2+\gamma\frac{d-2}2+\frac{d(d-2)}4$ \\ \hline
$\frac{ad}2+\alpha\frac{d-2}2-\frac{d(d-2)}4$ & - & $\frac{cd}2+(k-\gamma)\frac{d-2}2+\frac{d(d-2)}4$ & $\frac{bd}2+(k-\beta)\frac{d-2}2$ \\ \hline
$\frac{bd}2+\beta\frac{d-2}2+\frac{d(d-2)}2$ & $\frac{cd}2+(k-\gamma)\frac{d-2}2-\frac{d(d-2)}4$ & - & $\frac{ad}2+(k-\alpha)\frac{d-2}2+\frac{d(d-2)}4$ \\ \hline
$\frac{cd}2+\gamma\frac{d-2}2-\frac{d(d-2)}4$ & $\frac{bd}2+(k-\beta)\frac{d-2}2+\frac{d(d-2)}2$ & $\frac{ad}2+(k-\alpha)\frac{d-2}2-\frac{d(d-2)}4$ & - \\ \hline
\end{tabular}
\hfil\vskip 2ex}

Offenbar zählen wir dabei einige Geraden mehrfach. Möchte man eine eindeutige Zuordnung zwischen Parametern und Geraden erhalten, kann man beispielsweise $p_{02} = 1$ setzen. Damit ist $b = \beta = 0$. Man überzeugt sich leicht, dass damit Eindeutigkeit hergestellt ist.

Nun überprüfen wir noch, dass die Gleichung der \textsc{Grassmann}-Mannigfaltigkeit \eqref{eq:grcond} erfüllt ist:
\begin{align*}
0 &\overrel{=}^! \zeta^{\frac{ad}2+\alpha\frac{d-2}2+\frac{d(d-2)}4} \zeta^{\frac{ad}2+(k-\alpha)\frac{d-2}2+\frac{d(d-2)}4} - \zeta^{\frac{bd}2+\beta\frac{d-2}2} \zeta^{\frac{bd}2+(k-\beta)\frac{d-2}2} \\
  &\qquad + \zeta^{\frac{cd}2+\gamma\frac{d-2}2+\frac{d(d-2)}4} \zeta^{\frac{cd}2+(k-\gamma)\frac{d-2}2+\frac{d(d-2)}4} \\
  &= \zeta^{ad + k\frac{d-2}2 + \frac{d(d-2)}2} - \zeta^{bd + k\frac{d-2}2} + \zeta^{cd + k\frac{d-2}2 + \frac{d(d-2)}2} \\
  &= -\zeta^{k\frac{d-2}2}(\zeta^{ad} + \zeta^{bd} + \zeta^{cd})
\end{align*}
Der Term in Klammern verschwindet, also ist die Gleichung automatisch erfüllt. Nun gilt es noch, je zwei Punkte auf den Geraden zu finden. Das liefern Schnitte mit den Ebenen $X=0$ und $Y=0$: wir erhalten die Punkte $(0:p_{01}:p_{02}:p_{03})$, $(p_{10}:0:p_{12}:p_{13})$. Oder, mit den Substitutionen $\lambda = \zeta^{k \frac{d-2}2}$, $\eta = \zeta^{\alpha \frac{d-2}2}$, $\theta = \zeta^{\gamma\frac{d-2}2}$, weiterhin $\mu = \zeta^{ad/2}$, $\nu = \zeta^{cd/2}$ sowie $i = \zeta^{\frac{d(d-2)}4}$:
\begin{equation}
(0:\mu\eta i:1:\nu\theta i),\qquad
(-\mu\eta i:0:\nu\lambda i/\theta:\lambda).
\end{equation}
Die Bedingungen werden dann zu $\lambda \in \mu_d$, $\eta, \theta \in \mu_{2d}$ und $\mu, \nu \in \mu_{2(d-2)}$ mit $\mu^2 + 1 + \nu^2 = 0$, außerdem $(i\eta)^d = \mu^{d-2}$ und $(i\theta)^d = \nu^{d-2}$.

Wir sahen in Korollar~\ref{cor:projrootsum}, dass es für $(a,0,c) \in (\Zmod (d-2)Z)^3$ mit $\zeta^{ad} + 1 + \zeta^{cd} = 0$ genau $p^n-2$ Lösungen gibt. Analog gibt es $p^n-2$ Lösungen für $(\mu^2, 1, \nu^2)$, und von den zwei Wurzeln kann man sich je eine aussuchen. Für die verbleibenden drei Variablen $\lambda$, $\eta$ und $\theta$ gibt es dann jeweils $d=p^n+1$ Lösungen, da $\eta^d$ und $\theta^d$ durch die Wahl von $\mu$ und $\nu$ festgelegt sind. Weitere Beschränkungen gibt es nicht, daher kommen wir auf $(d-3)d^3$ bzw. $(p^n-2)(p^n+1)^3$ Geraden.

Nun zu dem Fall, dass Plückerkoordinaten verschwinden. Sei also $p_{ij}=0$, dann ist wegen~\eqref{eq:ratios}
\begin{align*}
p_{ij}^{d-1}p_{kj} + p_{il}^{d-1}p_{kl} = 0 \qquad\Rightarrow p_{il} = 0 \text{ oder } p_{kl} = 0, \\
p_{ij}^{d-1}p_{lj} + p_{ik}^{d-1}p_{lk} = 0 \qquad\Rightarrow p_{ik} = 0 \text{ oder } p_{kl} = 0, \\
p_{ji}^{d-1}p_{ki} + p_{jl}^{d-1}p_{kl} = 0 \qquad\Rightarrow p_{jl} = 0 \text{ oder } p_{kl} = 0, \\
p_{ji}^{d-1}p_{li} + p_{jk}^{d-1}p_{lk} = 0 \qquad\Rightarrow p_{jk} = 0 \text{ oder } p_{kl} = 0.
\end{align*}
Also gilt $p_{kl} = 0$ oder $p_{il} = p_{ik} = p_{jl} = p_{jk} = 0$. Im zweiten Fall verschwinden dann alle Plückerkoordinaten wegen \eqref{eq:powers}, der erste führt auf die regulären Geraden aus dem vorigen Fakt.
\end{proof}
\begin{remarks}
Es fehlt eine Betrachtung des Falles $\Char K = 2$. In diesem Fall ist $d$ ungerade, es ist also ein anderes Ergebnis zu erwarten. Vermutlich kommt man aber mit einer ähnlichen Vorgehensweise zum Ziel.
\end{remarks}

\section{Gemeinsame Darstellung}
Betrachtet man die letzte Tabelle in vorigem Beweis, so erkennt man vielleicht, dass die regulären Geraden als Spezialfall der Geistergeraden aufgefasst werden können. Mit den Substitutionen $\lambda_1 = \zeta^{ad/2}$, $\lambda_2 = \zeta^{bd/2}$, $\lambda_3 = \zeta^{cd/2}$, $i = \zeta^{d(d-2)/4}$, $\nu_1 = \zeta^{\alpha(d-2)/2}$, $\nu_2 = \zeta^{\beta(d-2)/2}$, $\nu_3 = \zeta^{\gamma(d-2)/2}$, $\theta = \zeta^{k(d-2)/2}$ erhält man folgende Verallgemeinerung, indem man für $\lambda_{1,2,3}$ auch den Wert $0$ zulässt.
\begin{coroll}
Sei $\Char K = p > 0$ und $d = q+1$, $q = p^n$. Dann liegen auf $F_d$ genau die Geraden mit den Plückerkoordinaten
\begin{align*}
p_{01} &= \lambda_1 \nu_1 i &p_{02} &= \lambda_2 \nu_2 &p_{03} &= \lambda_3 \nu_3 i \\
 & &p_{12} &= \lambda_3 \theta i / \nu_3 &p_{13} &= \lambda_2 \theta / \nu_2 \\
 & & & &p_{23} &= \lambda_1 \theta i / \nu_1
\end{align*}
mit den Parametern $(\lambda_{1:2:3}^2) \in \proj 2(\mathbb F_q)$, $\theta \in \mu_d$ und $\nu_{1,2,3} \in \mu_{2d}$ und den Bedingungen $\lambda_1^2 + \lambda_2^2 + \lambda_3^2 = 0$ und $(i^j \nu_j)^d = \lambda_j^{d-2}$ für $j=1,2,3$.
\end{coroll}
\begin{proof}
Die Plückerkoordinaten $(:\!p_{ij}) \in \mathbb P(\bigwedge^2 K^4)$ sind wohldefiniert, da die Parameter $(\lambda_{1:2:3})$ darin homogen eingehen. Es sind zwei Fälle zu unterscheiden: der Fall $\lambda_{1,2,3} \neq 0$ führt auf die Geistergeraden aus dem vorigen Satz, da $\mathbb F_q^* = \mu_{d-2}$. Offenbar erhalten wir so auch alle, wie sich aus den Substitutionen ergibt.

Verschwindet hingegen eines der $\lambda_j$, dann sind die beiden anderen verschieden von null und unterscheiden sich um einen Faktor $\pm i$, da sich ihre Quadrate um den Faktor $-1$ unterscheiden. Es genügt, eine der beiden Möglichkeiten zu untersuchen: die andere erhält man, indem man $\nu_{1/2/3}$ um den Faktor $-1$ abändert. Weiterhin können wir eines der beiden nicht verschwindenden $\lambda_j$ festlegen.

Für $\lambda_1 = 0$, $\lambda_2 = 1$ und $\lambda_3 = -i$ ergeben sich die regulären Geraden der Klasse~(I), für $\lambda_1 = 1$, $\lambda_2 = 0$ und $\lambda_3 = i$ ergeben sich die regulären Geraden der Klasse~(II), $\lambda_1 = -i$, $\lambda_2 = 1$ und $\lambda_3 = 0$ ergeben sich die regulären Geraden der Klasse~(III). Das sieht man so: die Situation der regulären Geraden liegt vor, denn zwei Plückerkoordinaten verschwinden. Die Verhältnisse der übrigen sind wie durch \eqref{eq:ratios} gefordert:
\begin{align*}
\mathrm{(I)}\qquad p_{02}/p_{12} &= \nu_2 \nu_3 / \theta = p_{03}/p_{13} &p_{02}/p_{03} &= \nu_2/\nu_3 = p_{12}/p_{13} \\
\mathrm{(II)}\qquad p_{01}/p_{21} &= \nu_1 \nu_3 i/ \theta = p_{03}/p_{23} &p_{01}/p_{03} &= -\nu_1 i/\nu_3 = p_{21}/p_{23} \\
\mathrm{(III)}\qquad p_{01}/p_{31} &= -\nu_1 \nu_2/\theta = p_{02}/p_{32} &p_{01}/p_{02} &= \nu_1/\nu_2 = p_{31}/p_{32}.
\end{align*}

Bleibt nur noch zu prüfen, dass die Verhältnisse hoch~$d$ gleich~$-1$ sind. Wir rechnen es hier nur für die Klasse~(I) nach, für die anderen folgt es analog:
\begin{align*}
\left( \frac{\nu_2 \nu_3}{\theta} \right)^d &= \frac{\nu_2^d \nu_3^d}{\theta^d} = \lambda_2^{d-2} \cdot i^d \lambda_3^{d-2} = i^d (-i)^{d-2} = i^{2d-2} = (-1)^{d-1} = -1\qquad\text{und} \\
\left( \frac{\nu_2}{\nu_3} \right)^d &= \frac{\nu_2^d}{\nu_3^d} = \frac{\lambda_2^{d-2}}{i^d \lambda_3^{d-2}} = i^{-2d+2} = -1,
\end{align*}
wenn man beachtet, dass $d$ gerade ist. Dass wir dabei alle regulären Geraden, soll wieder nur am Beispiel der Klasse~(I) gezeigt werden: wir wollen zeigen, dass $(\nu_2 \nu_3 / \theta, \nu_2 / \nu_3)$ den Bereich $(\mu_{2d} \setminus \mu_d) \times (\mu_{2d} \setminus \mu_d)$ durchläuft. Setze dazu $\nu_2$ auf einen beliebigen zugelassenen Wert. Setzt man nun für~$\nu_3$ alle $d$ zugelassenen Werte ein, dann nimmt $\nu_2 / \nu_3$ ebenfalls $d$ verschiedene Werte an. Wie wir gerade gesehen haben, sind diese aus $\mu_{2d} \setminus \mu_d$ und diese Menge hat genau~$d$ Elemente.

Wegen $\nu_3^2 \in \mu_d$ ist dann auch $\nu_2 \nu_3 \in \mu_{2d} \setminus \mu_d$. Wenn~$\theta$ wiederum alle Werte in~$\mu_d$ durchläuft, nimmt $\nu_2 \nu_3 / \theta$ unabhängig von der Wahl von~$\nu_3$ alle Werte in~$\mu_{2d} \setminus \mu_d$ an. Also werden tatsächlich alle regulären Geraden aufgezählt.
\end{proof}

\chapter{Symmetrien der Geradenkonfiguration} \label{chap:configsymm}
\section{Lineare und Kombinatorische Symmetrien}
Die Fermat-Flächen haben einen hohen Grad an Symmetrie. Offenbar lassen sowohl Permutationen der Koordinaten als auch Multiplikation der Koordinaten mit $d$-ten Einheitswurzeln die Fläche invariant. Damit operieren die Gruppen $S_4$ und $\mu_d^4$, letztere mit Stabilisator \note Heißt das so? isomorph zu $\mu_d$. Beide Gruppen schneiden sich nur in $\{\id\}$, also haben wir eine Aktion von
\begin{equation}
S_4 \ltimes \mu_d^4 / \mu_d \subset \PGL 4K,
\end{equation}
auf $F_d$, wobei der Homomorphismus $S_4 \rightarrow \Aut(\mu_d^4 / \mu_d)$ auf Permutationen der Koordinaten abbildet. In bestimmten Fällen gibt es sogar noch mehr Symmetrien, wie wir später sehen werden. Allgemein interessieren wir uns für Abbildungen aus~$\PGL 4K$ bzw.~$\PGaL 4K$, die die Fläche $F_d$ invariant lassen.\footnote{Für eine Theorie der linearen Gruppen, speziell der $\PGaL nK$, siehe \cite{Dieudonne}.}
\begin{defin}
Sei $S \in \proj n(K)$ eine beliebige Menge. Als lineare Symmetriegruppe von $S$ bezeichnen wir die Untergruppe
\begin{equation}
G_l = G_l(S) = \{ g \in \PGL{n+1}K: gS = S \}.
\end{equation}
\end{defin}

Solche linearen Abbildungen lassen nicht nur die Fläche invariant, sondern schicken auch Geraden auf Geraden. Damit permutieren sie die Geraden auf der Fläche, offenbar bleibt dabei aber ihre Schnittkonfiguration erhalten. Die Schnittkonfiguration wird durch einen Graphen $\mathcal G = (L,E)$ kodiert, dabei ist die $L$ die Menge der Geraden, und $(l_1, l_2) \in E$, wenn $l_1$ und $l_2$ sich schneiden.

Die Frage liegt nahe, welche Permutationen der Geraden es denn gibt, die die Schnittkonfiguration erhalten.
\begin{defin}
Sei nun $S \in \proj 3(K)$ eine projektive Fläche, $\mathcal G = (L,E)$ die Geradenkonfiguration. Dann ist die kombinatorische Symmetriegruppe von $S$ die Automorphismengruppe des Graphen $\mathcal G$, d.h.
\begin{equation}
G_k = G_k(S) = \{ \sigma \in \Sym(L): (l_1, l_2) \in E \Leftrightarrow (\sigma(l_1), \sigma(l_2)) \in E \}
\end{equation}
\end{defin}
Wir wollen in diesem Kapitel die Beziehung zwischen diesen beiden Gruppen ausarbeiten. Die folgende Betrachtung ist inspiriert durch \cite[Bem.~4.10.1, S.~404]{Hartshorne}, und \cite[Aufg. C--D, S.~180]{Mumford}. Dort wird die Situation für allgemeine reguläre Flächen dritten Grades untersucht.

Wie oben bemerkt, induziert jede lineare Symmetrie eine kombinatorische, also haben wir einen Homomorphismus $G_l(F_d) \to G_k(F_d)$.
\begin{prop}
Gibt es unter den Schnittpunkten der Geraden auf $S$ mindestens fünf in allgemeiner Lage,\footnote{das heißt: keine vier davon liegen auf einer projektiven Ebene.} so ist der Homomorphismus $G_l(S) \to G_k(S)$ injektiv.
\end{prop}
\begin{proof}
Sei $g \in \PGL 4K$ so gewählt, dass alle Geraden auf $S$ auf sich selbst abgebildet werden. Dann werden auch vier der Schnittpunkte auf sich selbst abgebildet. Sei $A \in \GL 4K$ ein Urbild von $g$, dann lässt also $A$ vier linear unabhängige eindimensionale Teilräume in $K^4$ invariant. Folglich hat $A$ in einer geeigneten Basis ${v_1, \dots v_4}$ des $K^4$ als Matrix Diagonalgestalt.

Sei $v_5 \in K^4 \setminus 0$ ein zugehöriger Vektor zum fünften Schnittpunkt. In der Basis $\{v_i\}_{i<5}$ hat $v_5$ dann die Gestalt $\sum_i \alpha_i v_i$ mit $\alpha_i \neq 0$ für alle $i$, da die fünf Schnittpunkte in allgemeiner Lage sind. Weil der fünfte Punkt von $g$ auf sich selbst geschickt wird, ist $A v_5 = \lambda v_5$ für ein $\lambda \in K^*$. Damit sind die Diagonaleinträge der Matrix alle gleich $\lambda$, da die $\alpha_i$ nicht verschwinden. Also ist $A = \lambda \id$ bzw.~$g = \id$.
\end{proof}
Wie wir weiter unten sehen werden, schneiden sich zwei Geraden einer Klasse in Punkten der Form $(1:\zeta^n:0:0)$, $\zeta \in \mu_{2d}$ primitiv und $n$ ungerade; sowie Permutationen der Koordinaten. Durch Probieren findet man damit leicht fünf Punkte in allgemeiner Lage. Die fünf $4 \times 4$-Untermatrizen von
\begin{equation*}
\begin{pmatrix}
1 & 0 & 1 & 0 & 1 \\
\zeta & 0 & 0 & 1 & 0 \\
0 & 1 & \zeta^3 & 0 & 0 \\
0 & \zeta & 0 & \zeta^5 & \zeta^3 \\
\end{pmatrix}
\end{equation*}
haben Determinanten $(\zeta^2-1)\zeta^4$, $(\zeta+1)\zeta^4$, $-(\zeta^3+1)\zeta^3$, $-(\zeta^3+1)\zeta^6$ bzw.~$(\zeta+1)\zeta^3$. Für $d>3$ verschwindet keine davon.

\section{Reguläre Geraden}
\paragraph{Konfiguration} Wir schreiben die Geraden aus \eqref{eq:regular} mit einer fixierten primitiven Einheitswurzel $\zeta \in \mu_{2d}$ und $a,b \in (2\mathbb Z + 1)/2d\mathbb Z \subset \Zmod 2dZ$:
\begin{equation}
\begin{split}
\Lcl(I)_{a,b}  :\qquad	&\langle (1,\zeta^a,0,0), (0,0,1,\zeta^b)\rangle \\
\Lcl(II)_{a,b} :\qquad	&\langle (1,0,\zeta^a,0), (0,1,0,\zeta^b)\rangle \\
\Lcl(III)_{a,b}:\qquad	&\langle (1,0,0,\zeta^a), (0,1,\zeta^b,0)\rangle.
\end{split}
\end{equation}
Ob sich zwei verschiedene projektive Geraden schneiden, stellt man anhand der Determinante der Matrix aus ihren vier Basisvektoren fest: verschwindet sie, dann hat die Matrix nicht vollen Rang, also schneiden sie sich in einem projektiven Punkt.

Damit überlegt man sich leicht, dass sich zwei Geraden aus derselben Familie genau dann schneiden, wenn sie in einem der beiden Parameter $a$, $b$ übereinstimmen. Für Geraden aus verschiedenen Klassen ergibt sich folgendes: zwei Geraden $\Lcl(I)_{a,b}$ und $\Lcl(II)_{a',b'}$ schneiden sich, wenn
\begin{equation}
\det \begin{pmatrix}
1 & \zeta^a & 0 & 0 \\
0 & 0 & 1 & \zeta^b \\
1 & 0 & \zeta^{a'} & 0 \\
0 & 1 & 0 & \zeta^{b'}
\end{pmatrix} = 0 \quad\Longleftrightarrow\quad \zeta^{a'} \zeta^b = \zeta^a \zeta^{b'} \quad\Leftrightarrow\quad a-b \equiv a'-b' \pmod{2d}.
\end{equation}
Analog erhält man, dass sich $\Lcl(I)_{a,b}$ und $\Lcl(III)_{a'',b''}$ schneiden, wenn $a''-b'' \equiv a+b$, und $\Lcl(II)_{a',b'}$ und $\Lcl(III)_{a'',b''}$, wenn $a''+b'' \equiv a'+b' \pmod{2d}$. \note Muss man das nachrechnen?

Die Schnittkonfiguration kann man sich also wie folgt vorstellen: Die Geraden teilen sich in drei Klassen, jede aus jeweils $d \times d$ Geraden bestehend. Ordnen wir die entsprechend in einer Matrix an, so schneiden sich alle Geraden in einer Zeile oder Spalte, die Zeilen und Spalten bilden also vollständige Graphen $K_d$. Zwischen den verschiedenen Klassen gibt es auch Schnitte: die Diagonalen verschiedener Klassen bilden einen bipartiten Graphen $K_{d,d}$. In der folgenden Grafik sind einige dieser Diagonalen dargestellt. Jede Gerade in einer der Diagonalen schneidet jede andere in der Diagonalen gleicher Farbe in der entsprechenden anderen Klasse. \note Ist das hilfreich?

\begin{figure}[h]
\centering
\begin{tikzpicture}
\draw[color=gray]
	[xshift=0cm] (0,0) rectangle (2,2) (1,1) node{$\Lcl(I)$}
	[xshift=3cm] (0,0) rectangle (2,2) (1,1) node{$\Lcl(II)$}
	[xshift=3cm] (0,0) rectangle (2,2) (1,1) node{$\Lcl(III)$};
\draw[->,xshift=0cm] (0,2.2) -- (2,2.2) node[above,midway] {$a$};
\draw[->,xshift=0cm] (-0.2,2) -- (-0.2,0)  node[left,midway] {$b$};
\draw[->,xshift=3cm] (0,2.2) -- (2,2.2) node[above,midway] {$a'$};
\draw[->,xshift=3cm] (-0.2,2) -- (-0.2,0)  node[left,midway] {$b'$};
\draw[->,xshift=6cm] (0,2.2) -- (2,2.2) node[above,midway] {$a''$};
\draw[->,xshift=6cm] (-0.2,2) -- (-0.2,0)  node[left,midway] {$b''$};
\draw[color=red]
	[xshift=0cm] (0.5,0) -- (0,0.5) (0.5,2) -- (2,0.5)
	[xshift=3cm] (0.5,0) -- (0,0.5) (0.5,2) -- (2,0.5);
\draw[color=green]
	[xshift=3cm] (1.3,0) -- (2,0.7) (0,0.7) -- (1.3,2)
	[xshift=3cm] (1.3,0) -- (2,0.7) (0,0.7) -- (1.3,2);
\draw[color=blue]
	[xshift=0cm] (0,1.7) -- (0.3,2) (0.3,0) -- (2,1.7)
	[xshift=6cm,yscale=-1,yshift=-2cm]
		(0,1.7) -- (0.3,2) (0.3,0) -- (2,1.7);
\end{tikzpicture}
\caption{Schnitte zwischen verschiedenen Klassen der regulären Geraden}\label{fig:reg}
\end{figure}

\paragraph{Symmetrien} Wir haben uns bereits klargemacht, dass wir auf $F_d$ eine Aktion der Gruppe $S_4 \ltimes \mu_d^4 / \mu_d$ haben, die nach obiger Proposition auch auf der Geradenkonfiguration transitiv operiert. Machen wir uns zunächst klar, auf welche Weise das geschieht. Multiplikation der $i$-ten Koordinate mit $\zeta^{s_i}$, $s_i$ gerade, bewirkt offenbar eine Verschiebung der Konfiguration in den einzelnen Klassen, und zwar wie folgt:

\begin{figure}[h]
\centering
\begin{tikzpicture}
\draw[color=black]
	[xshift=0cm] (0,0) rectangle (2,2) (1,1) node{$\Lcl(I)$}
	[xshift=4cm] (0,0) rectangle (2,2) (1,1) node{$\Lcl(II)$}
	[xshift=4cm] (0,0) rectangle (2,2) (1,1) node{$\Lcl(III)$};
\draw[->>,xshift=0cm] (0.5,2.2) -- (1.5,2.2) node[above,midway] {$s_0$};
\draw[->>,xshift=0cm] (-0.2,1.5) -- (-0.2,0.5) node[left,midway] {$s_2$};
\draw[->>,xshift=0cm] (1.5,-0.2) -- (0.5,-0.2) node[below,midway] {$s_1$};
\draw[->>,xshift=0cm] (2.2,0.5) -- (2.2,1.5) node[right,midway] {$s_3$};

\draw[->>,xshift=4cm] (0.5,2.2) -- (1.5,2.2) node[above,midway] {$s_0$};
\draw[->>,xshift=4cm] (-0.2,1.5) -- (-0.2,0.5) node[left,midway] {$s_1$};
\draw[->>,xshift=4cm] (1.5,-0.2) -- (0.5,-0.2) node[below,midway] {$s_2$};
\draw[->>,xshift=4cm] (2.2,0.5) -- (2.2,1.5) node[right,midway] {$s_3$};

\draw[->>,xshift=8cm] (0.5,2.2) -- (1.5,2.2) node[above,midway] {$s_0$};
\draw[->>,xshift=8cm] (-0.2,1.5) -- (-0.2,0.5) node[left,midway] {$s_1$};
\draw[->>,xshift=8cm] (1.5,-0.2) -- (0.5,-0.2) node[below,midway] {$s_3$};
\draw[->>,xshift=8cm] (2.2,0.5) -- (2.2,1.5) node[right,midway] {$s_2$};
\end{tikzpicture}
\caption{Aktion von $\mu_d^4 / \mu_d$ auf der Geradenkonfiguration}\label{fig:coordmult}
\end{figure}

Die Aktion der Koordinatenpermutation ist schwieriger zu beschreiben. Man kann die drei Matrizen wie folgt auf einen Würfel kleben:

\begin{figure}[h] % we might need transform canvas={...} or sloping and slanting node text (http://tex.stackexchange.com/questions/62038/text-placed-in-pespective-on-3d-object)
\centering
\begin{tikzpicture}[x  = {(0.9659cm,0.25882cm)},
                    y  = {(-0.5cm,0.5cm)},
                    z  = {(0cm,1cm)}, scale = 2]
\begin{scope}[canvas is yx plane at z=0]
  \path[draw=black] (0,0) rectangle (2,2);
\end{scope}
\begin{scope}[canvas is zx plane at y=0]
  \path[draw=black] (0,0) rectangle (2,2);
\end{scope}
\begin{scope}[canvas is zy plane at x=0]
  \path[draw=black] (0,0) rectangle (2,2);
\end{scope}
\begin{scope}[canvas is yx plane at z=2]
  \path[draw=black] (0,0) rectangle (2,2);
\end{scope}
\begin{scope}[canvas is zx plane at y=2]
  \path[draw=black] (0,0) rectangle (2,2);
\end{scope}
\begin{scope}[canvas is zy plane at x=2]
  \path[draw=black] (0,0) rectangle (2,2);
\end{scope}
\end{tikzpicture}
\caption{Visualisierung der Aktion von $S_4$ auf der Geradenkonfiguration}\label{fig:perm}
\end{figure}

Dann entspricht die Aktion der $S_4$ der Drehgruppe des Würfels. \todo Hier fehlt noch einiges.

\section{Geistergeraden}
\paragraph{Konfiguration}
\paragraph{Symmetrien}
% Betrachtung der irregulären Situation

\printbibliography

\appendix
% Eidessattliche Erklärung, Danksagung etc.

\end{document}
