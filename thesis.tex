\documentclass[a4paper, BCOR=1cm, DIV=12,
	headsepline=true, numbers=noenddot]{scrreprt}

% Standardpakete
\usepackage[ngerman]{babel}
\usepackage[utf8]{inputenc}
\usepackage[T1]{fontenc}

% Mathematik
\usepackage{amsmath,amssymb,amsthm}

% Sonstiges
\usepackage{tikz}
\usetikzlibrary{3d}
\usepackage[shortcuts]{extdash}

% Literaturverzeichnis
\usepackage[backend=biber, bibstyle=authoryear, citestyle=authoryear,
	natbib=false, abbreviate=false]{biblatex}
\bibliography{thesis}

% KOMA-Script Optionen
% Kopf- und Fußzeile
\pagestyle{headings}
\setkomafont{pageheadfoot}{\normalfont\scshape}
\deffootnote[1em]{0.5em}{1em}{\textsuperscript{\thefootnotemark}}
\setkomafont{captionlabel}{\sffamily\bfseries}
% Erlaube Zeilenumbrüche in AMS-Umgebungen
\allowdisplaybreaks[2]

% Theoreme etc.
\newtheorem*{defin}{Definition}
\newtheorem{theorem}{Satz}[chapter]
\newtheorem{lemma}[theorem]{Lemma}
\newtheorem{fact}[theorem]{Fakt}
\newtheorem*{prop}{Proposition}
\newtheorem{coroll}[theorem]{Korollar}
\newtheorem*{remarks}{Bemerkung}

% Allgemein
\def\note#1?{\textbf{#1?}\marginpar{$\leftarrow$}}
\def\todo#1.{\noindent\textsc{Todo}: \textbf{#1}.\marginpar{$\longleftarrow$}\par}
\def\wunderbrace#1_#2{\underbrace{#1}_\text{\makebox[0pt]{#2}}}
\def\overrel#1^#2{\mathrel{\mathop{#1}\limits^{#2}}}

% Mathematik
\def\to{\rightarrow}
\def\into{\hookrightarrow}
\def\mat#1{\mathbf{#1}}
\def\aff#1{\mathbb A^{#1}}
\def\proj#1{\mathbb P^{#1}}

\def\GL#1#2{\mathrm{GL}(#1,#2)}
\def\GaL#1#2{\mathrm{\Gamma L}(#1,#2)}
\def\PGL#1#2{\mathrm{PGL}(#1,#2)}
\def\PGaL#1#2{\mathrm{P\Gamma L}(#1,#2)}
\def\grass#1/#2{\mathrm{Gr}(#1,#2)}
\def\Zmod#1Z{\mathbb Z/#1\mathbb Z}
\def\Gal#1/#2{\mathrm{Gal}(#1/#2)}
\def\Lcl(#1){L^\text{(#1)}}

\DeclareMathOperator{\sign}{sign}
\DeclareMathOperator{\ord}{ord}
\DeclareMathOperator{\id}{id}
\DeclareMathOperator{\Stab}{Stab}
\DeclareMathOperator{\Char}{char}
\DeclareMathOperator{\Aut}{Aut}
\DeclareMathOperator{\Sym}{Sym}
\DeclareMathOperator{\im}{im}

\begin{document}
% Titel
\tableofcontents

% Kapitel als einzelne Dateien inkludieren
\chapter{Grundlagen} \label{chap:prelim}
Wir setzen grundlegende Kenntnisse der Algebra und der klassischen algebraischen Geometrie voraus. Einige nicht so bekannte Konzepte sollen aber im Folgenden vorgestellt werden. Wir arbeiten hier über algebraisch abgeschlossenen Körpern $K$, können also die Ergebnisse der klassischen algebraischen Geometrie voraussetzen. Diese kann man beispielsweise in \cite{Shafarevich} oder \cite{Mumford} nachlesen.

\paragraph{Semilineare Abbildungen} Zwischen Vektorräumen $V$, $W$ über einem Körper $K$ gibt es eine natürliche Klasse von Morphismen, die $K$-linearen Abbildungen. Eine Verallgemeinerung erhalten wir durch \emph{semilineare} Abbildungen: sei $\sigma \in \Aut K$, dann heißt $f: V \to W$ semilinear (relativ zu $\sigma$), wenn $f(a+b) = f(a) + f(b)$ und $f(\lambda a) = \sigma(\lambda)f(a)$ für alle $\lambda \in K$ und $a, b \in V$ gilt. Fixiert man in $V$ eine Basis, so lässt sich jede semilineare Abbildung $V \to W$ als Komposition der Anwendung eines Körperautomorphismus auf alle Komponenten und einer anschließenden linearen Abbildung schreiben.\footcite[Das folgt leicht aus den Bemerkungen in][S.~2--3]{Dieudonne} Die Fortsetzungen von $\sigma \in \Aut K$ auf $K^n$ bzw.~$\mathrm{Mat}(n \times n, K) \cong K^{n \times n}$ durch Anwendung auf alle Komponenten wollen wir ebenfalls $\sigma$ nennen.

Analog zur Gruppe der invertierbaren Endomorphismen eines $n$-dimensionalen Vektorraums $\GL nK$ kann man die Gruppe der invertierbaren semilinearen Abbildungen $\GaL nK$ definieren. Nach obiger Bemerkung lässt sich diese Gruppe schreiben als
\begin{equation}
\GaL nK = \GL nK \rtimes \Aut K \quad\text{mit Multiplikation } (\mat A, \sigma)(\mat B, \tau) = (\mat A \sigma(\mat B), \sigma \tau).
\end{equation}
Die Operation auf $V$ ist dann gegeben durch $(\mat A, \sigma)v = \mat A \sigma(v)$. Die Konstruktion überträgt sich auf den projektiven Fall, die Gruppe der semilinear-projektiven Transformationen heißt entsprechend $\PGaL nK$.

\paragraph{Unitäre Gruppen} Auf ähnliche Weise können wir nun Sesquilinearformen $b: V \times V \to K$ relativ zu $\sigma \in \Aut K$ über beliebigen Körpern definieren. Das klassische Beispiel ist $K = \mathbb C$ mit $\sigma$ als der komplexen Konjugation. Man verlangt:
\begin{itemize}
\item $b(x, \cdot)$ ist linear für alle $x \in V$,
\item $b(\cdot, y)$ ist semilinear relativ zu $\sigma$ für alle $y \in V$ und
\item $b(x,y) = \sigma(b(y,x))$ für alle $x,y \in V$.
\end{itemize}
Offenbar folgt die zweite Eigenschaft aus den anderen beiden. Die dritte Eigenschaft liefert $\sigma^2 = \id$ auf dem Bild von $b$, also ist $\sigma$ Involution für $b \neq 0$. Sei $b$ nicht entartet, d.h. von der Form $b(x,y) = \sigma(y^\top) \mat M x$ mit $\mat M \in \GL nK$, dann können wir unitäre Abbildungen bzw.~Matrizen bezüglich $b$ definieren. Für $\mat M = \id$ bilden diese die unitäre Gruppe
\begin{equation}
\Unit nK = \{ \mat A \in \GL nK: \sigma(\mat A^\top) \mat A = \id \}.
\end{equation}
Das ist die Gruppe der $b$-invarianten linearen Transformationen: sei $\mat A \in \GL nK$ mit $b(x,y) = b(\mat Ax, \mat Ay)$, d.\,h. $\sigma(y^\top) x = \sigma(\mat Ay)^\top (\mat Ax) = \sigma(y^\top) (\sigma(\mat A^\top) \mat A) x$ für alle $x, y \in V$, dann folgt $\id = \sigma(\mat A^\top) \mat A$.

Die entsprechende Sesquilinearform schreiben wir $\langle x, y \rangle = \sigma(y^\top) x$. Wir haben damit eine Form $Q: K^n \to K^\sigma: x \mapsto \langle x, x \rangle$, die wir Norm nennen wollen. Wie im komplexen Fall gibt es auch für $\sigma \neq \id$ eine Polarisierungsformel: $K$ ist dann eine \textsc{Galois}-Erweiterung des Fixpunktkörpers $K^\sigma$ vom Grad $[K:K^\sigma] = 2$. Damit haben wir eine Norm $\norm$ und eine Spur $\tr \colon K \to K^\sigma$. Fixiere nun eine $K^\sigma$-Basis $\{a_1, a_2\}$ von $K$. Dann gilt
\begin{equation*}
Q(a_i v + w) = \norm(a_i) Q(v) + \tr(a_i \langle v, w \rangle) + Q(w)
\end{equation*}
für $i = 1,2$. Stellt man diese nach dem Spurterm um und beachtet, dass die Spurform separabler Erweiterungen nicht entartet ist,\footcite[S.~199, Satz~7]{Bosch} kann man $\langle v, w \rangle$ durch Normen ausdrücken.
\begin{fact} \label{fact:norminv}
Die unitäre Gruppe $\Unit nK$ ist die Gruppe der linearen Abbildungen $K^n \to K^n$, die die Form $Q: K^n \to K^\sigma, (x_1, \dots, x_n) \mapsto \sum_i \sigma(x_i) x_i$ invariant lässt.
\end{fact}
\begin{proof}
Offenbar lässt jede unitäre Abbildung die Norm invariant. Betrachte nun eine lineare Abbildung, die die Norm invariant lässt. Nach der Polarisierungsformel lässt eine solche Abbildung auch das Skalarprodukt $K^n \times K^n \to K$ invariant, mithin ist sie unitär.
\end{proof}

\paragraph{Äußere Produkte} Zur Beschreibung von Untervektorräumen eines Vektorraums werden wir im nächsten Kapitel äußere Produkte benötigen. So wie das Tensorprodukt eine universelle Bilinearform liefert, erhalten wir durch das äußere Produkt eine universelle alternierende Form. Eine multilineare Form heißt alternierend, wenn sie bei zwei übereinstimmenden Argumenten verschwindet.

\begin{defin}
Sei $R$ (kommutativer) Ring und $V$ Modul über $R$. Das $n$-fache äußere Produkt $V^n \to \Lambda^n V$, $(v_1, \dots, v_n) \mapsto v_1 \wedge \dots \wedge v_n$, ist dann eine alternierende Abbildung mit der folgenden universellen Eigenschaft: sei $W$ ein $R$-Modul und $\phi: V^n \to W$ eine alternierende Form, dann gibt es eine eindeutige lineare Abbildung $\pi: \Lambda^n V \to A$ mit der Eigenschaft $\phi(v_1, \dots, v_n) = \pi(v_1 \wedge \dots \wedge v_n)$.
\end{defin}

Ist $R$ Körper und $V$ endlichdimensional, dann kann man die Dimension des äußeren Produkts direkt angeben: hat $V$ die Basis $(e_i)_{1 \leq i \leq n}$, dann hat $\Lambda_k V$ die Basis $(e_{i_1} \wedge \dots \wedge e_{i_k})_{1 \leq i_1 \leq \dots \leq i_k \leq n}$. Folglich ist $\dim \Lambda^k V = \binom nk$, insbesondere gilt $\Lambda^k V = 0$ für $k > n$.

Die direkte Summe $\bigoplus_{n=0}^\infty \Lambda^n V$ heißt dann äußere Algebra $\Lambda(V)$, man erhält sie auch als Quotient der Tensoralgebra $T(V) = \bigoplus_{n=0}^\infty \bigotimes^n V$ nach dem Ideal erzeugt von allen Elementen der Form $v \otimes v$, $v \in V$. Sie wird gradierte Algebra durch die Operationen $\wedge: \Lambda_i V \times \Lambda_j V \to \Lambda_{i+j} V$, die sich auf die Algebra fortsetzen. Für endlich-dimensionales~$V$ ist sie auch endlich-dimensional, es gilt $\dim \Lambda(V) = 2^{\dim V}$.

\chapter{Allgemeine Untersuchungen} \label{chap:general}
\section{Die \textsc{Grassmann}-Mannigfaltigkeit} \label{sec:grassmann}
Die \textsc{Grassmann}-Mannigfaltigkeit~$\grass r/n$ ist die Mannigfaltigkeit der $r$\-/dimensionalen Untervektorräume eines $n$\-/dimensionalen Vektorraums über einem Körper~$K$. Ihre Struktur ergibt sich wie folgt: die Gruppe der allgemeinen linearen Transformationen~$\GL nK$ operiert auf den $r$-dimensionalen Unterräumen von~$K^n$ transitiv. Der Stabilisator jedes Unterraums ist isomorph zu $\GL rK \times K^{r(n-r)} \times \GL{n-r}K$, damit ergibt sich
\begin{equation}
\grass r/n = \GL nK / (\GL rK \times K^{r(n-r)} \times \GL{n-r}K)
\end{equation}
und wegen $\dim \GL nK = n^2$ ist $\dim \grass r/n = r(n-r)$. Das folgt aus dem Satz über die Faserdimension für die kanonische Projektion $\GL nK \to \grass r/n$.

Man kann die Grassmann-Mannigfaltigkeit als projektive Varietät im $r$\-/fachen äußeren Produkt $\mathbb P(\Lambda^r K^n)$ auffassen: wir haben eine Abbildung $(K^n)^r \to \Lambda^r K^n \to \mathbb P(\Lambda^r K^n)$, die eine Basis $(v_1, \dots, v_r)$ eines Unterraums auf $K^*(v_1 \wedge \dots \wedge v_r)$ schickt.\footcite[siehe hierzu auch][S.~42]{Shafarevich} Man überlegt sich leicht, dass das Bild eines Unterraums unabhängig von der Wahl der Basis ist: ist $(w_1, \dots, w_r)$ eine andere Basis gegeben durch $w_j = \sum_i a_{ij} v_i$ mit einer geeigneten Matrix $\mat A = (a_{ij}) \in \GL nK$, dann ist
\begin{align*}
\Lambda_j w_j &= \Lambda_j \sum_i a_{ij} v_i = \sum_{\sigma: \{1,\dots,r\} \rightarrow \{1,\dots,r\}} \Lambda_j a_{\sigma(j)j} v_{\sigma(j)} \\
	&= \sum_{\sigma: \{1,\dots,r\} \rightarrow \{1,\dots,r\}} \prod_j a_{\sigma(j)j} \wunderbrace{\Lambda_j v_{\sigma(j)}}_{$= 0$, falls $\sigma$ nicht injektiv} = \sum_{\sigma \in S_r} \prod_j a_{\sigma(j)j} \Lambda_j v_{\sigma(j)} \\
	&= \sum_{\sigma \in S_r} \sign \sigma \prod_j a_{\sigma(j)j} \Lambda_j v_j = \det \mat A \cdot \Lambda_j v_j
\end{align*}
Das induziert eine Einbettung $\grass r/n \into \mathbb P(\Lambda^r K^n)$: seien $U, W \subset K^n$ verschiedene Unterräume, ihre Basen seien $(u_1, \dots, u_r)$ bzw.~$(w_1, \dots, w_r)$. Mit~$x \in U \setminus W$ gilt dann
\begin{equation*}
u_1 \wedge \dots \wedge u_r \wedge x = 0, \qquad w_1 \wedge \dots \wedge w_r \wedge x \neq 0.
\end{equation*}
Also sind $\Lambda_j u_j$ und $\Lambda_j w_j$ in $\mathbb P(\Lambda^r K^n)$ auch verschieden.

Sei $(e_1, \dots, e_n)$ die Standardbasis des $K^n$, dann ist $(e_{i_1} \wedge \dots \wedge e_{i_r})_{i_1 < \dots < i_r}$ eine Basis von~$\Lambda^r K^n$. Die Koordinaten~$(p_{i_1 \dots i_r})$ eines Punktes nennt man dann \textsc{Plücker}\-/Koordinaten.

Die Abbildung ist aber nicht surjektiv, nicht alle Elemente des äußeren Produkts entstehen durch eine Basis eines $r$\-/dimensionalen Teilraums. Das Bild ist vielmehr eine Untervarietät gegeben durch die Gleichungen\footcite[siehe][S.~42]{Shafarevich}
\begin{equation} \label{eq:grcond}
\sum_{t=1}^{r+1} (-1)^t p_{i_1 \dots i_{r-1} j_t} p_{j_1 \dots \hat{j_t} \dots j_{r+1}} = 0 \quad\text{für } i_1 < \dots < i_{r-1}, j_1 < \dots < j_{r+1}.
\end{equation}
Wir wollen nun zeigen, dass die Grassmann-Mannigfaltigkeit diesen Namen verdient. Dazu untersuchen wir zunächst die darauf operierende Gruppe~$\GL nK$.

\begin{lemma}
Für jede natürliche Zahl $n$ und jeden Körper $K$ ist die quasiprojektive Varietät~$\GL nK$ glatt und irreduzibel.
\end{lemma}
\begin{proof}
Glattheit ist offensichtlich, da $\GL nK$ algebraische Gruppe ist. Für die andere Aussage zeigen wir zunächst, dass $\SL nK$ und $K^*$ irreduzibel sind: $K^*$ ist birational isomorph zu der irreduziblen Varietät $\{ XY = 1 \} \subset \aff 2$ durch die Projektion $(x,y) \mapsto x$ mit Inversem $x \mapsto (x,1/x)$.

Zur Irreduzibilität von $\SL nK = \{\det \mat A = 1\}$: sei $(\mathrm{det}-1)=fg \in K[(X_{ij})]$ eine Faktorisierung, die konstanten Koeffizienten von $f$ und $g$ sind $\neq 0$. Treten in beiden Polynomen Monome höherer Grade auf, so treten in ihrem Produkt Monome mindestens dreier verschiedener Grade auf. Da alle Monome in der Determinante aber den gleichen Grad haben, kann das nicht sein, also ist $f$ oder $g$ konstant.

Nun haben wir eine reguläre Abbildung $\GL nK \to \SL nK$ definiert durch $\mat A \mapsto \mat (\!\det \mat A)^{-1} \mat A$. Diese Abbildung ist surjektiv mit irreduziblem Bild, die Fasern sind isomorph zu $K^*$ und damit auch irreduzibel. Nach Theorem~8 aus \cite[S.~77]{Shafarevich} ist damit auch $\GL nK$ irreduzibel.
\end{proof}

\begin{fact} \label{fact:grassirred}
Die Grassmann-Mannigfaltigkeit ist glatt und irreduzibel, also tatsächlich eine Mannigfaltigkeit.
\end{fact}
\begin{proof}
Wie oben erwähnt, operiert $\GL nK$ transitiv auf $\grass r/n$. Wäre $\grass r/n$ reduzibel, würde daher auch $\GL nK$ in mehrere Komponenten zerfallen---als Urbilder unter $\GL nK \to \grass r/n$.

Glattheit folgt ähnlich: da eine Gruppe transitiv operiert, muss $\grass r/n$ überall glatt sein.
\end{proof}

\section{Projektive Unterräume auf projektiven Hyperflächen} \label{sec:linesproj}
Wir wollen nun untersuchen, wann auf Hyperflächen projektive Unterräume liegen. Dazu verallgemeinern wir die Vorgehensweise für Geraden auf kubische Flächen.\footcite[siehe][S.~78ff]{Shafarevich} Als Hyperfläche vom Grad $d$ bezeichnen wir die Nullstellenmenge eines homogenen Polynoms vom Grad $d$. Alle Komponenten einer solchen Hyperfläche haben Codimension 1, wie aus der Literatur bekannt.\footcite[siehe][S.~74, Theorem~4]{Shafarevich}

Sei im Folgenden $n$ die Dimension des Raums, $r$ die Dimension der gesuchten Unterräume. Die homogenen Formen vom Grad $d$ in $n+1$ Variablen bilden einen Raum $H_{n+1}^d$ der Dimension $\binom{n+d}{n}$: sind $X_0, \dots, X_n$ die Variablen, so bilden $X_0^{d_0} \dots X_n^{d_n}$ mit $d_0 + \dots + d_n = d$, $d_i \geq 0$ eine Basis. Die projektiven Hyperflächen vom Grad $d$ in einem projektiven Raum der Dimension $n$ bilden also eine Varietät isomorph zu $\proj{\binom{n+d}{n} - 1}$.

Projektive Unterräume der Dimension $r$ in $\proj n$ korrespondieren zu Untervektorräumen der Dimension $r+1$ in $\aff{n+1}$. Also bilden sie eine Varietät isomorph zu $\grass r+1/{n+1}$.

Wir definieren nun eine Teilmenge $\Gamma_{r,n}^d$ im Produkt $\mathbb P(H_{n+1}^d) \times \grass r+1/{n+1}$:
\begin{equation}
\Gamma_{r,n}^d = \{(F,L) \in \mathbb P(H_{n+1}^d) \times \grass r+1/{n+1} \colon \text{$L$ liegt auf der durch $F$ definierten Varietät} \}.
\end{equation}

\begin{fact} \label{fact:gammaproj}
Die Menge $\Gamma_{r,n}^d$ ist eine projektive Varietät.
\end{fact}
\begin{proof}
Es sind also zunächst zur verbalen Beschreibung äquivalente algebraische Gleichungen zu finden. Betrachten wir dazu einen $(r+1)$-dimensionalen Unterraum~$W$ in~$\aff{n+1}$ mit Basis $(v_i)_{0 \leq i \leq r}$, die einzelnen Komponenten eines Basisvektors~$v_i$ mögen $v_{ij}$, $0 \leq j \leq n$, heißen. Dann gilt für die Plückerkoordinaten des Unterraums:
\begin{equation}
p_{i_0 \dots i_r} = \sum_{\sigma \in S_{r+1}} \sign \sigma \cdot v_{0i_{\sigma(0)}} \dots v_{ri_{\sigma(r)}},
\end{equation}
wobei $S_{r+1}$ die Permutationsgruppe von $\{0,\dots,r\}$ ist. Wie lässt sich nun der Unterraum aus den Plückerkoordinaten rekonstruieren? Zunächst macht man sich klar, dass sich $W$ schreiben lässt als
\begin{equation}
W = \left\{ \sum\limits_{\sigma \in S_{r+1}} \sign \sigma \cdot \phi(v_{\sigma(1)}, \dots, v_{\sigma(r)}) v_{\sigma(0)} \;\middle|\; \phi \colon (K^{n+1})^r \to K \text{ multilinear} \right\}.
\end{equation}
Eine solche Multilinearform $\phi \colon (K^{n+1})^r \to K$ wiederum hat die Form
\begin{equation}
\phi(x_1, x_2, \dots, x_r) = \sum_{\iota: \{1, \dots, r\} \to \{0, \dots, n\}} \alpha_{\iota(1) \dots \iota(r)} \langle x_1, e_{\iota(1)} \rangle \dots \langle x_r, e_{\iota(r)} \rangle
\end{equation}
mit Koeffizienten $\alpha_{i_1 \dots i_r} \in K$. Sei nun $w \in W$, wir setzen die Multilinearform oben ein:
\begin{align*}
w &= \sum\limits_{\sigma \in S_{r+1}} \sign \sigma \sum_{\iota: \{1, \dots, r\} \to \{0, \dots, n\}} \alpha_{\iota(1) \dots \iota(r)} \langle v_{\sigma(1)}, e_{\iota(1)} \rangle \dots \langle v_{\sigma(r)}, e_{\iota(r)} \rangle v_{\sigma(0)} \\
	&= \sum\limits_{\sigma \in S_{r+1}} \sign \sigma \sum_{\iota: \{1, \dots, r\} \to \{0, \dots, n\}} \alpha_{\iota(1) \dots \iota(r)} v_{\sigma(1)\iota(1)} \dots v_{\sigma(r)\iota(r)} \sum_{j=0}^n v_{\sigma(0)j} e_j \\
	&= \sum_{j=0}^n \left(\sum\limits_{\sigma \in S_{r+1}} \sign \sigma \sum_{\iota: \{1, \dots, r\} \to \{0, \dots, n\}} \alpha_{\iota(1) \dots \iota(r)} v_{\sigma(1)\iota(1)} \dots v_{\sigma(r)\iota(r)} v_{\sigma(0)j}\right) e_j \\
	&= \sum_{j=0}^n \left(\sum\limits_{\sigma \in S_{r+1}} \sign \sigma \sum_{\substack{\iota: \{0, \dots, r\} \to \{0, \dots, n\} \\ \iota(0)=j }} \alpha_{\iota(1) \dots \iota(r)} v_{\sigma(0)\iota(0)} \dots v_{\sigma(r)\iota(r)}\right) e_j \\
	&= \sum_{j=0}^n \left(\sum_{\substack{\iota: \{0, \dots, r\} \to \{0, \dots, n\} \\ \iota(0)=j }} \alpha_{\iota(1) \dots \iota(r)} \sum\limits_{\sigma \in S_{r+1}} \sign \sigma \cdot v_{0\iota(\sigma^{-1}(0))} \dots v_{r\iota(\sigma^{-1}(r))}\right) e_j \\
\Rightarrow\quad w_j	&= \sum_{\substack{\iota: \{0, \dots, r\} \to \{0, \dots, n\} \\ \iota(0)=j }} \alpha_{\iota(1) \dots \iota(r)} \sum\limits_{\sigma \in S_{r+1}} \sign \sigma \cdot v_{0(\iota\circ\sigma)(0)} \dots v_{r(\iota\circ\sigma)(r)}.
\end{align*}
Die innere Summe verschwindet für nicht injektive $\iota$. Gilt nämlich $\iota(x) = \iota(y)$ für $x \neq y$, so ersetze man $\sigma$ durch $\sigma \circ \tau$ mit einer Transposition $\tau \colon x \leftrightarrow y$. Die Summanden für $\sigma$ und $\sigma \circ \tau$ heben sich dann genau auf.

Für die übrigen (also injektiven) $\iota$ ist diese Summe genau die Plückerkoordinate $\sign \tau \cdot p_{(\iota\circ\tau)(0)\dots(\iota\circ\tau)(r)}$, wobei $\tau \in S_{r+1}$ die Permutation ist, die $\iota \circ \tau$ streng monoton macht.

Ist nun~$F$ eine homogene Form über $X_0, \dots, X_n$, so setzen wir $X_j = w_j$ und erhalten eine algebraische Gleichung in den~$\alpha_{i_1 \dots i_r}$ und den~$p_{i_0 \dots i_r}$. Der entsprechende Unterraum zu den Plückerkoordinaten liegt genau dann auf der durch die Form beschriebene Fläche, wenn die Form in den~$\alpha_{i_1 \dots i_r}$ identisch erfüllt ist. Demnach erhalten wir die gesuchten Gleichungen durch Koeffizientenvergleich.
\end{proof}

Wir haben Projektionen $\phi \colon \Gamma_{r,n}^d \to \mathbb P(H_{n+1}^d)$ und $\psi \colon \Gamma_{r,n}^d \to \grass r+1/{n+1}$. Beide sind natürlich regulär. Interessant ist vor allem die erste, denn ihr Bild besteht genau aus den Flächen, auf denen Geraden liegen. Vorerst wollen wir aber die zweite betrachten, die uns einige Informationen über $\Gamma_{r,n}^d$ liefert.

\begin{prop}
Die Abbildung $\psi$ ist surjektiv mit Fasern isomorph zu einem projektiven Raum der Dimension $\binom{n+d}d - \binom{r+d}d - 1$.
\end{prop}
\begin{proof}
Betrachte den Unterraum $L_0 = \{X_{r+1} = \dots = X_n = 0\} \subset \proj n$. Sein Urbild ist isomorph zu dem der anderen Unterräume, da auf $\grass r+1/{n+1}$ sowie $\mathbb P(H_{n+1}^d)$ die Gruppe $\GL{n+1}K$ operiert, und das auf der \textsc{Grassmann}-Mannigfaltigkeit transitiv geschieht.

Seien $a_{i_0 \dots i_n}$ die Koeffizienten von $X_0^{i_0} \dots X_n^{i_n}$ für $i_0 + \dots + i_n = d$. Dann bilden die Fasern einen Unterraum in $\mathbb P(H_{n+1}^d)$ gegeben durch $a_{i_0 \dots i_r 0 \dots 0} = 0$: auf allen solchen Flächen liegt offenbar $L_0$. Ist hingegen einer der Koeffizienten $a_{i_0 \dots i_r 0 \dots 0}$ nicht null, dann erhalten wir nach Einsetzen von $X_{r+1} = \dots = X_n = 0$ eine nichttriviale Form in $K[X_0, \dots, X_r]$. Nach \textsc{Hilbert}s Nullstellensatz kann diese nicht identisch verschwinden auf $\aff{n-r}$, also liegt $L_0$ nicht auf der Fläche. Damit folgt
\begin{equation*}
\psi^{-1}(L_0) \cong \mathbb P(H_{n+1}^d / H_{r+1}^d) \cong \proj{\binom{n+d}d - \binom{r+d}d - 1}.
\end{equation*}
Die Fasern sind projektive Räume, da ihre Gleichungen linear sind.
\end{proof}

\begin{coroll}
Die Varietät $\Gamma_{r,n}^d$ ist irreduzibel und hat Dimension
\begin{equation}
(r+1)(n-r) + \binom{n+d}d - \binom{r+d}d - 1.
\end{equation}
\end{coroll}
\begin{proof}
Das folgt mit dem Satz\footcite[S.~77, Theorem~8]{Shafarevich} über irreduzible Fasern, den wir oben schon verwendet haben. Die Projektion $\psi$ von $\Gamma_{r,n}^d$ auf $\grass r+1/{n+1}$ hat Fasern isomorph zu einem projektiven Unterraum von $\mathbb P(H_{n+1}^d)$, sie sind also irreduzibel. Mit Fakt~\ref{fact:grassirred} folgt, dass auch $\Gamma_{r,n}^d$ irreduzibel ist.

Die Dimension ist $\dim \grass r+1/{n+1} + \dim \psi^{-1}(L) = (r+1)(n-r) + \binom{n+d}d - \binom{r+d}d - 1$ nach dem Satz über die Faserdimension.
\end{proof}

Betrachten wir nun die Projektion $\phi \colon \Gamma_{r,n}^d \to \mathbb P(H_{n+1}^d)$. Die Dimension des Bildes, also der Untervarietät derjenigen Flächen $d$-ten Grades, auf denen $r$-dimensionale Unterräume liegen, ist durch die Dimension von $\Gamma_{r,n}^d$ beschränkt. Damit haben wir ein notwendiges Kriterium dafür, dass auf allen solchen Flächen $r$-dimensionale Unterräume liegen. Interessant ist vor allem, was wir über die Umkehrung sagen können.

\begin{theorem}
(1)~Liegt auf jeder Hyperfläche $\{F = 0\} \subset \proj n$ mit $\deg F = d$ ein $r$-dimensionaler Unterraum, so gilt
\begin{equation}
(r+1)(n-r) \geq \binom{r+d}d \qquad\text{bzw.}\qquad n \geq \frac{\binom{r+d}d}{r+1} + r.
\end{equation}

(2)~Sei $F \in K[X_0, \dots, X_n]$ mit $n = \binom{r+d}d / (r+1) + r$ eine homogene Form vom Grad $d$, sodass auf $X = \{F = 0\}$ endlich viele Unterräume der Dimension~$r$ liegen. Dann liegen auf jeder Hyperfläche der Dimension~$n$ und Grad~$d$ Unterräume der Dimension~$r$. Weiterhin ist deren Zahl endlich auf einer nichtleeren offenen Teilmenge des Modulraums.

(3)~Sei $k = (r+1)(n-r) - \binom{r+d}d \geq 0$ und $F \in K[X_0, \dots, X_n]$ eine homogene Form von Grad $d$, sodass die Varietät $\phi^{-1}(F)$ der Untervektorräume der Dimension $r$ auf $X = \{F = 0\}$ Dimension $k$ hat. Dann liegen auf jeder Hyperfläche der Dimension~$n$ und Grad~$d$ Unterräume der Dimension~$r$. Diese bilden eine $k$-dimensionale Varietät für eine nichtleere offene Teilmenge des Modulraums.
\end{theorem}
\begin{proof}
Sei $Y = \im \phi \subset \mathbb P(H_{n+1}^d)$. Als Bild einer projektiven Varietät unter einer regulären Abbildung ist es selbst projektive Varietät in $\mathbb P(H_{n+1}^d)$. Nun ist $\phi \colon \Gamma_{r,n}^d \to Y$ surjektiv, also ist $\phi^* \colon K[Y] \to K[\Gamma_{r,n}^d]$ injektiv. Da $\Gamma_{r,n}^d$ irreduzibel ist, ist $K[\Gamma_{r,n}^d]$ integer, mithin auch~$K[Y]$, also ist~$Y$ irreduzibel. Nun wenden wir den Satz über die Faserdimension an.

(1)~Zunächst folgt $\dim \Gamma_{r,n}^d \geq \dim \im \phi$. Liegt auf jeder Fläche vom Grad~$d$ in~$\proj n$ ein Unterraum der Dimension~$r$, dann ist $\im \phi = \mathbb P(H_{n+1}^d)$, also
\begin{align*}
\dim \Gamma_{r,n}^d = (r+1)(n-r) + \binom{n+d}d - \binom{r+d}d - 1 &\geq \binom{n+d}d - 1 \\
\Leftrightarrow \qquad (r+1)(n-r) &\geq \binom{r+d}d.
\end{align*}

(2)~Der Satz über die Faserdimension liefert, dass $\dim \phi^{-1}(y) \geq \dim \Gamma_{r,n}^d - \dim Y$ für alle $y \in Y$ mit Gleichheit auf einer nichtleeren offenen Teilmenge von~$Y$ gilt. Mit $y = F$ folgt $\dim \Gamma_{r,n}^d \leq \dim Y \leq \dim \mathbb P(H_{n+1}^d)$. Nach Voraussetzung gilt $\dim \Gamma_{r,n}^d = \dim \mathbb P(H_{n+1}^d)$, also gilt Gleichheit. Nun ist~$Y$ abgeschlossen in $\mathbb P(H_{n+1}^d)$ und hat volle Dimension, folglich ist $Y = \im \phi = \mathbb P(H_{n+1}^d)$.

(3)~Das folgt analog zu (2).
\end{proof}

Uns interessiert in den folgenden Kapiteln die Situation $n=3$, $r=1$, also Geraden auf projektiven Flächen. Obige Bedingung wird hier zu $d \leq 3$, die Differenz $\dim \Gamma_{r,n}^d - \dim \mathbb P(H_{n+1}^d)$ ist $3-d$. Folgende Fälle ergeben sich:
\begin{itemize}
\item Für $d=1$ liegt eine Ebene vor. Die Geraden auf einer Ebene bilden eine Varietät isomorph zu $\grass 2/3$, diese hat tatsächlich Dimension $3-d = 2$.
\item Wie in der Einleitung angedeutet, liegen auf jeder nichtentarteten Quadrik zwei Familien von Geraden, die durch $\proj 1$ parametrisiert werden. Die Fasern sind also im generischen Fall isomorph zu $\proj 1 \sqcup \proj 1$, das hat Dimension $3-d = 1$.
\item In diesem Fall ist genau die Gleichheitsbedingung erfüllt. Auf $F_3(K) = \{X_0^3 + X_1^3 + X_2^3 + X_3^3 = 0\}$ liegen für $\Char K \neq 2,3$ nur endlich viele Geraden.\footnote{siehe~Fakt \ref{fact:regular}} Damit folgt: auf allen kubischen Flächen liegen Geraden und auf einer offenen nichtleeren Teilmenge des Modulraums sind es auch nur endlich viele. Weitergehende kombinatorische Betrachtungen zeigen, dass auf jeder regulären kubischen Fläche genau $27$ Geraden liegen.\footcite[siehe etwa][]{Henderson} Das soll aber nicht Gegenstand dieser Arbeit sein.
\item Wie oben gesehen, ist $\phi$ für $d \geq 4$ nicht surjektiv. Wir können aber zumindest die Codimension des Bildes ausrechnen: auf den Fermat-Flächen $F_d(K) = \{X_0^d + X_1^d + X_2^d + X_3^d = 0\}$ liegen endlich viele Geraden.\footnote{siehe~Fakt \ref{fact:regular} und Satz~\ref{th:irreg}} Folglich ist die generische Faserdimension auf dem Bild null, dieses hat also Codimension $d-3$ in $\mathbb P(H_{n+1}^d)$.
\end{itemize}

\section{Lokale Betrachtungen}
Wir wollen untersuchen, wie viele Geraden auf der Fläche sich in einem Punkt schneiden können.

\begin{theorem} \label{th:local}
Sei $F \in K[X_0, \dots, X_3]$ homogene Form vom Grad $d$ und $X = \{F = 0\} \subset \proj 3$. Sei $x$ glatter Punkt von $X$, dann ist die Komponente von $x$ in $X$ eine Ebene oder es treffen sich in $x$ maximal $d$ der auf $X$ liegenden Geraden.
\end{theorem}
\begin{proof}
Durch eine lineare Substitution können wir erreichen, dass $x = (1:0:0:0)$ und die Tangentialebene in~$x$ durch $X_3 = 0$ definiert ist. Wegen $x_0 = 1$ gilt $X_0 \neq 0$ in einer offenen Umgebung von~$x$, also können wir im~Folgenden die affine Karte ${X_0 = 1}$ betrachten.

Sei also nun $\tilde F = \left. F \right|_{X_0=1} \in K[X_1, X_2, X_3]$ Form vom Grad~$d$ mit $\tilde F(0,0,0) = 0$ und $\tilde X = \{\tilde F = 0\}$. Sei $\mathfrak m = (X_1, X_2, X_3) = \{ f \in K[X_1, X_2, X_3] \colon f(0) = 0 \}$ das maximale Ideal des~$\aff 3$ im Punkt~$0$, dann ist $\tilde F \in \mathfrak m$. Der Tangentialraum $T_{0,X}$ an~$\tilde X$ in~$(0,0,0)$ ist der Nullraum der Linearisierung~$L$ von~$\tilde F$, d.\,h. dem Bild von~$\tilde F$ unter $\mathfrak m \to \mathfrak m/\mathfrak m^2 \cong T_{0,\aff 3}^*$. Nach obiger Annahme ist $L = X_3$.

Liegt nun eine Gerade $\{G_1 = G_2 = 0\}$ durch $0$, also mit $G_1(0) = G_2(0) = 0$, auf $\tilde X$, so gilt $(F_1) \subset (G_1, G_2) \subset \mathfrak m$. Wenden wir darauf obigen Linearisierungshomomorphismus an, so erhalten wir $(L + \mathfrak m^2) \subset (G_1 + \mathfrak m^2, G_2 + \mathfrak m^2) \subset T_{0,\aff 3}^*$. Da in $\mathfrak m/\mathfrak m^2$ alle Monome höheren als ersten Grades verschwinden, ist $L$ bereits linear erzeugt durch $G_1, G_2$, es gilt also $X_3 \in \mathrm{Lin}\{G_1, G_2\}$. Ohne Weiteres können wir daher annehmen, dass $G_1 = X_3$ ist und $G_2 = bX_1 - aX_2$. (Wir können $G_2$ um Vielfache von $G_1$ abändern.)

Damit haben alle Geraden die Form $K(a,b,0)$ mit $(a:b) \in \proj 1$. Offenbar ändert sich die aufgespannte Gerade nicht, wenn man $a$ und $b$ mit einer Konstanten aus $K^*$ multipliziert. Sei nun $a_I = a_{i_1 i_2 i_3}$ der Koeffizient von $X^I = X_1^{i_1} X_2^{i_2} X_3^{i_3}$ in~$\tilde F$. Damit die Gerade auf der Fläche $\tilde X$ liegt, muss die Gleichung
\begin{equation*}
0 = \sum_{|I| \leq d} a_I X^I = \sum_{i_1+i_2+i_3 \leq d} a_{i_1 i_2 i_3} (\lambda a)^{i_1} (\lambda b)^{i_2} 0^{i_3} = \sum_{i_1+i_2 \leq d} a_{i_1 i_2 0} \lambda^{i_1+i_2} a^{i_1} b^{i_2}
\end{equation*}
identisch in $\lambda$ erfüllt sein. Schreiben wir das etwas um:
\begin{equation*}
0 = \sum_{k=0}^d \left( \sum_{i_1+i_2 = k} a_{i_1 i_2 0} a^{i_1} b^{i_2} \right) \lambda^k.
\end{equation*}
Also ist die Bedingung äquivalent zu $P_k(a,b) = \sum_{i_1+i_2 = k} a_{i_1 i_2 0} a^{i_1} b^{i_2} = 0$ für alle~$k \leq d$. Das sind homogene Polynome in $(a:b) \in \proj 1$. Nun gilt es zwei Fälle zu unterscheiden: entweder verschwinden alle diese Polynome, d.\,h. $a_{i_1 i_2 0} = 0$ für alle~$i_1, i_2$. Dann liegen alle Geraden in der Tangentialebene auch in der Fläche, die Komponente von $x$ in $\tilde X$ ist also eine Ebene.

Andernfalls ist $P_k \neq 0$ für ein $k \leq d$. Nach dem Fundamentalsatz der Algebra hat $P_k$ höchstens $k$ Nullstellen. Also gehen durch $x$ höchstens $k \leq d$ der Geraden auf $X$.
\end{proof}
\begin{remarks}
Die Schranke ist tatsächlich scharf: wir werden im nächsten Kapitel Flächen sehen, auf denen in bestimmten Charakteristiken so viele Geraden liegen, dass sich jeweils~$d$ davon in einem Punkt schneiden. (s.~auch Korollar~\ref{cor:dlines})
\end{remarks}

\chapter{Geraden auf Fermat-Flächen} \label{chap:fermat}
Wir untersuchen nun einen wichtigen Spezialfall der allgemeinen Theorie, nämlich Geraden auf Fermat-Flächen. Im Jahre 1637 vermutete \textsc{Pierre de Fermat}, dass die Gleichung
\begin{equation*}
X^d + Y^d = Z^d
\end{equation*}
für $d \geq 3$ keine nichttrivialen ganzzahligen Lösungen hat. Dazu äquivalent ist, dass die Varietäten
\begin{equation*}
F_d^{(2)} = \{ X^d + Y^d = Z^d \} \subset \proj 2(\mathbb C)
\end{equation*}
keine rationalen Punkte mit $X, Y, Z \neq 0$ haben. Diese nennen sich \emph{Fermat-Kurven}. Wir bezeichnen eine analoge Familie von Flächen im $\proj 3$ als \emph{Fermat-Flächen}:
\begin{equation}
F_d = F_d^{(3)} = \{ X_0^d + X_1^d + X_2^d + X_3^d = 0 \} \subset \proj 3(K).
\end{equation}
Dabei können wir o.\,E. $\Char K \nmid d$ annehmen: sei $\Char K = p$, $d = kp$, dann ist $X_0^d + X_1^d + X_2^d + X_3^d = (X_0^k + X_1^k + X_2^k + X_3^k)^p$. Mithin ist $F_{kp} = F_k$, nun iteriert man bis $p \nmid d$ gilt. Eine allgemeine Theorie sogenannter Fermat-Varietäten findet sich in \cite{Fermat}.

\section{Allgemeine Betrachtungen}
Wir wollen nun die Rechnungen aus dem Beweis von Fakt~\ref{fact:gammaproj} konkreter machen. Sei $W \in \grass 2/4$ ein Unterraum mit Basis $\{a = (a_0, \dots, a_3), b = (b_0, \dots, b_3)\}$. Dann sind die Plückerkoordinaten $p_{ij} = a_i b_j - a_j b_i$, wobei $p_{ij} + p_{ji} = 0$. Wir rekonstruieren nun den Unterraum aus den Plückerkoordinaten:
\begin{equation}
W = \{ \phi(a)b - \phi(b)a \colon \phi \in (K^4)^* \}.
\end{equation}
Ein $\phi \in (K^4)^*$ hat die Form $\phi(x) = \sum_{i=0}^3 \alpha_i \langle x, e_i \rangle$ mit geeigneten Koeffizienten $\alpha_i \in K$. Damit
\begin{align*}
\phi(a)b - \phi(b)a &= \sum_i \alpha_i \langle a, e_i \rangle b - \sum_i \alpha_i \langle b, e_i \rangle a \\
	&= \sum_i \alpha_i a_i \sum_j b_j e_j - \sum_i \alpha_i b_i \sum_j a_j e_j \\
	&= \sum_j \sum_i \left(\alpha_i a_i b_j - \alpha_i b_i a_j \right) e_j \\
\intertext{Dabei laufen die Summen jeweils über $\{0,\dots,3\}$. Für $i=j$ verschwinden die Summanden jeweils, also erhalten wir}
\phi(a)b - \phi(b)a &= \sum_j \left(\sum_{i \neq j} \alpha_i p_{ij} \right) e_j
\end{align*}

Das setzen wir nun in die Gleichung der Fermat-Fläche $F_d$ ein:
\begin{align*}
0 = \sum_{j=0}^3 X_j^d &= \sum_{j=0}^3 \left(\sum_{i \neq j} \alpha_i p_{ij} \right)^d \\
\text{(Multinomialtheorem)}\qquad &= \sum_{j=0}^3 \sum_{\substack{(d_0,\dots,d_3) \\ \sum d_i=d,\;d_j=0}} \binom d{d_0,\dots,d_3} \prod_{i=0}^3 \alpha_i^{d_i} p_{ij}^{d_i} \\
	&= \sum_{\substack{(d_0,\dots,d_3) \\ \sum d_i=d}} \binom d{d_0,\dots,d_3} \left(\sum_{\substack{j=0 \\ d_j=0}}^3 \prod_{i=0}^3 p_{ij}^{d_i} \right) \prod_{i=0}^3 \alpha_i^{d_i}
\end{align*}
Da die Gleichung in den $\alpha_i$ identisch gelten soll, können wir nach \textsc{Hilbert}s Nullstellensatz einen Koeffizientenvergleich machen. Ein Vergleich der Koeffizienten zu $\prod_{i=0}^3 \alpha_i^{d_i}$ für ein $(d_0,\dots,d_3)$ ergibt
\begin{equation}
\binom d{d_0,\dots,d_3} \sum_{\substack{j=0 \\ d_j=0}}^3 \prod_{i=0}^3 p_{ij}^{d_i} = 0.
\end{equation}
Das liefert uns den folgenden Fakt.

\begin{fact}
Eine projektive Gerade mit Plückerkoordinaten $(p_{ij})$ liegt genau dann auf der Fermat-Fläche vom Grad $d$, wenn für alle $(d_0,\dots,d_3)$ mit $d_0 + \dots + d_3 = d$ und $\binom d{d_0,\dots,d_3} \neq 0$ die folgende Gleichung gilt:
\begin{equation}
\sum_{\substack{j=0 \\ d_j=0}}^3 \prod_{i=0}^3 p_{ij}^{d_i} = 0.
\end{equation}
\end{fact}

\section{Reguläre Geraden}
Es ist bekannt, dass auf einer Fermat-Fläche vom Grad $d$ eine Familie von $3d^2$ Geraden liegt.\footcite[siehe u.\,a.][S.~5]{LinesOnFermat} Wir zeigen im Folgenden, dass es in den meisten Fällen auch nicht mehr als diese gibt.

Seien Indizes $i,j,k,l$ gewählt mit $\{i,j,k,l\} = \{0,1,2,3\}$. Unabhängig von $d$ ist dann $\binom d{d,0,0,0} = 1 \neq 0$ (und analog für Permutationen), damit haben wir
\begin{equation} \label{eq:powers}
p_{ij}^d + p_{ik}^d + p_{il}^d = 0.
\end{equation}
Weiterhin ist $\binom d{d-1,1,0,0} = d \neq 0$, daher erhalten wir Gleichungen der Form
\begin{equation} \label{eq:ratios}
p_{jk}^{d-1} p_{ik} + p_{jl}^{d-1} p_{il} = 0 \qquad\overrel\Longleftrightarrow^{p_{il}, p_{jk} \neq 0}\qquad \frac{p_{ik}}{p_{il}} = -\left(\frac{p_{jl}}{p_{jk}}\right)^{d-1}.
\end{equation}
Inwiefern weitere Gleichungen erfüllt sein müssen, beantwortet die folgende Proposition.
\begin{prop}
Sei $p \in \mathbb P$ und $d$ kein Vielfaches von $p$. Gilt weiterhin, dass $d-1$ keine $p$-Potenz ist, dann gibt es $d_0, d_1, d_2 > 0$ mit $p \nmid \binom d{d_0,d_1,d_2,0}$.
\end{prop}
\begin{proof}
Nach der $p$-adischen Stirlingformel ist $\ord_p n! = \frac{n - \sigma_p(n)}{p-1}$, wobei $\sigma_p$ die $p$-adische Quersumme ist.\footcite[Kap.~2, §8, Lemma~1, S.~171]{LieGroups} Damit folgt $(p-1)\ord_p \binom d{d_0,d_1,d_2,d_3} = \sigma_p(d) - \sum_i \sigma_p(d_i)$. Damit $p \nmid \binom d{d_0,d_1,d_2,d_3}$ gilt, darf also bei der Summe $\sum_i d_i$ kein $p$-adischer Übertrag stattfinden. Um die Proposition zu zeigen, genügt es daher, $d$ in drei nichtverschwindende Summanden zu zerlegen, sodass die Summation keinen Übertrag ergibt. Das ist offenbar für $\sigma_p(d) \geq 3$ immer möglich.

Die Fälle $\sigma_p(d) = 0, 1$ führen auf $d = 0, p^n$ mit $n \in \mathbb N$, was der Voraussetzung widerspricht. Für $\sigma_p(d) = 2$ hat $d$ wegen $p \nmid d$ die Form $d=p^n+1$, was ebenfalls ausgeschlossen ist. Also ist $\sigma_p(d) \geq 3$.
\end{proof}

\begin{fact} \label{fact:regular}
Sei $K$ Körper der Charakteristik~$p \in \mathbb P \cup \{0\}$, $d \geq 3$ mit $p \nmid d$, und $d-1$ keine Potenz von~$p$. Dann liegen auf $F_d(K)$ genau die drei Familien von Geraden
\begin{equation} \label{eq:regular}
\begin{split}
\text{(I)}\qquad	&(1:\theta:0:0)(0:0:1:\eta) \\
\text{(II)}\qquad	&(1:0:\theta:0)(0:1:0:\eta) \\
\text{(III)}\qquad	&(1:0:0:\theta)(0:1:\eta:0)
\end{split} \qquad \theta, \eta \in \mu_{2d} \setminus \mu_d,
\end{equation}
wobei $\mu_d \subset K$ die Menge der $d$-ten Einheitswurzeln ist.
\end{fact}
\begin{proof}
Nach voriger Proposition ist $\binom d{d_0,d_1,d_2,0} \neq 0$ für geeignete $d_0, d_1, d_2 > 0$. Damit haben wir
\begin{equation} \label{eq:products}
p_{03}^{d_0} p_{13}^{d_1} p_{23}^{d_2} = 0
\end{equation}
und Varianten. Das bedeutet: für jedes $i$ verschwindet mindestens eines der $p_{ij}$. Wir machen uns zunächst klar, dass nicht mehr verschwinden können: sei o.\,E. $p_{01} = p_{02} = 0$, dann ist wegen \eqref{eq:powers} auch $p_{03} = 0$. Mit \eqref{eq:ratios} folgt daraus, dass $p_{13}^{d-1}p_{23} = p_{12}^{d-1}p_{23} = p_{12}^{d-1}p_{13} = 0$, also verschwinden mindestens zwei von der drei Koordinaten $p_{12}$, $p_{13}$, $p_{23}$. Dass nur eine Koordinate nicht verschwindet, geht aber wegen \eqref{eq:powers} nicht.

Also verschwindet jeweils einer der Summanden in Gleichung \eqref{eq:powers} und diese bekommen die Form $X^d + Y^d = 0$. Das ist äquivalent zu $(X/Y)^d = -1$ bzw. $X/Y \in \mu_{2d} \setminus \mu_d$. Schreiben wir nun die Plückerkoordinaten in einer Tabelle auf:

{\vskip 2ex\hfil
\begin{tabular}{|c|c|c|c|} \hline
0 & $p_{01}$ & $p_{02}$ & $p_{03}$ \\ \hline
$-p_{01}$ & 0 & $p_{12}$ & $p_{13}$ \\ \hline
$-p_{02}$ & $-p_{12}$ & 0 & $p_{23}$ \\ \hline
$-p_{03}$ & $-p_{13}$ & $-p_{23}$ & 0 \\ \hline
\end{tabular}
\hfil\vskip 2ex}

In jeder Spalte und Zeile steht zusätzlich eine Null, insgesamt hat die Tabelle also acht Nulleinträge. Von den vier Nulleinträgen, die nicht auf der Diagonale liegen, sind jeweils zwei oberhalb und zwei unterhalb, also verschwinden zwei der sechs $p_{ij}$, seien dies $p_{ij}$ und $p_{kl}$, dabei gilt $\{i,j,k,l\} = \{0,1,2,3\}$. Wir können also ohne Einschränkung annehmen, dass $p_{01}$ und $p_{23}$ verschwinden.

Setzen wir nun $p_{02} = 1$, dann ergibt sich $p_{03} = \theta$ und $p_{12} = \eta$ mit $\theta, \eta \in \mu_{2d}$ und $\theta^d = \eta^d = -1$. Mit \eqref{eq:grcond} folgt $p_{13} = \theta\eta$. Die entsprechende projektive Gerade wird durch $(1,\theta,0,0)$ und $(0,0,1,\eta)$ aufgespannt. Man überzeugt sich leicht, dass diese tatsächlich auf der Fermat-Fläche liegt. Die anderen Klassen ergeben sich für $p_{02} = p_{13} = 0$ bzw.~$p_{03} = p_{12} = 0$.
\end{proof}

Im generischen Fall liegen also $3d^2$ Geraden auf eine Fermat-Fläche vom Grad $d$. Im nächsten Kapitel werden wir ihre Konfiguration untersuchen. Betrachten wir aber zunächst noch die Spezialfälle.

\section{Geistergeraden}
Die Existenz von zusätzlichen Geraden ist auch bekannt.\footcite[siehe][S.~14f]{LinesOnFermat} Wir wollen nun genau untersuchen, wann es sie gibt und wie viele davon.
\begin{prop}
Sei $p \in \mathbb P$ und $d = p^n+1$ mit $n \in \mathbb N$. Dann sind alle Multinomialkoeffizienten $\binom d{d_0,d_1,d_2,d_3}$ mit $d_0, d_1, d_2, d_3 \not\in \{d-1, d\}$ durch $p$ teilbar.
\end{prop}
\begin{proof}
Wir zählen, wie oft die Faktoren in Zähler und Nenner dieses Bruches verschiedene Potenzen von $p$ enthalten:
\begin{equation*}
\binom d{d_0,d_1,d_2,d_3} = \frac{d!}{d_0! d_1! d_2! d_3!}.
\end{equation*}
Für $0 < k < n$ enthalten im Zähler $p^{n-k}$ Faktoren einen Faktor $p^k$, im Nenner sind es $\sum_i \lfloor d_i/p^k \rfloor \leq \sum_i d_i/p^k = d/p^k$, also höchstens ebenso viele. Der Zähler enthält allerdings einen Faktor $p^n$, der Nenner wegen $d_i \not\in \{d-1, d\} = \{p^n, p^n+1\}$ nicht. Damit ist der Bruch durch $p$ teilbar.
\end{proof}

\begin{lemma}
Sei $K$ ein Körper der Charakteristik $p$. Die Gleichung $X+Y+Z=0$ mit $X,Y,Z \in \mu_{p^n-1}$ hat dann $(p^n-1)(p^n-2)$ Lösungen.
\end{lemma}
\begin{proof}
Offenbar gilt $\mu_{p^n-1} = \mathbb F_{p^n}^* \subset K$. Es sind also alle Lösungen von $X+Y+Z=0$ mit $X,Y,Z \in \mathbb F_{p^n} \setminus \{0\}$ zu finden. Das Folgende ist nun Kombinatorik: $X$ können wir aus $p^n-1$ verschiedenen Werten wählen. Haben wir $Y$ gewählt, ergibt sich $Z$ als $Z=-X-Y$. Damit $Z \neq 0$ ist, muss $X+Y \neq 0$ sein, also $Y \not\in \{0,-X\}$. Folglich gibt es für $Y$ genau $p^n-2$ mögliche Wahlen.
\end{proof}
\begin{coroll} \label{cor:projrootsum}
Sei $K$ ein Körper der Charakteristik $p$. Die Nullstellenmenge der Gleichung $X+Y+Z=0$ in $\proj 3(K)$ mit $X,Y,Z \in \mu_{p^n-1}$ besteht dann aus $p^n-2$ Elementen.
\end{coroll}

\begin{theorem}[Geistergeraden] \label{th:irreg}
Für alle Charakteristiken $p > 2$ und Grade $d = p^n + 1$ mit $n \in \mathbb N$ liegen auf $F_d$ neben den Familien von Geraden (I)--(III) zusätzlich die Geraden
\begin{equation} \label{eq:ghost}
\text{(IV)}\qquad (0:\mu\eta i:1:\nu\theta i)(-\mu\eta i:0:\nu\lambda i/\theta:\lambda)
\end{equation}
mit den Parametern $\lambda \in \mu_d$, $\eta, \theta \in \mu_{2d}$ und $\mu, \nu \in \mu_{2(d-2)}$ und den Bedingungen $\mu^2 + 1 + \nu^2 = 0$, $(i\eta)^d = \mu^{d-2}$ und $(i\theta)^d = \nu^{d-2}$. Weitere Geraden gibt es nicht.
\end{theorem}
\begin{remarks}
Ändert man $\eta$ und $\mu$ oder $\theta$ und $\nu$ um den Faktor $-1$ ab, erhält man diesselbe Gerade. Die Zuordnung von Geraden zu Parametern ist also nicht eindeutig. Legt man sich allerdings für $\mu$ und $\nu$ auf eine Quadratwurzel von $\mu^2$ bzw.~$\nu^2$ fest, ist die Eindeutigkeit wiederhergestellt, wie sich aus dem Beweis ergibt.
\end{remarks}
Die Geraden \eqref{eq:regular} nennen wir \emph{reguläre} Geraden, die in \eqref{eq:ghost} \emph{Geistergeraden}. Die Anzahl der regulären Geraden ist $3d^2$, die der Geistergeraden $(d-3)d^3$.
\begin{proof}
Nach obiger Proposition haben wir neben \eqref{eq:grcond} genau die Gleichungen für $\binom{d}{d,0,0,0}=1$ und $\binom{d}{d-1,1,0,0}$ und Varianten. Der erste Fall führt auf die Gleichungen \eqref{eq:powers}, der zweite auf \eqref{eq:ratios}. Letztere in sich selbst eingesetzt liefern
\begin{equation*}
\frac{p_{12}}{p_{13}} = -\left(\frac{p_{03}}{p_{02}}\right)^{d-1} = -\left(-\left(\frac{p_{12}}{p_{13}}\right)^{d-1}\right)^{d-1} = \left(\frac{p_{12}}{p_{13}}\right)^{(d-1)^2}, \qquad\text{da $p$ ungerade,}
\end{equation*}
falls $p_{12}, p_{13} \neq 0$ und $p_{03} p_{02} \neq 0$. Verschwindet also keine der Plückerkoordinaten, so sind ihre Verhältnisse $k$-te Einheitswurzeln mit $k=(d-1)^2-1=d(d-2)$. Da die Plückerkoordinaten homogene Koordinaten sind, können wir annehmen, dass $p_{ij} \in \mu_{d(d-2)}$ für alle $i \neq j$.

Betrachten wir nun \eqref{eq:ratios} zur $d$-ten Potenz erhoben:
\begin{equation*}
\frac{p_{ik}^d}{p_{il}^d} = \left(\frac{p_{ik}}{p_{il}}\right)^d = \left(\frac{p_{jl}}{p_{jk}}\right)^{d(d-1)} = \left(\frac{p_{jl}}{p_{jk}}\right)^d = \frac{p_{jl}^d}{p_{jk}^d}.
\end{equation*}
Mit der Substitution $i \leftrightarrow k$, $j \leftrightarrow l$ erhalten wir
\begin{equation*}
\frac{p_{ik}^d}{p_{jk}^d} = \frac{p_{ki}^d}{p_{kj}^d} = \frac{p_{lj}^d}{p_{li}^d} = \frac{p_{jl}^d}{p_{il}^d}.
\end{equation*}
Durcheinander geteilt ergibt das
\begin{equation*}
\frac{p_{jk}^d}{p_{il}^d} = \frac{p_{il}^d}{p_{jk}^d} \qquad\text{bzw.}\qquad \frac{p_{il}^d}{p_{jk}^d} = \pm 1.
\end{equation*}

Setze $\mu = p_{01}^d$, $\eta = p_{02}^d$, $\nu = p_{03}^d$. Die Plückerkoordinaten in $d$-ter Potenz verhalten sich dann wie folgt:
{\vskip 2ex\hfil
\begin{tabular}{|c|c|c|c|} \hline
0 & $\mu$ & $\eta$ & $\nu$ \\ \hline
$\mu$ & 0 & $\pm \nu$ & $\pm \eta$ \\ \hline
$\eta$ & $\pm \nu$ & 0 & $\pm \mu$ \\ \hline
$\nu$ & $\pm \eta$ & $\pm \mu$ & 0 \\ \hline
\end{tabular}
\hfil\vskip 2ex}
Wegen \eqref{eq:powers} müssen die Summen über alle Zeilen und Spalten gleich sein. Eine leichte Überlegung ergibt dann, dass für alle $\pm$ nur $+$ infrage kommt. Steht an nur einer Stelle ein $-$, sei also beispielsweise $p_{12}=-\nu$, aber $p_{13}=\eta$. Dann ergibt eine Subtraktion der Gleichungen \eqref{eq:powers} für die ersten beiden Zeilen $\nu = -\nu$, also $\nu = 0$. Das ist ein Widerspruch. Steht an mindestens zwei Stellen ein $-$, sei also o.\,E. $p_{12}=-\nu$ und $p_{13}=-\eta$. Dann ergibt eine Addition der ersten beiden Zeilen, dass $2\mu = 0$, also $\mu = 0$. Auch das geht nicht.

Die Gleichung ist daher genau dann erfüllt, wenn $\mu+\eta+\nu = 0$ mit $\mu, \eta, \nu \in \mu_{d-2} = \mathbb F_{p^n}^*$. Ist $\zeta \in \mu_{d(d-2)}$ primitiv, so können wir $\mu = \zeta^{ad}$, $\eta = \zeta^{bd}$, $\nu = \zeta^{cd}$ schreiben. Nach dem vorigen Lemma gibt es genau $(p^n-1)(p^n-2)$ solche Tripel $(a,b,c) \in (\Zmod (d-2)Z)^3$. Wir liften sie nach $(\Zmod d(d-2)Z)^3$ und nennen sie $a_{01} = a_{23} \equiv a, a_{02} = a_{13} \equiv b$, $a_{03} = a_{12} \equiv c \pmod{d-2}$.

Damit haben die $p_{ij}$ die Form $\zeta^{a_{ij} + b_{ij}(d-2)}$ sind mit $b_{ij} \in \Zmod dZ$. Die Gleichungen \eqref{eq:ratios} werden dann zu
\begin{align*}
\frac{\zeta^{a_{ik} + b_{ik}(d-2)}}{\zeta^{a_{il} + b_{il}(d-2)}} = \frac{p_{ik}}{p_{il}} &= -\left(\frac{p_{jl}}{p_{jk}}\right)^{d-1} = -\left(\frac{\zeta^{a_{jl} + b_{jl}(d-2)}}{\zeta^{a_{jk} + b_{jk}(d-2)}}\right)^{d-1} \\
a_{ik} - a_{il} + (b_{ik} - b_{il})(d-2) &\equiv (a_{jl} - a_{jk} + (b_{jl} - b_{jk})(d-2))(d-1) + d(d-2)/2 &&\mod{d(d-2)} \\
(b_{ik} - b_{il} + b_{jl} - b_{jk})(d-2) &\equiv (a_{ik} - a_{il})(d-2) + d(d-2)/2 &&\mod{d(d-2)} \\
b_{ik} - b_{il} - b_{jk} + b_{jl} &\equiv a_{ik} - a_{il} + d/2 &&\mod d
\end{align*}
Man beachte dabei, dass $a_{ik} = a_{jl}$, $a_{il} = a_{jk}$. Weiterhin gilt $b_{ij} - b_{ji} \equiv d/2 \pmod d$ wegen $p_{ij} + p_{ji} = 0$. Wir müssen die Gleichung nicht für alle Permutationen testen, sondern wegen Symmetrie nur für die drei Quadrupel $(i,j,k,l) = (0,1,2,3), (0,2,1,3), (0,3,1,2)$.

Das ergibt ein inhomogenes Gleichungssystem in den $b_{ij} \in \Zmod dZ$, wir lösen sie aber zunächst im größeren $(\Zmod 2dZ)$-Modul $\frac 12 \Zmod dZ$. Dort können wir leicht eine Lösung angeben: für $i<j$ setze $b_{ij} = a_{ij}/2$ für $2 \mid i-j$ und $b_{ij} = a_{ij}/2 + d/4$ sonst. Es ist zu prüfen, ob für die drei Quadrupel die Gleichung erfüllt ist:
\begin{align*}
&(0,1,2,3): &(a_{02}/2) - (a_{03}/2+\tfrac d4) - (a_{12}/2+\tfrac d4) + (a_{13}/2) &\overrel{\equiv}^! a_{02} - a_{03} + d/2 \\
&(0,2,1,3): &(a_{01}/2+\tfrac d4) - (a_{03}/2+\tfrac d4) - (a_{12}/2-\tfrac d4) + (a_{23}/2+\tfrac d4) &\overrel{\equiv}^! a_{01} - a_{03} + d/2 \\
&(0,3,1,2): &(a_{01}/2+\tfrac d4) - (a_{02}/2) - (a_{13}/2+\tfrac d2) + (a_{23}/2-\tfrac d4) &\overrel{\equiv}^! a_{01} - a_{02} + d/2
\end{align*}

Nun zur Lösung des homogenen Gleichungssystems. Statt es über dem $\Zmod 2dZ$-Modul $\frac 12 \Zmod dZ$ zu lösen, kann man es auch als Gleichungssystem über dem Ring $\Zmod 2dZ$ selbst betrachten, indem man alle Koordinaten verdoppelt.
\begin{equation*}
\begin{pmatrix}
1 & 0 & -1 & -1 & 0 & 1 \\
0 & 1 & -1 & -1 & 1 & 0 \\
1 & -1 & 0 & 0 & -1 & 1
\end{pmatrix}
\begin{pmatrix}
b_{01} \\ b_{02} \\ b_{03} \\ b_{12} \\ b_{13} \\ b_{23}
\end{pmatrix}
= \underline{0}
\end{equation*}
Das ist äquivalent zu $b_{01} + b_{23} = b_{02} + b_{13} = b_{03} + b_{12} = k/2$, $k \in \Zmod 2dZ$. Die allgemeine Lösung mit Parametern $\alpha, \beta, \gamma, k \in \Zmod 2dZ$ ist daher:
{\vskip 2ex\hfil
\begin{tabular}{|c|c|c|c|} \hline
- & $(a+\alpha)/2+d/4$ & $(b+\beta)/2$ & $(c+\gamma)/2+d/4$ \\ \hline
$(a+\alpha)/2-d/4$ & - & $(c+k-\gamma)/2+d/4$ & $(b+k-\beta)/2$ \\ \hline
$(b+\beta)/2+d/2$ & $(c+k-\gamma)/2-d/4$ & - & $(a+k-\alpha)/2+d/4$ \\ \hline
$(c+\gamma)/2-d/4$ & $(b+k-\beta)/2+d/2$ & $(a+k-\alpha)/2-d/4$ & - \\ \hline
\end{tabular}
\hfil\vskip 2ex}
Um die Lösungen der Gleichung in $\Zmod dZ$ zu bekommen, schränken wir sie einfach ein. Das bedeutet $\alpha \equiv a+d/2 \equiv k-\alpha,\; \beta \equiv b \equiv k-\beta,\; \gamma \equiv c+d/2 \equiv k-\gamma \pmod 2$. Insbesondere gilt also $2 \mid k$, dies ist auch ausreichend für die Wahl von $k$. Die Parameter $\alpha$, $\beta$, $\gamma$ haben also die selbe Parität wie $a+d/2$, $b$, resp.~$c+d/2$. Die endgültigen Exponenten $a_{ij} + (d-2)b_{ij}$ sind damit:
{\vskip 2ex\hskip 0pt plus 1fil minus 1.5cm
\begin{tabular}{|c|c|c|c|} \hline
- & $\frac{ad}2+\alpha\frac{d-2}2+\frac{d(d-2)}4$ & $\frac{bd}2+\beta\frac{d-2}2$ & $\frac{cd}2+\gamma\frac{d-2}2+\frac{d(d-2)}4$ \\ \hline
$\frac{ad}2+\alpha\frac{d-2}2-\frac{d(d-2)}4$ & - & $\frac{cd}2+(k-\gamma)\frac{d-2}2+\frac{d(d-2)}4$ & $\frac{bd}2+(k-\beta)\frac{d-2}2$ \\ \hline
$\frac{bd}2+\beta\frac{d-2}2+\frac{d(d-2)}2$ & $\frac{cd}2+(k-\gamma)\frac{d-2}2-\frac{d(d-2)}4$ & - & $\frac{ad}2+(k-\alpha)\frac{d-2}2+\frac{d(d-2)}4$ \\ \hline
$\frac{cd}2+\gamma\frac{d-2}2-\frac{d(d-2)}4$ & $\frac{bd}2+(k-\beta)\frac{d-2}2+\frac{d(d-2)}2$ & $\frac{ad}2+(k-\alpha)\frac{d-2}2-\frac{d(d-2)}4$ & - \\ \hline
\end{tabular}
\hfil\vskip 2ex}

Offenbar zählen wir dabei einige Geraden mehrfach. Möchte man eine eindeutige Zuordnung zwischen Parametern und Geraden erhalten, kann man beispielsweise $p_{02} = 1$ setzen. Damit ist $b = \beta = 0$. Man überzeugt sich leicht, dass damit Eindeutigkeit hergestellt ist.

Nun überprüfen wir noch, dass die Gleichung der \textsc{Grassmann}-Mannigfaltigkeit \eqref{eq:grcond} erfüllt ist:
\begin{align*}
0 &\overrel{=}^! \zeta^{\frac{ad}2+\alpha\frac{d-2}2+\frac{d(d-2)}4} \zeta^{\frac{ad}2+(k-\alpha)\frac{d-2}2+\frac{d(d-2)}4} - \zeta^{\frac{bd}2+\beta\frac{d-2}2} \zeta^{\frac{bd}2+(k-\beta)\frac{d-2}2} \\
  &\qquad + \zeta^{\frac{cd}2+\gamma\frac{d-2}2+\frac{d(d-2)}4} \zeta^{\frac{cd}2+(k-\gamma)\frac{d-2}2+\frac{d(d-2)}4} \\
  &= \zeta^{ad + k\frac{d-2}2 + \frac{d(d-2)}2} - \zeta^{bd + k\frac{d-2}2} + \zeta^{cd + k\frac{d-2}2 + \frac{d(d-2)}2} \\
  &= -\zeta^{k\frac{d-2}2}(\zeta^{ad} + \zeta^{bd} + \zeta^{cd})
\end{align*}
Der Term in Klammern verschwindet, also ist die Gleichung automatisch erfüllt. Nun gilt es noch, je zwei Punkte auf den Geraden zu finden. Das liefern Schnitte mit den Ebenen $X=0$ und $Y=0$: wir erhalten die Punkte $(0:p_{01}:p_{02}:p_{03})$, $(p_{10}:0:p_{12}:p_{13})$. Oder, mit den Substitutionen $\lambda = \zeta^{k \frac{d-2}2}$, $\eta = \zeta^{\alpha \frac{d-2}2}$, $\theta = \zeta^{\gamma\frac{d-2}2}$, weiterhin $\mu = \zeta^{ad/2}$, $\nu = \zeta^{cd/2}$ sowie $i = \zeta^{\frac{d(d-2)}4}$:
\begin{equation}
(0:\mu\eta i:1:\nu\theta i),\qquad
(-\mu\eta i:0:\nu\lambda i/\theta:\lambda).
\end{equation}
Die Bedingungen werden dann zu $\lambda \in \mu_d$, $\eta, \theta \in \mu_{2d}$ und $\mu, \nu \in \mu_{2(d-2)}$ mit $\mu^2 + 1 + \nu^2 = 0$, außerdem $(i\eta)^d = \mu^{d-2}$ und $(i\theta)^d = \nu^{d-2}$.

Wir sahen in Korollar~\ref{cor:projrootsum}, dass es für $(a,0,c) \in (\Zmod (d-2)Z)^3$ mit $\zeta^{ad} + 1 + \zeta^{cd} = 0$ genau $p^n-2$ Lösungen gibt. Analog gibt es $p^n-2$ Lösungen für $(\mu^2, 1, \nu^2)$, und von den zwei Wurzeln kann man sich je eine aussuchen. Für die verbleibenden drei Variablen $\lambda$, $\eta$ und $\theta$ gibt es dann jeweils $d=p^n+1$ Lösungen, da $\eta^d$ und $\theta^d$ durch die Wahl von $\mu$ und $\nu$ festgelegt sind. Weitere Beschränkungen gibt es nicht, daher kommen wir auf $(d-3)d^3$ bzw. $(p^n-2)(p^n+1)^3$ Geraden.

Nun zu dem Fall, dass Plückerkoordinaten verschwinden. Sei also $p_{ij}=0$, dann ist wegen~\eqref{eq:ratios}
\begin{align*}
p_{ij}^{d-1}p_{kj} + p_{il}^{d-1}p_{kl} = 0 \qquad\Rightarrow p_{il} = 0 \text{ oder } p_{kl} = 0, \\
p_{ij}^{d-1}p_{lj} + p_{ik}^{d-1}p_{lk} = 0 \qquad\Rightarrow p_{ik} = 0 \text{ oder } p_{kl} = 0, \\
p_{ji}^{d-1}p_{ki} + p_{jl}^{d-1}p_{kl} = 0 \qquad\Rightarrow p_{jl} = 0 \text{ oder } p_{kl} = 0, \\
p_{ji}^{d-1}p_{li} + p_{jk}^{d-1}p_{lk} = 0 \qquad\Rightarrow p_{jk} = 0 \text{ oder } p_{kl} = 0.
\end{align*}
Also gilt $p_{kl} = 0$ oder $p_{il} = p_{ik} = p_{jl} = p_{jk} = 0$. Im zweiten Fall verschwinden dann alle Plückerkoordinaten wegen \eqref{eq:powers}, der erste führt auf die regulären Geraden aus dem vorigen Fakt.
\end{proof}
\begin{remarks}
Es fehlt eine Betrachtung des Falles $\Char K = 2$. In diesem Fall ist $d$ ungerade, es ist also ein anderes Ergebnis zu erwarten. Vermutlich kommt man aber mit einer ähnlichen Vorgehensweise zum Ziel.
\end{remarks}

\section{Gemeinsame Darstellung}
Betrachtet man die letzte Tabelle in vorigem Beweis, so erkennt man vielleicht, dass die regulären Geraden als Spezialfall der Geistergeraden aufgefasst werden können. Mit den Substitutionen $\lambda_1 = \zeta^{ad/2}$, $\lambda_2 = \zeta^{bd/2}$, $\lambda_3 = \zeta^{cd/2}$, $i = \zeta^{d(d-2)/4}$, $\nu_1 = \zeta^{\alpha(d-2)/2}$, $\nu_2 = \zeta^{\beta(d-2)/2}$, $\nu_3 = \zeta^{\gamma(d-2)/2}$, $\theta = \zeta^{k(d-2)/2}$ erhält man folgende Verallgemeinerung, indem man für $\lambda_{1,2,3}$ auch den Wert $0$ zulässt.
\begin{coroll}
Sei $\Char K = p > 0$ und $d = q+1$, $q = p^n$. Dann liegen auf $F_d$ genau die Geraden mit den Plückerkoordinaten
\begin{align*}
p_{01} &= \lambda_1 \nu_1 i &p_{02} &= \lambda_2 \nu_2 &p_{03} &= \lambda_3 \nu_3 i \\
 & &p_{12} &= \lambda_3 \theta i / \nu_3 &p_{13} &= \lambda_2 \theta / \nu_2 \\
 & & & &p_{23} &= \lambda_1 \theta i / \nu_1
\end{align*}
mit den Parametern $(\lambda_{1:2:3}^2) \in \proj 2(\mathbb F_q)$, $\theta \in \mu_d$ und $\nu_{1,2,3} \in \mu_{2d}$ und den Bedingungen $\lambda_1^2 + \lambda_2^2 + \lambda_3^2 = 0$ und $(i^j \nu_j)^d = \lambda_j^{d-2}$ für $j=1,2,3$.
\end{coroll}
\begin{proof}
Die Plückerkoordinaten $(:\!p_{ij}) \in \mathbb P(\Lambda^2 K^4)$ sind wohldefiniert, da die Parameter $(\lambda_{1:2:3})$ darin homogen eingehen. Es sind zwei Fälle zu unterscheiden: der Fall $\lambda_{1,2,3} \neq 0$ führt auf die Geistergeraden aus dem vorigen Satz, da $\mathbb F_q^* = \mu_{d-2}$. Offenbar erhalten wir so auch alle, wie sich aus den Substitutionen ergibt.

Verschwindet hingegen eines der $\lambda_j$, dann sind die beiden anderen verschieden von null und unterscheiden sich um einen Faktor $\pm i$, da sich ihre Quadrate um den Faktor $-1$ unterscheiden. Es genügt, eine der beiden Möglichkeiten zu untersuchen: die andere erhält man, indem man $\nu_{1/2/3}$ um den Faktor $-1$ abändert. Weiterhin können wir eines der beiden nicht verschwindenden $\lambda_j$ festlegen.

Für $\lambda_1 = 0$, $\lambda_2 = 1$ und $\lambda_3 = -i$ ergeben sich die regulären Geraden der Klasse~(I), für $\lambda_1 = 1$, $\lambda_2 = 0$ und $\lambda_3 = i$ ergeben sich die regulären Geraden der Klasse~(II), $\lambda_1 = -i$, $\lambda_2 = 1$ und $\lambda_3 = 0$ ergeben sich die regulären Geraden der Klasse~(III). Das sieht man so: die Situation der regulären Geraden liegt vor, denn zwei Plückerkoordinaten verschwinden. Die Verhältnisse der übrigen sind wie durch \eqref{eq:ratios} gefordert:
\begin{align*}
\mathrm{(I)}\qquad p_{02}/p_{12} &= \nu_2 \nu_3 / \theta = p_{03}/p_{13} &p_{02}/p_{03} &= \nu_2/\nu_3 = p_{12}/p_{13} \\
\mathrm{(II)}\qquad p_{01}/p_{21} &= \nu_1 \nu_3 i/ \theta = p_{03}/p_{23} &p_{01}/p_{03} &= -\nu_1 i/\nu_3 = p_{21}/p_{23} \\
\mathrm{(III)}\qquad p_{01}/p_{31} &= -\nu_1 \nu_2/\theta = p_{02}/p_{32} &p_{01}/p_{02} &= \nu_1/\nu_2 = p_{31}/p_{32}.
\end{align*}

Bleibt nur noch zu prüfen, dass die Verhältnisse hoch~$d$ gleich~$-1$ sind. Wir rechnen es hier nur für die Klasse~(I) nach, für die anderen folgt es analog:
\begin{align*}
\left( \frac{\nu_2 \nu_3}{\theta} \right)^d &= \frac{\nu_2^d \nu_3^d}{\theta^d} = \lambda_2^{d-2} \cdot i^d \lambda_3^{d-2} = i^d (-i)^{d-2} = i^{2d-2} = (-1)^{d-1} = -1\qquad\text{und} \\
\left( \frac{\nu_2}{\nu_3} \right)^d &= \frac{\nu_2^d}{\nu_3^d} = \frac{\lambda_2^{d-2}}{i^d \lambda_3^{d-2}} = i^{-2d+2} = -1,
\end{align*}
wenn man beachtet, dass $d$ gerade ist. Dass wir dabei alle regulären Geraden, soll wieder nur am Beispiel der Klasse~(I) gezeigt werden: wir wollen zeigen, dass $(\nu_2 \nu_3 / \theta, \nu_2 / \nu_3)$ den Bereich $(\mu_{2d} \setminus \mu_d) \times (\mu_{2d} \setminus \mu_d)$ durchläuft. Setze dazu $\nu_2$ auf einen beliebigen zugelassenen Wert. Setzt man nun für~$\nu_3$ alle $d$ zugelassenen Werte ein, dann nimmt $\nu_2 / \nu_3$ ebenfalls $d$ verschiedene Werte an. Wie wir gerade gesehen haben, sind diese aus $\mu_{2d} \setminus \mu_d$ und diese Menge hat genau~$d$ Elemente.

Wegen $\nu_3^2 \in \mu_d$ ist dann auch $\nu_2 \nu_3 \in \mu_{2d} \setminus \mu_d$. Wenn~$\theta$ wiederum alle Werte in~$\mu_d$ durchläuft, nimmt $\nu_2 \nu_3 / \theta$ unabhängig von der Wahl von~$\nu_3$ alle Werte in~$\mu_{2d} \setminus \mu_d$ an. Also werden tatsächlich alle regulären Geraden aufgezählt.
\end{proof}

\chapter{Symmetrien der Geradenkonfiguration} \label{chap:configsymm}
Die hohe Zahl an Geraden auf \textsc{Fermat}-Flächen legt es nahe, dass ihre Struktur durch die Geraden in hohem Maß festgelegt wird. Hier wollen wir den Zusammenhang untersuchen zwischen semilinearen Symmetrien der Fläche und Permutationen der Geraden, die das Schnittverhalten respektieren.

Wir führen für dieses Kapitel neue Notation ein: den Primkörper von $K$ bezeichnen wir als $k$. Ohne Einschränkung sei $K = \overline k$ angenommen, denn offenbar sind alle Geraden über $\overline k$ definiert.

\section{Lineare und Kombinatorische Symmetrien}
Die Fermat-Flächen haben einen hohen Grad an Symmetrie. Offenbar lassen sowohl Permutationen der Koordinaten als auch Multiplikation der Koordinaten mit $d$-ten Einheitswurzeln die Fläche invariant. Das sind alles lineare Transformationen, diese reichen aber nicht, um alle Symmetrien der Geradenkonfiguration zu erklären. Dazu müssen wir noch die Automorphismengruppe von $K$ hinzunehmen. Also operieren die Gruppen $S_4$, $\Gal K/k$ und $\mu_d^4$, in letzterer operiert eine Untergruppe isomorph zu $\mu_d$ trivial: Multiplikation aller Koordinaten mit derselben Einheitswurzel ändert nichts. Die Gruppen schneiden sich nur in $\{\id\}$, also haben wir eine Aktion von
\begin{equation}
\mu_d^4 / \mu_d \rtimes (S_4 \times \Gal K/k) \subset \PGaL 4K,
\end{equation}
auf $F_d$, wobei der Homomorphismus $S_4 \times \Gal K/k \rightarrow \Aut(\mu_d^4 / \mu_d)$ so definiert ist: $(\sigma, \tau)$ wird abgebildet auf $\mu_d^4 / \mu_d \to \mu_d^4 / \mu_d$, $(x_0:\dots:x_3) \mapsto (\tau(x_{\sigma(0)}):\dots:\tau(x_{\sigma(3)}))$. Wenn es Geistergeraden gibt, treten sogar noch mehr Symmetrien auf, wie wir später sehen werden.

\begin{defin}
Sei $S \in \proj n(K)$ eine beliebige Menge. Als semilineare Symmetriegruppe von $S$ bezeichnen wir die Untergruppe
\begin{equation}
G_l = G_l(S) = \{ g \in \PGaL{n+1}K: gS = S \} \subset \PGaL{n+1}K.
\end{equation}
\end{defin}

Solche semilinearen Abbildungen lassen nicht nur die Fläche invariant, sondern schicken auch Geraden auf Geraden. Damit permutieren sie die Geraden auf der Fläche, offenbar bleibt dabei aber ihre Schnittkonfiguration erhalten. Die Schnittkonfiguration wird durch einen Graphen $\mathcal G = (L,E)$ kodiert, dabei ist die $L$ die Menge der Geraden, und $(l_1, l_2) \in E$, wenn $l_1$ und $l_2$ sich schneiden.

Die Frage liegt nahe, welche Permutationen der Geraden es denn gibt, die die Schnittkonfiguration erhalten.
\begin{defin}
Sei nun $S \in \proj 3(K)$ eine projektive Fläche, $\mathcal G = (L,E)$ die Geradenkonfiguration. Dann ist die kombinatorische Symmetriegruppe von $S$ die Automorphismengruppe des Graphen $\mathcal G$, d.h.
\begin{equation}
G_k = G_k(S) = \{ \sigma \in \Sym(L): (l_1, l_2) \in E \Leftrightarrow (\sigma(l_1), \sigma(l_2)) \in E \}
\end{equation}
\end{defin}

Wir wollen in diesem Kapitel die Beziehung zwischen diesen beiden Gruppen ausarbeiten. Wie oben bemerkt, induziert jede lineare Symmetrie eine kombinatorische, also haben wir einen Homomorphismus $G_l(F_d) \to G_k(F_d)$. Wir berechnen zunächst den Kern, und zeigen später Surjektivität. Die folgende Betrachtung ist inspiriert durch \cite[Bem.~4.10.1, S.~404]{Hartshorne}, und \cite[Aufg. C--D, S.~180]{Mumford}. Dort wird die Situation für allgemeine reguläre Flächen dritten Grades untersucht.

An dieser Stelle bemerken wir, dass $\GaL 4K = \GL 4K \rtimes \Aut K$ und $\PGaL 4K = \PGL 4K \rtimes \Aut K$.\footnote{Das folgt leicht aus den Bemerkungen in \cite[S.~2--3]{Dieudonne}.}

\begin{lemma}
Gibt es unter den Schnittpunkten der Geraden auf $S$ mindestens fünf in allgemeiner Lage,\footnote{das heißt: keine vier davon liegen auf einer projektiven Ebene.} so besteht der Kern des Homomorphismus $G_l(S) \to G_k(S)$ in einer geeigneten Basis nur aus Automorphismen von $K$.
\end{lemma}
\begin{proof}
Sei $g \in \PGaL 4K$ so gewählt, dass alle Geraden auf $S$ auf sich selbst abgebildet werden. Dann werden auch die fünf Schnittpunkte auf sich selbst abgebildet. Sei $(A, \sigma) \in \GaL 4K = \GL 4K \rtimes \Aut K$ ein Urbild von $g$ und $v_1, \dots, v_5$ Urbilder der Schnittpunke in $K^4$, dann gilt also $(A, \sigma)v_i = \lambda_i v_i$ mit $\lambda_i \in K^*$. Sei $P$ die Matrix mit den Spalten $v_1, \dots, v_4$, dann bildet also $(P^{-1},\id)(A,\sigma)(P,\id) = (P^{-\sigma}AP,\sigma)$ die Basisvektoren auf Vielfache ihrer selbst ab. Folglich hat die Matrix $P^{-\sigma}AP$ Diagonalgestalt. (Hier ist $P^{-\sigma}$ eine Kurzschreibweise für $(P^{-1})^\sigma$.)

Betrachte nun den Vektor $v_5 \in K^4 \setminus 0$ zum fünften Schnittpunkt. In der Basis $\{v_i\}_{i<5}$ hat $v_5$ dann die Gestalt $\sum_i \alpha_i v_i$ mit $\alpha_i \neq 0$ für alle $i$, da die fünf Schnittpunkte in allgemeiner Lage sind. Nun gilt aber $(A,\sigma) v_5 = \lambda_5 v_5$, oder $(P^{-\sigma}AP,\sigma)(\alpha_1,\dots,\alpha_4) = \lambda_5(\lambda_1\alpha_1,\dots,\lambda_4\alpha_4)$. Damit sind die Diagonaleinträge $\lambda_1, \dots, \lambda_4$ der Matrix $P^{-\sigma}AP$ alle gleich $\lambda$, da die $\alpha_i$ nicht verschwinden. Also ist $P^{-\sigma}AP = \lambda \id$, mithin ist $g$ zu $(\id, \sigma)$ linear konjugiert. Man beachte dabei, dass $P$ unabhängig von $g$ ist: man kann also durch einen Basiswechsel alle Elemente des Kerns gleichzeitig auf die Form $(\id, *)$ bringen.
\end{proof}
Wie wir weiter unten sehen werden, schneiden sich zwei Geraden einer Klasse in Punkten der Form $(1:\zeta^n:0:0)$, $\zeta \in \mu_{2d}$ primitiv und $n$ ungerade; sowie Permutationen der Koordinaten. Durch Probieren findet man damit leicht fünf Punkte in allgemeiner Lage. Die fünf $4 \times 4$-Untermatrizen von
\begin{equation*}
\begin{pmatrix}
1 & 0 & 1 & 0 & 1 \\
\zeta & 0 & 0 & 1 & 0 \\
0 & 1 & \zeta^3 & 0 & 0 \\
0 & \zeta & 0 & \zeta^5 & \zeta^3 \\
\end{pmatrix}
\end{equation*}
haben Determinanten $(\zeta^2-1)\zeta^4$, $(\zeta+1)\zeta^4$, $-(\zeta^3+1)\zeta^3$, $-(\zeta^3+1)\zeta^6$ bzw.~$(\zeta+1)\zeta^3$. Für $d>3$ verschwindet keine davon. Im Folgenden nehmen wir die ersten drei Vektoren als Basis.

Nun müssen wir noch ermitteln, welche Automorphismen im Kern liegen. Offenbar gilt im Fall der regulären Geraden $\Gal K/{k(\mu_{2d})} \subset \ker(G_l \to G_k)$, gibt es Geistergeraden, dann $\Gal K/{k(\mu_{d(d-2)})} \subset \ker(G_l \to G_k)$. Wir wollen nun zeigen, dass der Kern nicht größer ist.
\begin{lemma}
Sei $E$ die Körpererweiterung von $k$, die durch die Quotienten der Plückerkoordinaten der Geraden auf der Fläche $S$ erzeugt wird. Der Kern des Homomorphismus $G_l(S) \to G_k(S)$ ist dann $\Gal K/E$.
\end{lemma}
\begin{proof}
Sei die Basis des Raumes wie im vorigen Lemma gewählt, sodass $\ker(G_l \to G_k) \subset \Aut K$. Mit den Geraden sind auch die Schnittpunkte über $E$ definiert, die Verhältnisse der Plückerkoordinaten der transformierten Geraden erzeugen daher auch $E$.

Da diese von einer semilinearen Symmetrie aus dem Kern fix gelassen werden sollen, müssen solche die Verhältnisse der Plückerkoordinaten fix lassen. Damit folgt $\ker(G_l \to G_k) \subset \Gal K/E$. Die Umkehrung ist offensichtlich.
\end{proof}

Damit können wir den Kern explizit angeben: im Fall $d \neq p^n+1$, wenn also keine Geistergeraden existieren, ist $E = k(\mu_{2d})$. Weiter haben wir oben gesehen, dass $\mu_d^4 / \mu_d \rtimes (S_4 \times \Gal K/k) \subset G_l(F_d)$. Nun gilt nach \textsc{Galois}-Theorie
\begin{equation}
\Gal K/k \;/\; \Gal K/{k(\mu_{2d})} \cong \Gal k(\mu_{2d})/k;
\end{equation}
damit folgt, dass $G_k(F_d)$ eine Untergruppe isomorph zu $\mu_d^4 / \mu_d \rtimes (S_4 \times \Gal \mathbb Q(\mu_{2d})/{\mathbb Q})$ enthält. Wir wollen im nächsten Abschnitt zeigen, dass sogar Gleichheit gilt.

\section{Reguläre Geraden}
\paragraph{Konfiguration} Wir schreiben die Geraden aus \eqref{eq:regular} mit einer fixierten primitiven Einheitswurzel $\zeta \in \mu_{2d}$ und $a,b \in (2\mathbb Z + 1)/2d\mathbb Z \subset \Zmod 2dZ$:
\begin{equation}
\begin{split}
\Lcl(I)_{a,b}  :\qquad	&\langle (1,\zeta^a,0,0), (0,0,1,\zeta^b)\rangle \\
\Lcl(II)_{a,b} :\qquad	&\langle (1,0,\zeta^a,0), (0,1,0,\zeta^b)\rangle \\
\Lcl(III)_{a,b}:\qquad	&\langle (1,0,0,\zeta^a), (0,1,\zeta^b,0)\rangle.
\end{split}
\end{equation}
Ob sich zwei verschiedene projektive Geraden schneiden, stellt man anhand der Determinante der Matrix aus ihren vier Basisvektoren fest: verschwindet sie, dann hat die Matrix nicht vollen Rang, also schneiden sie sich in einem projektiven Punkt.

Damit überlegt man sich leicht, dass sich zwei Geraden aus derselben Familie genau dann schneiden, wenn sie in einem der beiden Parameter $a$, $b$ übereinstimmen. Für Geraden aus verschiedenen Klassen ergibt sich folgendes: zwei Geraden $\Lcl(I)_{a,b}$ und $\Lcl(II)_{a',b'}$ schneiden sich, wenn
\begin{equation}
\det \begin{pmatrix}
1 & \zeta^a & 0 & 0 \\
0 & 0 & 1 & \zeta^b \\
1 & 0 & \zeta^{a'} & 0 \\
0 & 1 & 0 & \zeta^{b'}
\end{pmatrix} = 0 \quad\Longleftrightarrow\quad \zeta^{a'} \zeta^b = \zeta^a \zeta^{b'} \quad\Leftrightarrow\quad a-b \equiv a'-b' \pmod{2d}.
\end{equation}
Analog erhält man, dass sich $\Lcl(I)_{a,b}$ und $\Lcl(III)_{a'',b''}$ schneiden, wenn $a''-b'' \equiv a+b$, und $\Lcl(II)_{a',b'}$ und $\Lcl(III)_{a'',b''}$, wenn $a''+b'' \equiv a'+b' \pmod{2d}$.

Wir wollen zur Beschreibung der Schnittkonfiguration noch einige Begriffe einführen: die drei Teilmengen von Geraden $\Lcl(I)$, $\Lcl(II)$ und $\Lcl(III)$ wollen wir Klassen nennen. Die Teilmengen davon mit konstantem Parameter $b$ nennen wir Zeilen, die mit konstantem Parameter $a$ Spalten. Dann bilden die Zeilen und Spalten vollständige Graphen $K_d$.

Teilmengen der Klassen mit konstanter Summe bzw. Differenz der Parameter $a$, $b$ mögen Diagonalen heißen. Bestimmte Diagonalen verschiedener Klassen bilden bipartite Graphen $K_{d,d}$. In der folgenden Grafik sind einige dieser Diagonalen dargestellt. Jede Gerade in einer der Diagonalen schneidet jede andere in der Diagonalen gleicher Farbe in der entsprechenden anderen Klasse.

\begin{figure}[h]
\centering
\begin{tikzpicture}
\draw[color=gray]
	[xshift=0cm] (0,0) rectangle (2,2) (1,1) node{$\Lcl(I)$}
	[xshift=3cm] (0,0) rectangle (2,2) (1,1) node{$\Lcl(II)$}
	[xshift=3cm] (0,0) rectangle (2,2) (1,1) node{$\Lcl(III)$};
\draw[->,xshift=0cm] (0,2.2) -- (2,2.2) node[above,midway] {$a$};
\draw[->,xshift=0cm] (-0.2,2) -- (-0.2,0)  node[left,midway] {$b$};
\draw[->,xshift=3cm] (0,2.2) -- (2,2.2) node[above,midway] {$a'$};
\draw[->,xshift=3cm] (-0.2,2) -- (-0.2,0)  node[left,midway] {$b'$};
\draw[->,xshift=6cm] (0,2.2) -- (2,2.2) node[above,midway] {$a''$};
\draw[->,xshift=6cm] (-0.2,2) -- (-0.2,0)  node[left,midway] {$b''$};
\draw[color=red]
	[xshift=0cm] (0.5,0) -- (0,0.5) (0.5,2) -- (2,0.5)
	[xshift=3cm] (0.5,0) -- (0,0.5) (0.5,2) -- (2,0.5);
\draw[color=green]
	[xshift=3cm] (1.3,0) -- (2,0.7) (0,0.7) -- (1.3,2)
	[xshift=3cm] (1.3,0) -- (2,0.7) (0,0.7) -- (1.3,2);
\draw[color=blue]
	[xshift=0cm] (0,1.7) -- (0.3,2) (0.3,0) -- (2,1.7)
	[xshift=6cm,yscale=-1,yshift=-2cm]
		(0,1.7) -- (0.3,2) (0.3,0) -- (2,1.7);
\end{tikzpicture}
\caption{Schnitte zwischen verschiedenen Klassen der regulären Geraden}\label{fig:reg}
\end{figure}

\paragraph{Symmetrien} Wir haben uns bereits klargemacht, dass wir auf $F_d$ eine Aktion der Gruppe $\mu_d^4 / \mu_d \rtimes (S_4 \times \Gal \overline k/k)$ haben, deren Quotient nach $\Gal \overline k/{k(\mu_{2d})}$ auf der Geradenkonfiguration nichttrivial operiert. Machen wir uns zunächst klar, auf welche Weise das geschieht. Multiplikation der $i$-ten Koordinate mit $\zeta^{s_i}$, $s_i$ gerade, bewirkt offenbar eine Verschiebung der Konfiguration in den einzelnen Klassen, und zwar wie folgt:

\begin{figure}[h]
\centering
\begin{tikzpicture}
\draw[color=black]
	[xshift=0cm] (0,0) rectangle (2,2) (1,1) node{$\Lcl(I)$}
	[xshift=5cm] (0,0) rectangle (2,2) (1,1) node{$\Lcl(II)$}
	[xshift=5cm] (0,0) rectangle (2,2) (1,1) node{$\Lcl(III)$};
\draw[->,xshift=0cm] (0.5,2.2) -- (1.5,2.2) node[above,midway] {$+s_0$};
\draw[->,xshift=0cm] (-0.2,1.5) -- (-0.2,0.5) node[left,midway] {$+s_2$};
\draw[->,xshift=0cm] (1.5,-0.2) -- (0.5,-0.2) node[below,midway] {$+s_1$};
\draw[->,xshift=0cm] (2.2,0.5) -- (2.2,1.5) node[right,midway] {$+s_3$};

\draw[->,xshift=5cm] (0.5,2.2) -- (1.5,2.2) node[above,midway] {$+s_0$};
\draw[->,xshift=5cm] (-0.2,1.5) -- (-0.2,0.5) node[left,midway] {$+s_1$};
\draw[->,xshift=5cm] (1.5,-0.2) -- (0.5,-0.2) node[below,midway] {$+s_2$};
\draw[->,xshift=5cm] (2.2,0.5) -- (2.2,1.5) node[right,midway] {$+s_3$};

\draw[->,xshift=10cm] (0.5,2.2) -- (1.5,2.2) node[above,midway] {$+s_0$};
\draw[->,xshift=10cm] (-0.2,1.5) -- (-0.2,0.5) node[left,midway] {$+s_1$};
\draw[->,xshift=10cm] (1.5,-0.2) -- (0.5,-0.2) node[below,midway] {$+s_3$};
\draw[->,xshift=10cm] (2.2,0.5) -- (2.2,1.5) node[right,midway] {$+s_2$};
\end{tikzpicture}
\caption{Aktion von $\mu_d^4 / \mu_d$ auf der Geradenkonfiguration}\label{fig:coordmult}
\end{figure}

Die \textsc{Galois}-Automorphismen aus dem Quotienten $\Gal {k(\mu_{2d})}/k$ wirken ähnlich: ein Automorphismus $\zeta \mapsto \zeta^t$ mit einer fixierten primitiven Einheitswurzel $\zeta \in \mu_{2d}$ und $t \in (\Zmod 2dZ)^*$ bewirkt eine Multiplikation der Indizes $a$, $b$ einer Klasse mit $t$.

\begin{figure}[h]
\centering
\begin{tikzpicture}
\draw[color=black]
	[xshift=0cm] (0,0) rectangle (2,2) (1,1) node{$\Lcl(I)$}
	[xshift=5cm] (0,0) rectangle (2,2) (1,1) node{$\Lcl(II)$}
	[xshift=5cm] (0,0) rectangle (2,2) (1,1) node{$\Lcl(III)$};
\draw[->>,xshift=0cm] (0.5,2.2) -- (1.5,2.2) node[above,midway] {$\phantom{x} \cdot t$};
\draw[->>,xshift=0cm] (-0.2,1.5) -- (-0.2,0.5) node[left,midway] {$\phantom{x} \cdot t$};

\draw[->>,xshift=5cm] (0.5,2.2) -- (1.5,2.2) node[above,midway] {$\phantom{x} \cdot t$};
\draw[->>,xshift=5cm] (-0.2,1.5) -- (-0.2,0.5) node[left,midway] {$\phantom{x} \cdot t$};

\draw[->>,xshift=10cm] (0.5,2.2) -- (1.5,2.2) node[above,midway] {$\phantom{x} \cdot t$};
\draw[->>,xshift=10cm] (-0.2,1.5) -- (-0.2,0.5) node[left,midway] {$\phantom{x} \cdot t$};
\end{tikzpicture}
\caption{Aktion von $\Gal k(\mu_{2d})/k$ auf der Geradenkonfiguration}\label{fig:galois}
\end{figure}

Die Aktion der Koordinatenpermutation ist schwieriger zu beschreiben. Man kann die drei Matrizen wie in Abb.~\ref{fig:perm} gezeigt auf einen Würfel kleben. Dann entspricht die Aktion der $S_4$ der Drehgruppe des Würfels.

\begin{figure}[h] % we might need transform canvas={...} or sloping and slanting node text (http://tex.stackexchange.com/questions/62038/text-placed-in-pespective-on-3d-object)
\centering
\begin{tikzpicture}[x  = {(0.9659cm,0.25882cm)},
                    y  = {(-0.5cm,0.5cm)},
                    z  = {(0cm,1cm)}, scale = 2]
\begin{scope}[canvas is yx plane at z=0]
  \path[draw=black] (0,0) rectangle (2,2);
  \path[fill=gray!80] (0.5,0.5) rectangle (1.5,1.5);
\end{scope}
\path (0,1,0) -- node[sloped, rotate=180, xslant=-0.6]{$\Lcl(III)$} (2,1,0);
\begin{scope}[canvas is zx plane at y=2]
  \path[draw=black] (0,0) rectangle (2,2);
  \path[fill=gray!80] (0.5,0.5) rectangle (1.5,1.5);
\end{scope}
\path (1,2,0) -- node[sloped, xslant=-0.4]{$\Lcl(I)$} (1,2,2);
\begin{scope}[canvas is zy plane at x=2]
  \path[draw=black] (0,0) rectangle (2,2);
  \path[fill=gray!80] (0.5,0.5) rectangle (1.5,1.5);
\end{scope}
\path (2,0,1) -- node[sloped, xscale=-1, xslant=0.8]{$\Lcl(II)$} (2,2,1);
\begin{scope}[canvas is zx plane at y=0]
  \path[draw=black] (0,0) rectangle (2,2);
  \path[fill=gray!50, fill opacity=0.5] (0.5,0.5) rectangle (1.5,1.5);
\end{scope}
\path (0,0,1) -- node[sloped, xslant=0.3]{$\Lcl(I)$} (2,0,1);
\begin{scope}[canvas is zy plane at x=0]
  \path[draw=black] (0,0) rectangle (2,2);
  \path[fill=gray!50, fill opacity=0.5] (0.5,0.5) rectangle (1.5,1.5);
\end{scope}
\path (0,1,2) -- node[sloped, yscale=-1, xslant=-0.8]{$\Lcl(II)$} (0,1,0);
\begin{scope}[canvas is yx plane at z=2]
  \path[draw=black] (0,0) rectangle (2,2);
  \path[fill=gray!50, fill opacity=0.5] (0.5,0.5) rectangle (1.5,1.5);
\end{scope}
\path (1,2,2) -- node[sloped, rotate=180, xslant=0.5]{$\Lcl(III)$} (1,0,2);
\end{tikzpicture}
\caption{Visualisierung der Aktion von $S_4$ auf der Geradenkonfiguration}\label{fig:perm}
\end{figure}

\paragraph{Vergleich der Symmetriegruppen} Wir zeigen nun, dass diese Symmetrien der Geradenkonfiguration die einzigen sind, die das Schnittverhalten respektieren. Zunächst müssen wir die Konfiguration aber genauer untersuchen.

\begin{lemma}
Sei $d \geq 4$, dann sind die einzigen Subgraphen isomorph zu $K_d$ die aus den Geraden einer Klasse bestehenden, die in einem der beiden Parameter übereinstimmen.
\end{lemma}
\begin{remarks}
Für $d=3$ ist das nicht der Fall: die drei Schnittpunkte der Diagonalen in Abb.~\ref{fig:reg} bilden ebenfalls einen $K_3$. (Die Grafik ist hier etwas irreführend: die Diagonalen schneiden sich für ungerades $d$ nur genau einmal.)
\end{remarks}
\begin{proof}
Sei $l \in L$ eine beliebige Gerade, dann schneidet sie jeweils $d-1$ Geraden in derselben Zeile bzw.~Spalte und jeweils $d$ Geraden in den entsprechenden Diagonalen in den anderen beiden Klassen. Wir wollen nun untersuchen, zu welchen vollständigen Graphen $K_d$ die Gerade $l$ gehört.

Von den $d$ Geraden in anderen Klassen, die $l$ schneiden, kann jeweils nur eine Teil des $K_d$ sein: denn diese schneiden sich untereinander nicht, da die Diagonalen von jeder Zeile und jeder Spalte jeweils nur eine Gerade beinhalten. Weiterhin gibt es keine Schnitte zwischen den Geraden in den anderen Klassen, die $l$ schneiden, und denen in derselben Klasse: denn die Geraden in den anderen Klassen schneiden die aus $l$'s Klasse genau dann, wenn sie in derselben Diagonale wie $l$ liegen. Die Diagonalen durch $l$ und die Zeile und Spalte von $l$ schneiden sich aber nur in $l$.

Da es zwischen den Geraden der Zeile von $l$ und der Spalte von $l$ keine weiteren Schnitte gibt, bleiben also drei Möglichkeiten für vollständige Graphen: die Geraden in der Zeile und Spalte von $l$ bilden jeweils einen $K_d$, und $l$ zusammen mit zwei ausgewählten Schnittpunkten in den anderen Klassen bildet einen $K_3$. Mehr ist aber nicht drin, und für $d \geq 4$ ist das nicht genug. Also sind sind alle auftretenden Subgraphen $K_d$ die Zeilen und Spalten einer Klasse.
\end{proof}
Das nächste Resultat besagt, dass kombinatorische Symmetrien die Partition in die Klassen respektieren.
\begin{lemma}
Sei $d \geq 4$, $\sigma \in G_k$ eine kombinatorische Symmetrie und $l_1, l_2 \in L$ zwei Geraden. Sind $l_1$ und $l_2$ in derselben Klasse, dann auch ihre Bilder $\sigma(l_1)$, $\sigma(l_2)$.
\end{lemma}
\begin{proof}
Das vorige Lemma besagt, dass alle vollständigen Subgraphen $K_d$ vollständig in einer Klasse enthalten sind. Gleichzeitig wissen wir, dass jede Gerade zu zwei solchen Subgraphen gehört. Damit ist dieses Lemma offensichtlich: man setze die symmetrische, reflexive Relation
\begin{equation*}
l_1 \sim l_2 \Longleftrightarrow l_1, l_2 \text{ gehören zu einem gemeinsamen Subgraphen } K_d
\end{equation*}
auf den Geraden transitiv fort, dann sind die Äquivalenzklassen genau die drei Klassen $\Lcl(I)$, $\Lcl(II)$, $\Lcl(III)$. Weil die Relation sich offenbar mit Symmetrien verträgt, bleiben also die drei Klassen invariant.
\end{proof}

Damit haben wir einen Homomorphismus $\pi: G_k \to S_3$, der eine kombinatorische Symmetrie auf die entsprechende Permutation der Klassen schickt. Der Kern besteht dann aus denjenigen Symmetrien, die alle drei Klassen invariant lassen. Sei $\mathcal G'$ der Graph einer Klasse, d.\,h.~ein Graph mit Knotenmenge $((2\mathbb Z+1)/2d\mathbb Z)^2$ und Kanten $(a,b) \sim (a',b') \Longleftrightarrow a=a' \vee b=b'$. Dann gilt offenbar
\begin{equation}
\ker(G_k \to S_3) \subseteq (\Aut \mathcal G')^3.
\end{equation}
Daher liegt es nahe, zunächst $\Aut \mathcal G'$ zu untersuchen.

Offenbar können sowohl die Zeilen als auch die Spalten beliebig untereinander permutiert werden, und eine Vertauschung von Zeilen und Spalten (Transposition) ist auch möglich. Wir zeigen jetzt, dass mehr nicht geht.
\begin{lemma}
Betrachte den Subgraph $\mathcal G'$ einer Klasse $\Lcl(I/II/III)$. Sei $\sigma \in \Aut \mathcal G'$, dann bildet $\sigma$ entweder Zeilen auf Zeilen und Spalten auf Spalten ab, oder Zeilen auf Spalten und Spalten auf Zeilen.
\end{lemma}
\begin{proof}
Wir betrachten die Zeile $(2\mathbb Z+1)/2d\mathbb Z \times \{1 + 2d\mathbb Z\}$, ihr Bild ist eine Zeile oder Spalte. Ohne Beschränkung der Allgemeinheit können wir annehmen, dass sie auf sich selbst abgebildet wird. (Sonst transponiere entsprechend und permutiere die Zeilen, das ändert nichts an der Aussage.)

Die Menge der Spalten ist nun $\{\{i + 2d\mathbb Z\} \times (2\mathbb Z+1)/2d\mathbb Z : i \in (2\mathbb Z+1)/2d\mathbb Z\}$. Die Knoten einer Spalte haben die Eigenschaft, dass sie alle mit einem und demselben Knoten in der Zeile $(2\mathbb Z+1)/2d\mathbb Z \times \{1 + 2d\mathbb Z\}$ verbunden sind. Damit sind ihre Bilder alle mit einem und demselben Knoten im Bild der Zeile $(2\mathbb Z+1)/2d\mathbb Z \times \{1 + 2d\mathbb Z\}$ verbunden. Folglich sind die Bilder der Spalten wieder Spalten. Damit werden aber auch die restlichen Zeilen auf Zeilen abgebildet. Was anderes bleibt ihnen ja nicht übrig, da es nur $2d$ vollständige Subgraphen $K_d$ in $\tilde{\mathcal G}$ gibt.
\end{proof}

Man kann also die Automorphismengruppe $\Aut \mathcal G'$ in das Kranzprodukt $S_d^2 \rtimes_{\text{kan.}} S_2$ einbetten. Dabei kodieren die Faktoren $S_d$ jeweils die Permutation der Zeilen bzw.~Spalten untereinander, und der Faktor $S_2$, ob Zeilen und Spalten miteinander vertauscht wurden. Wir wollen ein Element $((\sigma_1, \sigma_2), \tau) \in S_d^2 \rtimes S_2$ dabei so verstehen, dass zuerst $\sigma_1$ die Zeilen und $\sigma_2$ die Spalten permutiert, und dann Zeilen und Spalten vertauscht werden, falls $\tau \neq \id$. Mit den Schnitten zwischen verschiedenen Klassen werden wir das nun auf eine deutlich kleinere Gruppe einschränken.
\begin{prop}
Zwei Geraden einer Klasse liegen auf derselben Diagonale genau dann, wenn sie einander nicht schneiden, aber es mindestens $d$ Geraden gibt, die beide schneiden.
\end{prop}
\begin{proof}
Die eine Richtung ist klar. Seien nun zwei Geraden $l_1$, $l_2$ gegeben mit $l_1 \not\sim l_2$, sodass aber $d$ Geraden $k_1, \dots, k_d \in L$ existieren mit $l_1 \sim k_i \sim l_2$ für $1 \leq i \leq d$. Wegen $l_1 \not\sim l_2$ liegen beide nicht in derselben Zeile oder Spalte, also liegen auch die $k_i$ nicht in derselben Zeile oder Spalte, sie liegen also in anderen Klassen.

Die zu $l_1$, $l_2$ inzidenten Geraden in anderen Klassen liegen auf jeweils parallelen Diagonalen in diesen Klassen. Gemeinsame inzidente Geraden gibt es also nur dann, wenn entsprechende Diagonalen zusammenfallen. Damit liegen aber auch $l_1$, $l_2$ auf einer gemeinsamen Diagonale.
\end{proof}
\begin{coroll}
Damit bilden kombinatorische Symmetrien Diagonalen auf Diagonalen ab.
\end{coroll}
\begin{prop}
Eine Symmetrie $\sigma \in \ker \pi$ schickt Diagonalen in einer Klasse auf zu Ihnen parallele Diagonalen.
\end{prop}
\begin{proof}
Wegen Symmetrie der Klassen können wir annehmen, dass es sich um die rote Diagonale in Klasse $\Lcl(I)$ von Abb.~\ref{fig:reg}, d.\,h. eine Diagonale mit konstanter Differenz $a-b$ handelt. Die Geraden auf dieser Diagonale haben gemeinsame inzidente Geraden in $\Lcl(II)$. Da $\sigma$ die Klassen invariant lässt, gilt dies auch für das Bild der Diagonale. Also handelt es sich um eine zur roten parallele Diagonale.
\end{proof}

\begin{lemma}
Ein Automorphismus des Graphen $\mathcal G'$, der Diagonalen auf dazu parallele Diagonalen schickt, hat die Form $((2\mathbb Z+1)/2d\mathbb Z)^2 \ni (x,y) \mapsto (Cx+D_1, Cy+D_2) \in ((2\mathbb Z+1)/2d\mathbb Z)^2$ oder $(x,y) \mapsto (Cy+D_2, Cx+D_1)$ mit $C \in (\Zmod 2dZ)^*$, $D_1, D_2 \in 2\Zmod 2dZ$.
\end{lemma}
\begin{proof}
Sei also $((\sigma_1, \sigma_2), \tau) \in S_d^2 \rtimes S_2$ ein Automorphismus von $\mathcal G'$ und $\{(a,b) : a-b \equiv c \pmod{2d}\}$ mit $c \in 2\Zmod 2dZ$ eine Diagonale. Dann soll ihr Bild eine dazu parallele Diagonale sein, d.\,h. $\sigma_1(a) - \sigma_2(b) \equiv c' \pmod{2d}$ für alle $(a,b)$ mit $a-b \equiv c \pmod{2d}$. Damit ist $\sigma_2(b) \equiv c' + \sigma_1(a) \equiv c' + \sigma_1(c+b)$, es ergibt sich also $\sigma_2$ aus $\sigma_1$. Subtrahiert man die Gleichungen für $c_1=0$, $c_2=2$, so erhält man
\begin{equation*}
\sigma_1(b+2) - \sigma_1(b) \equiv c_2' - c_1' =: 2C \qquad\Longleftrightarrow\qquad \sigma_1(b+2) \equiv \sigma_1(b) + 2C  \mod{2d}.
\end{equation*}
Man beachte dabei, dass die Differenz $\sigma_1(b+2) - \sigma_1(b)$ gerade ist, damit können wir $c_2' - c_1' = 2C$ setzen.

Damit ist $\sigma_1$ offenbar von der Form $\sigma_1(x) = Cx+D_1$ und $\sigma_2$ entsprechend von der Form $\sigma_2(y) = c' + C(c+y) + D_1 = Cy + (c'+Cc+D_1) = Cy + D_2$, $D_2 := c'+Cc+D_1$. Wegen Bijektivität von $\sigma_{1/2}$ ist $C$ offenbar eine Einheit im Ring $\Zmod 2dZ$. Ebenso sieht man leicht, dass $D_1$ und $D_2$ aus dem Ideal $2\Zmod 2dZ$ stammen.
\end{proof}

Aus den letzten beiden Resultaten folgt, dass sich $\ker \pi$ sogar in das deutlich kleinere Produkt von Gruppen
\begin{equation*}
[(2\Zmod 2dZ)^2 \rtimes_\alpha ((\Zmod 2dZ)^* \times S_2)]^3
\end{equation*}
einbetten lässt. Dabei bildet $\alpha \colon ((\Zmod 2dZ)^* \times S_2) \to \Aut ((2\Zmod 2dZ)^2)$ das Paar $(C,\tau)$ auf $(x_1,x_2) \mapsto (Cx_{\tau(1)},Cx_{\tau(2)})$ ab. Für eine weitere Reduktion der kombinatorischen Symmetriegruppe wollen wir nun die Aktionen auf den einzelnen Klassen miteinander korrelieren. Dazu untersuchen wir zunächst, wie obige Gruppe auf den Diagonalen wirkt.

Die Diagonalen $D_n^{k,l}$ bilden selbst auch einen Graphen: sie zerfallen in drei Klassen, die wir mit $(k,l) \in \{\mathrm{I}, \mathrm{II}, \mathrm{III}\} \times \{+,-\}$ indizieren wollen. Die erste Komponente steht für die Klasse, die zweite dafür, ob die Diagonale durch eine konstante Summe oder eine konstante Differenz der Parameter charakterisiert ist. Weiterhin ist eine Diagonale natürlich durch ebenjene Summe bzw.~Differenz $n \in 2\Zmod 2dZ$ bestimmt. Die Kanten sind dann wie folgt gegeben:
\begin{itemize}
\item $D_n^{\mathrm{I},-} \sim D_n^{\mathrm{II},-}$ für alle $n \in 2\Zmod 2dZ$,
\item $D_n^{\mathrm{I},+} \sim D_n^{\mathrm{III},-}$ für alle $n \in 2\Zmod 2dZ$,
\item $D_n^{\mathrm{II},+} \sim D_n^{\mathrm{III},+}$ für alle $n \in 2\Zmod 2dZ$.
\end{itemize}
Ein Automorphismus des Graphen $\mathcal G$ induziert dann offenbar einen Automorphismus des Diagonalengraphen. Ein Automorphismus aus $\ker \pi$ lässt nach obiger Proposition auch die Klassen $(k,l)$ von Diagonalen invariant. Folglich ist die Aktion von $\ker \pi$ auf den Diagonalen in $S_d^6$. Mit vorigem Lemma können wir das erheblich einschränken.

\begin{lemma}
Eine kombinatorische Symmetrie aus $\ker \pi$ von der Form $(D_1, D_2, C, \tau)^{\{\mathrm{I, II, III}\}} \in [(2\Zmod 2dZ)^2 \rtimes_\alpha ((\Zmod 2dZ)^* \times S_2)]^3$ operiert auf den Diagonalen durch $D_n^{k,l} \mapsto D_{\phi_l^k(n)}^{k,l}$ mit $\phi_+^k(n) = C^k n + D_1^k + D_2^k$ und $\phi_-^k(n) = \sign(\tau^k)(C^k n + D_1^k - D_2^k)$. Dabei ist $\sign$ das Signum einer Permutation, also hier der eindeutige Isomorphismus $S_2 \overrel\to^\sim \{\pm 1\}$.
\end{lemma}
\begin{proof}
Die Symmetrie $(D_1, D_2, C, \tau)^{\{\mathrm{I, II, III}\}}$ lässt sich wie folgt zerlegen: zuerst werden die Parameter $a$ und $b$ einer Klasse $k$ mit dem entsprechenden $C^k$ multipliziert, dann wird auf $a$ das entsprechende $D_1^k$ und auf $b$ das entsprechende $D_2^k$ addiert. Zuletzt werden beide Parameter vertauscht, falls $\tau^k \neq \id$.

Das übersetzt sich in folgende Operationen auf den Diagonalen: die Multiplikation der Parameter $a$, $b$ mit $C^k$ bewirkt offenbar eine Multiplikation der Summe/Differenz $n$ mit $C^k$. Die Verschiebungen von $a$, $b$ um $D_{1/2}^k$ bewirken eine entsprechende Verschiebung der Summe/Differenz um denselben Betrag. Hier muss man nur auf das Vorzeichen achten: Verschiebungen in $a$ wirken sich auf Summe und Differenz positiv aus, Verschiebungen in $b$ auf die Summe positiv, auf die Differenz jedoch negativ.

Zuletzt die Transposition: sie lässt die Diagonalen $D_n^{k,+}$ unangetastet, die Differenz der Diagonalen $D_n^{k,-}$ wird jedoch mit $-1$ multipliziert, daher der Faktor $\sign(\tau)$.
\end{proof}

Da nun ein Automorphismus des Diagonalengraphen die Relationen zwischen den Diagonalen erhält, ergeben sich folgende Gleichungen:
\begin{align*}
D_{\phi_-^\mathrm{I}(n)}^{\mathrm{I},-} &\sim D_{\phi_-^\mathrm{II}(n)}^{\mathrm{II},-}
	&\Longrightarrow\qquad \phi_-^\mathrm{I}(n) &= \phi_-^\mathrm{II}(n) \\
D_{\phi_+^\mathrm{I}(n)}^{\mathrm{I},+} &\sim D_{\phi_-^\mathrm{III}(n)}^{\mathrm{III},-}
	&\Longrightarrow\qquad \phi_+^\mathrm{I}(n) &= \phi_-^\mathrm{III}(n) \\
D_{\phi_+^\mathrm{II}(n)}^{\mathrm{II},+} &\sim D_{\phi_+^\mathrm{III}(n)}^{\mathrm{III},+}
	&\Longrightarrow\qquad \phi_+^\mathrm{II}(n) &= \phi_+^\mathrm{III}(n)
\end{align*}
\begin{prop}
Sei $p = aX+b \in \Zmod nZ[X]$ ein lineares Polynom. Dann verschwindet $p$ auf einem Ideal $m\Zmod nZ$ identisch genau dann, wenn $b=0$ und $a \in \frac n m \Zmod nZ$.
\end{prop}
\begin{proof}
Verschwindet $p$ auf $m\Zmod nZ$, dann auch in $0$: damit ist $b=0$. Mit $0 = p(m) - p(0) = am$ folgt dann $a \in \frac n m \Zmod nZ$. Dass die Bedingungen an die Koeffizienten hinreichend sind, ist offensichtlich.
\end{proof}
Da obige Gleichungen lineare Polynomgleichungen sind, die in $n \in 2\Zmod 2dZ$ identisch erfüllt sind, können wir die also Koeffizienten vergleichen. In der ersten Spalte stehen jeweils die Gleichungen für die linearen Koeffizienten modulo $d$, in der zweiten die für die konstanten Koeffizienten.
\begin{align*}
\sign(\tau^\mathrm{I}) C^\mathrm{I} &\equiv \sign(\tau^\mathrm{II}) C^\mathrm{II}
& \sign(\tau^\mathrm{I}) (D_1^\mathrm{I} - D_2^\mathrm{I}) &= \sign(\tau^\mathrm{II}) (D_1^\mathrm{II} - D_2^\mathrm{II}) \\
C^\mathrm{I} &\equiv \sign(\tau^\mathrm{III}) C^\mathrm{III}
& D_1^\mathrm{I} + D_2^\mathrm{I} &= \sign(\tau^\mathrm{III}) (D_1^\mathrm{III} - D_2^\mathrm{III}) \\
C^\mathrm{II} &\equiv C^\mathrm{III}
& D_1^\mathrm{II} + D_2^\mathrm{II} &= D_1^\mathrm{III} + D_2^\mathrm{III}
\end{align*}
Aus den Gleichungen der ersten Spalte folgt dann durch Einsetzen $C^\mathrm{I} \equiv \sign(\tau^\mathrm{III}) C^\mathrm{III} \equiv \sign(\tau^\mathrm{III}) C^\mathrm{II} \equiv \sign(\tau^\mathrm{III})\sign(\tau^\mathrm{I}) / \sign(\tau^\mathrm{II}) C^\mathrm{I}$, oder $\prod_k \sign(\tau^k) = 1$, da $C^\mathrm{I}$ (auch $\mod d$) invertierbar ist und $\sign$ nach $\{\pm 1\}$ abbildet. Mit anderen Worten: entweder gibt es keine oder zwei Transpositionen.

Für ungerades $d$ ist $\Zmod 2dZ \to \Zmod dZ$ injektiv, folglich gilt dann in der ersten Spalten jeweils Gleichheit. Setzen wir $C := C^\mathrm{II} = C^\mathrm{III}$, dann ist $C^\mathrm{I} = \sign(\tau^\mathrm{III}) C$. Es gibt also für ungerade~$d$ nur einen freien Skalierungsparameter $C \in (\Zmod 2dZ)^*$.

Die Gleichungen der zweiten Spalte bilden nun für feste $\sign(\tau^k)$ ein homogenes lineares Gleichungssystem. Offenbar können wir dieses über $\Zmod dZ$ statt $2\Zmod 2dZ$ lösen. Wir interessieren uns also für den Kern der $(\Zmod dZ)$-Modulhomomorphismen $(\Zmod dZ)^6 \to (\Zmod dZ)^3$,
\begin{equation}
v \mapsto
\begin{pmatrix}
\sign(\tau^\mathrm{I}) & -\sign(\tau^\mathrm{I}) & -\sign(\tau^\mathrm{II}) & \sign(\tau^\mathrm{II}) & 0 & 0 \\
1 & 1 & 0 & 0 & -\sign(\tau^\mathrm{III}) & \sign(\tau^\mathrm{III}) \\
0 & 0 & 1 & 1 & -1 & -1
\end{pmatrix}v.
\end{equation}
Die Matrix lässt sich leicht auf Zeilenstufenform bringen, indem man $\sign(\tau^\mathrm{I})$-mal die erste Zeile von der zweiten subtrahiert. Dann sind die Pivoteinträge offenbar $(2,1,\pm 1)$. Zieht man die dritte Zeile noch $\sign(\tau^\mathrm{III})$-mal von der zweiten ab, erhält man
\begin{equation*}
\begin{pmatrix}
\sign(\tau^\mathrm{I}) & -\sign(\tau^\mathrm{I}) & -\sign(\tau^\mathrm{II}) & \sign(\tau^\mathrm{II}) & 0 & 0 \\
0 & 2 & 0 & -2\sign(\tau^\mathrm{III}) & 0 & 2\sign(\tau^\mathrm{III}) \\
0 & 0 & 1 & 1 & -1 & -1
\end{pmatrix}
\end{equation*}
wegen $\prod_k \sign(\tau^k) = 1$. Als Einheit ist $1$ immer invertierbar; und $2$ ist es genau dann, wenn $d$ ungerade ist. Damit ist der Lösungsraum $(\Zmod dZ)^3$ für ungerades $d$ und $(\Zmod dZ)^3 \times (\frac d2\Zmod dZ) \cong (\Zmod dZ)^3 \times \Zmod 2Z$ für gerades $d$.

\begin{theorem}
Sei $d \geq 4$ ungerade und $\Char K=0$. Dann sind die einzigen Symmetrien der Geradenkonfiguration die durch die semilineare Gruppe erzeugten.
\end{theorem}
\begin{proof}
Wir sahen im vorigen Abschnitt, dass für $\Char K=0$ die kombinatorische Symmetriegruppe eine Gruppe isomorph zu $\mu_d^4 / \mu_d \rtimes (S_4 \times \Gal \mathbb Q(\mu_{2d})/{\mathbb Q})$ enthält. Gerade haben wir eine obere Schranke für $G_k$ hergeleitet: nämlich $[C_d^3 \rtimes (C_2^2 \times (\Zmod 2dZ)^*)] \rtimes S_3$ für ungerades $d$.

Beide haben offenbar dieselbe Anzahl Elemente, man sieht sogar leicht den Isomorphismus wegen $S_4/V_4 \cong S_3$. Damit muss $G_l \to G_k$ surjektiv sein.
\end{proof}

\section{Eine Stratifizierung von $F_d(\mathbb F_{q^2})$}
Wir wollen nun die Betrachtung der Konfiguration des Geistergeraden vorbereiten. Dazu folgen noch einige Betrachtungen.

Offenbar sind alle Geraden (reguläre und Geistergeraden) über $\mathbb F_p(\mu_{d(d-2)}) = \mathbb F_{q^2}$ mit $q = p^n$, $d = q+1$, definiert. Damit können wir effektiv $F_d(\mathbb F_{q^2})$ betrachten. Insbesondere wollen wir berechnen, wie groß diese (nun endliche) Menge ist.

Dazu zerlegen wir zunächst $F_d(K)$ in drei Teile:
\begin{align*}
S_0(K) &= \{\text{alle vier Koord.}\neq 0\} = \{(X,Y,Z,W) \in F_d : XYZW \neq 0\} \\
S_1(K) &= \{\text{genau eine Koord.}= 0\} \\
S_2(K) &= \{\text{genau zwei Koord.}= 0\} = \{\text{mind. zwei Koord.}= 0\} \\
	&= \{X = 0, Y = 0\} \cup \{X = 0, Z = 0\} \cup \dots \cup \{Z = 0, W = 0\}
\end{align*}
Dabei ist $S_0$ offen, $S_2$ abgeschlossen. Jede projektive Gerade besteht aus $q^2+1$ Punkten über $\mathbb F_{q^2}$. Die regulären Geraden liegen in $S_0 \cup S_2$: ist eine Koordinate null, dann auch eine andere. Genauer: es liegen zwei Punkte in $S_2$, die anderen $q^2-1$ in $S_0$.

Die Geistergeraden hingegen liegen in $S_0 \cup S_1$: gäbe es einen Punkt aus $S_2$ darauf, könnte man ihn als Basisvektor nehmen. Dann würde aber eine \textsc{Plücker}-Koordinate verschwinden. Man kann wieder präzisieren: es liegen 4 Punkte in $S_1$ und $q^2-3$ in $S_0$.

Wir wollen nun die Anzahl der Lösungen von $X^d+Y^d+Z^d+W^d=0$ in $\mathbb F_{q^2}$ bestimmen. Dazu bemerken wir zunächst, dass $\lambda^{q+1} \in \mathbb F_q^*$ für $\lambda \in \mathbb F_{q^2}^*$. Andersherum gibt es für jedes $\alpha \in \mathbb F_q^*$ genau $q+1$ verschiedene $\beta \in \mathbb F_{q^2}^*$ mit $\beta^{q+1} = \alpha$. Wir suchen daher Lösungen von $A+B+C+D=0$ in $\mathbb F_q$ und liften diese nach $\mathbb F_{q^2}$.

Für $S_0(\mathbb F_{q^2}) = \{\text{alle vier Koord.}\neq 0\}$ suchen wir also Lösungen von $A+B+C+D=0$ mit $A,B,C,D \in \mathbb F_q^*$. Die ersten beiden Koordinaten $A$, $B$ können wir beliebig wählen. Die vorletzte Koordinate $C$ muss dann so gewählt werden, dass die letzte nicht verschwindet. Also $C \neq 0, -A-B$. Für $A+B=0$ fallen beide Fälle zusammen, das passiert $(q-1)$-mal. Also haben wir
\begin{equation*}
\underbrace{(q-1)(q-2)(q-2)}_{A+B \neq 0} + \underbrace{(q-1)(q-1)}_{A+B=0} = (q-1)((q-2)^2+(q-1))
\end{equation*}
Lösungen. Geliftet sind das
\begin{equation}
(q-1)((q-2)^2+(q-1))(q+1)^4/(q^2-1) = ((q-2)^2+(q-1))(q+1)^3
\end{equation}
Punkte in $S_0(\mathbb F_{q^2})$.

Um die Punkte in $S_1(\mathbb F_{q^2}) = \{\text{genau eine Koord.}= 0\}$ zu zählen, platzieren wir zunächst die Null: dafür gibt es vier Möglichkeiten. Für die restlichen Koordinaten gehen wir wie oben vor: hat man die ersten beiden Koordinaten $A$ und $B$ gewählt, ergibt sich $C=-A-B$. Damit das nicht verschwindet, muss $B$ so gewählt werden, dass $A+B \neq 0$. Also wählt man $A \neq 0$ beliebig und dann $B \neq 0, -A$. Es ergeben sich $4(q-1)(q-2)$ Möglichkeiten. Insgesamt gibt es also in $S_1(\mathbb F_{q^2})$ genau
\begin{equation}
4(q-1)(q-2)(q+1)^3/(q^2-1) = 4(q-2)(q+1)^2 \quad\text{Punkte.}
\end{equation}

Zuletzt zu den Punkten in $S_2(\mathbb F_{q^2}) = \{\text{genau zwei Koord.}= 0\}$. Für die Platzierung der zwei Nullen gibt es zuächst $\binom 4 2 = 6$ Möglichkeiten. Von den zwei verbleibenden Koordinaten legt eine die andere fest, also gibt es $q-1$ Möglichkeiten dafür. Macht also $6(q-1)$, bzw.
\begin{equation}
6(q-1)(q+1)^2/(q^2-1) = 6(q+1)
\end{equation}
Punkte in $S_2(\mathbb F_{q^2})$.

Zusammengenommen liegen also auf $F_d(\mathbb F_{q^2})$ genau
\begin{equation}
((q-2)^2+(q-1))(q+1)^3 + 4(q-2)(q+1)^2 + 6(q+1) = (q+1)(q^2-q+1)(q^2+1)
\end{equation}
Punkte. Im letzten Kapitel sahen wir, dass sich auf der Fläche $3d^2 + (d-3)d^3 = 3(q+1)^2 + (q-2)(q+1)^3 = (q+1)^2(q^2-q+1)$ Geraden befinden. Jede dieser Geraden besteht aus $q^2+1$ Punkten, das sind insgesamt $(q+1)^2(q^2-q+1)(q^2+1)$ Punkte. Also gehen durch jeden Punkt der Fläche im Durchschnitt $q+1$ Punkte.

\section{Philosophische Anmerkung}
Man könnte das Folgende auch als missglückte Beweisidee verstehen. Sie liefert allerdings einen guten Grund, warum die alle Symmetrien der Geradenkonfiguration durch semilineare Symmetrien der Fläche induziert werden. \note Wollen wir das mit in die Arbeit aufnehmen?

Wir haben gerade gesehen, dass durch jeden Punkt von $F_d(\mathbb F_{q^2})$ im Durchschnitt $q+1$ Geraden gehen. Tatsächlich sind es für jeden Punkt genau so viele, denn die semilineare Symmetriegruppe, die wir im nächsten Abschnitt angeben werden, operiert transitiv auf $F_d(\mathbb F_{q^2})$. Damit muss durch jeden Punkt dieselbe Anzahl an Geraden gehen.

Damit korrespondieren Punkte in $F_d(\mathbb F_{q^2})$ zu vollständigen Subgraphen $K_d = K_{q+1}$ der Schnittkonfiguration: andere vollständige Graphen kann es nicht geben, wie man sich leicht überlegt. Damit induziert eine kombinatorische Symmetrie eine Permutation der Punkte auf $F_d(\mathbb F_{q^2})$. Diese erhält außerdem Geraden: mehrere Punkte liegen nämlich genau dann auf einer Geraden, wenn die dazugehörigen $K_d$ einen Knoten gemeinsam haben. Diese Eigenschaft wird durch Graphautomorphismen erhalten.

Wir haben also eine Abbildung $F_d(\mathbb F_{q^2}) \to F_d(\mathbb F_{q^2})$, die Kollinearität erhält. Nach dem \emph{Fundamentalsatz der projektiven Geometrie}\footnote{siehe bspw.~\cite[S.~72]{Dieudonne}.} ist jede Abbildung $\proj 3(\mathbb F_{q^2}) \to \proj 3(\mathbb F_{q^2})$, die Kollinearität erhält, eine semilineare Abbildung. Da $F_d(\mathbb F_{q^2})$ vollständig von Geraden abgedeckt wird, ist zumindest verständlich, warum der Homomorphismus $G_l \to G_k$ surjektiv ist.

\section{Geistergeraden}
\paragraph{Konfiguration}
\paragraph{Symmetrien}
% Betrachtung der irregulären Situation

\printbibliography

\appendix
% Eidessattliche Erklärung, Danksagung etc.

\end{document}
