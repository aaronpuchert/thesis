\chapter{Grundlagen} \label{chap:prelim}
Wir setzen grundlegende Kenntnisse der Algebra und der klassischen algebraischen Geometrie voraus. Einige nicht so bekannte Konzepte sollen aber im Folgenden vorgestellt werden. Wir arbeiten hier über algebraisch abgeschlossenen Körpern $K$, können also die Ergebnisse der klassischen algebraischen Geometrie voraussetzen. Diese kann man beispielsweise in \cite{Shafarevich} oder \cite{Mumford} nachlesen.

\paragraph{Semilineare Abbildungen} Zwischen Vektorräumen $V$, $W$ über einem Körper $K$ gibt es eine natürliche Klasse von Morphismen, die $K$-linearen Abbildungen. Eine Verallgemeinerung erhalten wir durch \emph{semilineare} Abbildungen: sei $\sigma \in \Aut K$, dann heißt $f: V \to W$ semilinear (relativ zu $\sigma$), wenn $f(a+b) = f(a) + f(b)$ und $f(\lambda a) = \sigma(\lambda)f(a)$ für alle $\lambda \in K$ und $a, b \in V$ gilt. Fixiert man in $V$ eine Basis, so lässt sich jede semilineare Abbildung $V \to W$ als Komposition der Anwendung eines Körperautomorphismus auf alle Komponenten und einer anschließenden linearen Abbildung schreiben.\footcite[Das folgt leicht aus den Bemerkungen in][S.~2--3]{Dieudonne} Die Fortsetzungen von $\sigma \in \Aut K$ auf $K^n$ bzw.~$\mathrm{Mat}(n \times n, K) \cong K^{n \times n}$ durch Anwendung auf alle Komponenten wollen wir ebenfalls $\sigma$ nennen.

Analog zur Gruppe der invertierbaren Endomorphismen eines $n$-dimensionalen Vektorraums $\GL nK$ kann man die Gruppe der invertierbaren semilinearen Abbildungen $\GaL nK$ definieren. Nach obiger Bemerkung lässt sich diese Gruppe schreiben als
\begin{equation}
\GaL nK = \GL nK \rtimes \Aut K \quad\text{mit Multiplikation } (\mat A, \sigma)(\mat B, \tau) = (\mat A \sigma(\mat B), \sigma \tau).
\end{equation}
Die Operation auf $V$ ist dann gegeben durch $(\mat A, \sigma)v = \mat A \sigma(v)$. Die Konstruktion überträgt sich auf den projektiven Fall, die Gruppe der semilinear-projektiven Transformationen heißt entsprechend $\PGaL nK$.

\paragraph{Unitäre Gruppen} Auf ähnliche Weise können wir nun Sesquilinearformen $b: V \times V \to K$ relativ zu $\sigma \in \Aut K$ über beliebigen Körpern definieren. Das klassische Beispiel ist $K = \mathbb C$ mit $\sigma$ als der komplexen Konjugation. Man verlangt:
\begin{itemize}
\item $b(x, \cdot)$ ist linear für alle $x \in V$,
\item $b(\cdot, y)$ ist semilinear relativ zu $\sigma$ für alle $y \in V$ und
\item $b(x,y) = \sigma(b(y,x))$ für alle $x,y \in V$.
\end{itemize}
Offenbar folgt die zweite Eigenschaft aus den anderen beiden. Die dritte Eigenschaft liefert $\sigma^2 = \id$ auf dem Bild von $b$, also ist $\sigma$ Involution für $b \neq 0$. Sei $b$ nicht entartet, d.h. von der Form $b(x,y) = \sigma(y^\top) \mat M x$ mit $\mat M \in \GL nK$, dann können wir unitäre Abbildungen bzw.~Matrizen bezüglich $b$ definieren. Für $\mat M = \id$ bilden diese die unitäre Gruppe
\begin{equation}
\Unit nK = \{ \mat A \in \GL nK: \sigma(\mat A^\top) \mat A = \id \}.
\end{equation}
Das ist offenbar die Gruppe der $b$-invarianten linearen Transformationen: sei $\mat A \in \GL nK$ mit $b(x,y) = b(\mat Ax, \mat Ay)$, d.\,h. $\sigma(y^\top) x = \sigma(\mat Ay)^\top (\mat Ax) = \sigma(y^\top) (\sigma(\mat A^\top) \mat A) x$ für alle $x, y \in V$, dann folgt $\id = \sigma(\mat A^\top) \mat A$.

Die entsprechende Sesquilinearform schreiben wir $\langle x, y \rangle = \sigma(y^\top) x$. Wir haben damit eine Form $Q: K^n \to K^\sigma: x \mapsto \langle x, x \rangle$, die wir Norm nennen wollen. Wie im komplexen Fall gibt es auch für $\sigma \neq \id$ eine Polarisierungsformel: $K$ ist dann eine \textsc{Galois}-Erweiterung des Fixpunktkörpers $K^\sigma$ vom Grad $[K:K^\sigma] = 2$. Damit haben wir eine Norm $\norm$ und eine Spur $\tr \colon K \to K^\sigma$. Fixiere nun eine $K^\sigma$-Basis $\{a_1, a_2\}$ von $K$. Dann gilt
\begin{equation*}
Q(a_i v + w) = \norm(a_i) Q(v) + \tr(a_i \langle v, w \rangle) + Q(w)
\end{equation*}
für $i = 1,2$. Stellt man diese nach dem Spurterm um und beachtet, dass die Spurform separabler Erweiterungen nicht entartet ist,\footcite[S.~199, Satz~7]{Bosch} kann man $\langle v, w \rangle$ durch Normen ausdrücken.
\begin{fact} \label{fact:norminv}
Die unitäre Gruppe $\Unit nK$ ist die Gruppe der linearen Abbildungen $K^n \to K^n$, die die Form $Q: K^n \to K^\sigma, (x_1, \dots, x_n) \mapsto \sum_i \sigma(x_i) x_i$ invariant lässt.
\end{fact}
\begin{proof}
Offenbar lässt jede unitäre Abbildung die Norm invariant. Betrachte nun eine lineare Abbildung, die die Norm invariant lässt. Nach der Polarisierungsformel lässt eine solche Abbildung auch das Skalarprodukt $K^n \times K^n \to K$ invariant, mithin ist sie unitär.
\end{proof}

\paragraph{Äußere Produkte} Zur Beschreibung von Untervektorräumen eines Vektorraums werden wir im nächsten Kapitel äußere Produkte benötigen. So wie das Tensorprodukt eine universelle Bilinearform liefert, erhalten wir durch das äußere Produkt eine universelle alternierende Form. Eine multilineare Form heißt alternierend, wenn sie bei zwei übereinstimmenden Argumenten verschwindet.

\begin{defin}
Sei $R$ (kommutativer) Ring und $V$ Modul über $R$. Das $n$-fache äußere Produkt $V^n \to \Lambda^n V$, $(v_1, \dots, v_n) \mapsto v_1 \wedge \dots \wedge v_n$, ist dann eine alternierende Abbildung mit der folgenden universellen Eigenschaft: sei $W$ ein $R$-Modul und $\phi: V^n \to W$ eine alternierende Form, dann gibt es eine eindeutige lineare Abbildung $\pi: \Lambda^n V \to A$ mit der Eigenschaft $\phi(v_1, \dots, v_n) = \pi(v_1 \wedge \dots \wedge v_n)$.
\end{defin}

Ist $R$ Körper und $V$ endlichdimensional, dann kann man die Dimension des äußeren Produkts direkt angeben: hat $V$ die Basis $(e_i)_{1 \leq i \leq n}$, dann hat $\Lambda_k V$ die Basis $(e_{i_1} \wedge \dots \wedge e_{i_k})_{1 \leq i_1 \leq \dots \leq i_k \leq n}$. Folglich ist $\dim \Lambda^k V = \binom nk$, insbesondere gilt $\Lambda^k V = 0$ für $k > n$.

Die direkte Summe $\bigoplus_{n=0}^\infty \Lambda^n V$ heißt dann äußere Algebra $\Lambda(V)$, man erhält sie auch als Quotient der Tensoralgebra $T(V) = \bigoplus_{n=0}^\infty \bigotimes^n V$ nach dem Ideal erzeugt von allen Elementen der Form $v \otimes v$, $v \in V$. Sie wird gradierte Algebra durch die Operationen $\wedge: \Lambda_i V \times \Lambda_j V \to \Lambda_{i+j} V$, die sich auf die Algebra fortsetzen. Für endlich-dimensionales~$V$ ist sie auch endlich-dimensional, es gilt offenbar $\dim \Lambda(V) = 2^{\dim V}$.
