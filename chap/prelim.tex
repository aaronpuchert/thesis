\chapter{Grundlagen} \label{chap:prelim}
Wir arbeiten hier immer über algebraisch abgeschlossenen Körpern $K$, können also die Ergebnisse der klassischen algebraischen Geometrie voraussetzen. Mit $\aff n$ bezeichnen wir den $n$-dimensionalen affinen Raum über $K$, mit $\proj n$ den $n$-dimensionalen projektiven Raum über $K$. Wir werden meistens im projektiven Raum arbeiten.

% Verschiedene Arten von Varietäten: affine, projektive, quasiprojektive. Affine lassen wir weg.
Als projektive Varietät oder abgeschlossene Menge $V = V(I) \in \proj n$ bezeichnen wir die Nullstellenmenge eines homogenen Ideals $I = I(V) \unlhd K[X_0,\dots,X_n]$. Nach Hilberts Nullstellensatz gibt es eine bijektive Korrespondenz zwischen den projektiven Varietäten und radikalen homogenen Idealen in $K[X_0,\dots,X_n]$.

Eine quasiprojektive Varietät ist eine offene Menge in einer abgeschlossenen Menge. Affine Varietäten fassen wir als quasiprojektive auf. Der Koordinatenring einer Varietät ist dann der Quotient $K[V] = K[X_0,\dots,X_n]^H/I(V)$. \note Wie bezeichnet man den Ring der homogenen Polynome? Reguläre Abbildungen $f: V \to \proj m$ sind komponentenweise homogene Polynome vom selben Grad.

% Definitionen einiger Grundbegriffe: Irreduzibilität, Dimension, Tangentialraum, Glattheit

% Dimensionssätze

% Wollen wir GL, PGL, SL und PSL einführen?
% Äußeres Produkt

% Bedeutung der S_n: ab 0
% Körper immer abgeschlossen, Charakteristik p
