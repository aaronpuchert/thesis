\chapter{Symmetrien der Geradenkonfiguration} \label{chap:configsymm}
\section{Lineare und Kombinatorische Symmetrien}
Die Fermat-Flächen haben einen hohen Grad an Symmetrie. Offenbar lassen sowohl Permutationen der Koordinaten als auch Multiplikation der Koordinaten mit $d$-ten Einheitswurzeln die Fläche invariant. Damit operieren die Gruppen $S_4$ und $\mu_d^4$, letztere mit Stabilisator \note Heißt das so? isomorph zu $\mu_d$. Beide Gruppen schneiden sich nur in $\{\id\}$, also haben wir eine Aktion von
\begin{equation}
S_4 \ltimes \mu_d^4 / \mu_d \subset \PGL 4K,
\end{equation}
wobei der Homomorphismus $S_4 \rightarrow \Aut(\mu_d^4 / \mu_d)$ auf Permutationen der Koordinaten abbildet. In bestimmten Fällen gibt es sogar noch mehr Symmetrien, wie wir später sehen werden.

Wir interessieren uns allgemein für Abbildungen aus $\PGaL 4K$, die die Fläche $F_d$ invariant lassen.\footnote{Für eine Theorie der linearen Gruppen, speziell der $\PGaL nK$, siehe \cite{Dieudonne}.} Solche linearen Abbildungen lassen nicht nur die Fläche invariant, sondern schicken auch Geraden auf Geraden. Damit permutieren sie die Geraden auf der Fläche, offenbar bleibt dabei aber ihre Schnittkonfiguration erhalten.

Die Frage liegt nahe, welche Permutationen der Geraden es denn gibt, die die Schnittkonfiguration erhalten. Wir wollen in diesem Kapitel die Beziehung zwischen diesen beiden Gruppen ausarbeiten. Die Betrachtung ist inspiriert durch \cite[S.~180]{Mumford}.

\section{Reguläre Geraden}
\paragraph{Konfiguration}
\paragraph{Symmetrien}
% Betrachtung der regulären Situation

\section{Geistergeraden}
\paragraph{Konfiguration}
\paragraph{Symmetrien}
% Betrachtung der irregulären Situation