\chapter{\textsc{Fermat}-Flächen} \label{chap:fermat}
Wir untersuchen nun einen wichtigen Spezialfall der allgemeinen Theorie, nämlich Geraden auf Fermat-Flächen. Im Jahre 1637 vermutete \textsc{Pierre de Fermat}, dass die Gleichung
\begin{equation*}
X^d + Y^d = Z^d
\end{equation*}
für $d \geq 3$ keine nichttrivialen ganzzahligen Lösungen hat. Dazu äquivalent ist, dass die Varietäten
\begin{equation*}
F_d^{(2)} = \{ X^d + Y^d = Z^d \} \subset \proj 2(\mathbb C)
\end{equation*}
keine rationalen Punkte haben. Diese nennen sich \emph{Fermat-Kurven}. Wir bezeichnen eine analoge Familie von Flächen im $\proj 3$ als \emph{Fermat-Flächen}:
\begin{equation}
F_d = F_d^{(3)} = \{ X_0^d + X_1^d + X_2^d + X_3^d = 0 \} \subset \proj 3(K),
\end{equation}
wobei $\Char K \nmid d$. \note Warum ist dieser Fall nicht interessant? Man beachte, dass wir dann primitive $d$-te Einheitswurzeln in $K$ haben.
\todo Falls wir es brauchen: Die Fläche ist irreduzibel und sogar regulär.

Wir wollen nun die Rechnungen aus dem Beweis von Fakt~\ref{fact:gammaproj} konkreter machen. Sei $W \in \grass 2/4$ ein Unterraum mit Basis $\{a = (a_0, \dots, a_3), b = (b_0, \dots, b_3)\}$. Dann sind die Plücker-Koordinaten $p_{ij} = a_i b_j - a_j b_i$, wobei $p_{ij} + p_{ji} = 0$. Wir rekonstruieren nun den Unterraum aus den Plücker-Koordinaten:
\begin{equation}
W = \{ \phi(a)b - \phi(b)a \colon \phi \in (K^4)^* \}.
\end{equation}
Ein $\phi \in (K^4)^*$ hat die Form $\phi(x) = \sum_{i=0}^3 \alpha_i \langle x, e_i \rangle$ mit geeigneten Koeffizienten $\alpha_i \in K$. Damit
\begin{align*}
\phi(x)y - \phi(y)x &= \sum_i \alpha_i \langle a, e_i \rangle b - \sum_i \alpha_i \langle b, e_i \rangle a \\
	&= \sum_i \alpha_i a_i \sum_j b_j e_j - \sum_i \alpha_i b_i \sum_j a_j e_j \\
	&= \sum_j \sum_i \left(\alpha_i a_i b_j - \alpha_i b_i a_j \right) e_j \\
\intertext{Dabei laufen die Summen jeweils über $\{0,\dots,3\}$. Für $i=j$ verschwinden die Summanden jeweils, also erhalten wir}
\phi(x)y - \phi(y)x &= \sum_j \left(\sum_{i \neq j} \alpha_i p_{ij} \right) e_j
\end{align*}

Das setzen wir nun in die Gleichung der Fermat-Fläche $F_d$ ein:
\begin{align*}
0 = \sum_{j=0}^3 X_j^d &= \sum_{j=0}^3 \left(\sum_{i \neq j} \alpha_i p_{ij} \right)^d \\
\text{(Multinomialtheorem)}\qquad &= \sum_{j=0}^3 \sum_{\substack{(d_0,\dots,d_3) \\ \sum d_i=d,\;d_j=0}} \binom d{d_0,\dots,d_3} \prod_{i=0}^3 \alpha_i^{d_i} p_{ij}^{d_i} \\
	&= \sum_{\substack{(d_0,\dots,d_3) \\ \sum d_i=d}} \binom d{d_0,\dots,d_3} \left(\sum_{\substack{j=0 \\ d_j=0}}^3 \prod_{i=0}^3 p_{ij}^{d_i} \right) \prod_{i=0}^3 \alpha_i^{d_i}
\end{align*}
Da die Gleichung in den $\alpha_i$ identisch gelten soll, können wir einen Koeffizientenvergleich machen. \note Nach \textsc{Hilbert}s Nullstellensatz? Ein Vergleich der Koeffizienten zu $\prod_{i=0}^3 \alpha_i^{d_i}$ für ein $(d_0,\dots,d_3)$ ergibt
\begin{equation}
\binom d{d_0,\dots,d_3} \sum_{\substack{j=0 \\ d_j=0}}^3 \prod_{i=0}^3 p_{ij}^{d_i} = 0.
\end{equation}
Das liefert uns den folgenden Fakt.

\begin{fact}
Eine projektive Gerade mit Plücker-Koordinaten $(p_{ij})$ liegt genau dann auf der Fermat-Fläche vom Grad $d$, wenn für alle $(d_0,\dots,d_3)$ mit $d_0 + \dots + d_3 = d$ und $\binom d{d_0,\dots,d_3} \neq 0$ die folgende Gleichung gilt:
\begin{equation}
\sum_{\substack{j=0 \\ d_j=0}}^3 \prod_{i=0}^3 p_{ij}^{d_i} = 0.
\end{equation}
\end{fact}

\noindent Unabhängig von $d$ ist $\binom d{d,0,0,0} = 1 \neq 0$ (und analog für Permutationen), damit haben wir
\begin{equation} \label{eq:powers}
\sum_{j\in\{1,2,3\}} p_{0j}^d = 0
\end{equation}
und analog für Permutationen. Gilt zudem $p \nmid (d-1)$, so ist $\binom d{d-2,1,1,0} = d(d-1) \neq 0$ für $d \geq 3$. Damit haben wir
\begin{equation} \label{eq:products}
p_{03}^{d-2} p_{13} p_{23} = 0
\end{equation}
und analog für Permutationen. Das bedeutet: für jedes $i$ verschwindet eines der $p_{ij}$.

\begin{fact}
Sei $d \geq 3$ und die Charakteristik kein Teiler von $d(d-1)$. Dann liegen auf $F_d$ die drei Familien von Geraden
\begin{align*}
\text{(I)}\qquad	X_0 + \zeta X_1 &= 0, \qquad &X_2 + \eta X_3 &= 0; \\
\text{(II)}\qquad	X_0 + \zeta X_2 &= 0, \qquad &X_1 + \eta X_3 &= 0; \\
\text{(III)}\qquad	X_0 + \zeta X_3 &= 0, \qquad &X_1 + \eta X_2 &= 0;
\end{align*}
mit $\zeta, \eta \in \mu_d$, wobei $\mu_d$ die Menge der $d$-ten Einheitswurzeln ist.
\end{fact}
\begin{proof}
Aus Gleichung \ref{eq:products} folgt zunächst, dass für jedes $i$ eines der $p_{ij}$ verschwindet. Damit verschwindet jeweils einer der Summanden in Gleichung \ref{eq:powers} und diese bekommen die Form $X^d + Y^d = 0$. Das ist äquivalent zu $(X/Y)^d = -1$ bzw. $X/Y = \zeta^{2k+1}$, wobei $\zeta$ primitive $2d$-te Einheitswurzel ist.

Für $i,j,k \in \{0,\dots,3\}$ verschieden verschwindet also entweder eines von $p_{ij}$ und $p_{ik}$ oder ihr Verhältnis ist eine $2d$-te Einheitswurzel. Daraus ergeben sich genau obige Geraden. \note Sollte man die Kombinatorik hier genauer ausführen?
\end{proof}

Im generischen Fall liegen also $3d^2$ Geraden auf eine Fermat-Fläche vom Grad $d$. Nun untersuchen wir noch den Fall $p = d-1$. Ist $d-1$ Primzahl, dann ist das der einzige singuläre Fall.

\begin{prop}
Sei $d-1$ Primzahl und $d \geq 3$, dann liegen in Charakteristik $p=d-1$ auf $F_d$ zusätzlich die Geraden

\todo Hier fehlen noch die Gleichungen.
\end{prop}
\begin{proof}
Hier entfallen fast alle Gleichungen: $\binom d{d_0,d_1,d_2,d_3}$ ist nur dann nicht durch $d-1$ teilbar, wenn $d_i=d$ oder $d_i=d-1$ für ein $i$ gilt. Der erste Fall führt auf die Gleichungen \ref{eq:powers}, der zweite liefert Gleichungen der Form
\begin{equation}
p_{02}^{d-1} p_{12} + p_{03}^{d-1} p_{13} = 0 \qquad\Leftrightarrow\qquad \frac{p_{12}}{p_{13}} = -\left(\frac{p_{03}}{p_{02}}\right)^{d-1}
\end{equation}
und Permutationen. Iterieren liefert
\begin{equation*}
\frac{p_{12}}{p_{13}} = -\left(\frac{p_{03}}{p_{02}}\right)^{d-1} = \left(\frac{p_{12}}{p_{13}}\right)^{(d-1)^2},
\end{equation*}
also sind die Verhältnisse der Plücker-Koordinaten $k$-te Einheitswurzeln mit $k=(d-1)^2-1=d(d-2)$. \note Was ist, wenn Nenner verschwinden?

\todo Geraden ausrechnen.
\end{proof}
