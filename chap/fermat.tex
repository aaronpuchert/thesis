\chapter{\textsc{Fermat}-Flächen} \label{chap:fermat}
Wir untersuchen nun einen wichtigen Spezialfall der allgemeinen Theorie, nämlich Geraden auf Fermat-Flächen. Im Jahre 1637 vermutete \textsc{Pierre de Fermat}, dass die Gleichung
\begin{equation*}
X^d + Y^d = Z^d
\end{equation*}
für $d \geq 3$ keine nichttrivialen ganzzahligen Lösungen hat. Dazu äquivalent ist, dass die Varietäten
\begin{equation*}
F_d^{(2)} = \{ X^d + Y^d = Z^d \} \subset \proj 2(\mathbb C)
\end{equation*}
keine rationalen Punkte haben. Diese nennen sich \emph{Fermat-Kurven}. Wir bezeichnen eine analoge Familie von Flächen im $\proj 3$ als \emph{Fermat-Flächen}:
\begin{equation}
F_d = F_d^{(3)} = \{ X_0^d + X_1^d + X_2^d + X_3^d = 0 \} \subset \proj 3(K),
\end{equation}
wobei $\Char K \nmid d$. (Im Fall $\Char K \mid d$ ist $X_0^d + X_1^d + X_2^d + X_3^d = (X_0 + X_1 + X_2 + X_3)^d$, also wäre $F_d$ eine Ebene.) Man beachte, dass wir dann primitive $d$-te Einheitswurzeln in $K$ haben.
\todo Falls wir es brauchen: Die Fläche ist irreduzibel und sogar regulär.

Wir wollen nun die Rechnungen aus dem Beweis von Fakt~\ref{fact:gammaproj} konkreter machen. Sei $W \in \grass 2/4$ ein Unterraum mit Basis $\{a = (a_0, \dots, a_3), b = (b_0, \dots, b_3)\}$. Dann sind die Plücker-Koordinaten $p_{ij} = a_i b_j - a_j b_i$, wobei $p_{ij} + p_{ji} = 0$. Wir rekonstruieren nun den Unterraum aus den Plücker-Koordinaten:
\begin{equation}
W = \{ \phi(a)b - \phi(b)a \colon \phi \in (K^4)^* \}.
\end{equation}
Ein $\phi \in (K^4)^*$ hat die Form $\phi(x) = \sum_{i=0}^3 \alpha_i \langle x, e_i \rangle$ mit geeigneten Koeffizienten $\alpha_i \in K$. Damit
\begin{align*}
\phi(x)y - \phi(y)x &= \sum_i \alpha_i \langle a, e_i \rangle b - \sum_i \alpha_i \langle b, e_i \rangle a \\
	&= \sum_i \alpha_i a_i \sum_j b_j e_j - \sum_i \alpha_i b_i \sum_j a_j e_j \\
	&= \sum_j \sum_i \left(\alpha_i a_i b_j - \alpha_i b_i a_j \right) e_j \\
\intertext{Dabei laufen die Summen jeweils über $\{0,\dots,3\}$. Für $i=j$ verschwinden die Summanden jeweils, also erhalten wir}
\phi(x)y - \phi(y)x &= \sum_j \left(\sum_{i \neq j} \alpha_i p_{ij} \right) e_j
\end{align*}

Das setzen wir nun in die Gleichung der Fermat-Fläche $F_d$ ein:
\begin{align*}
0 = \sum_{j=0}^3 X_j^d &= \sum_{j=0}^3 \left(\sum_{i \neq j} \alpha_i p_{ij} \right)^d \\
\text{(Multinomialtheorem)}\qquad &= \sum_{j=0}^3 \sum_{\substack{(d_0,\dots,d_3) \\ \sum d_i=d,\;d_j=0}} \binom d{d_0,\dots,d_3} \prod_{i=0}^3 \alpha_i^{d_i} p_{ij}^{d_i} \\
	&= \sum_{\substack{(d_0,\dots,d_3) \\ \sum d_i=d}} \binom d{d_0,\dots,d_3} \left(\sum_{\substack{j=0 \\ d_j=0}}^3 \prod_{i=0}^3 p_{ij}^{d_i} \right) \prod_{i=0}^3 \alpha_i^{d_i}
\end{align*}
Da die Gleichung in den $\alpha_i$ identisch gelten soll, können wir einen Koeffizientenvergleich machen. \note Nach \textsc{Hilbert}s Nullstellensatz? Ein Vergleich der Koeffizienten zu $\prod_{i=0}^3 \alpha_i^{d_i}$ für ein $(d_0,\dots,d_3)$ ergibt
\begin{equation}
\binom d{d_0,\dots,d_3} \sum_{\substack{j=0 \\ d_j=0}}^3 \prod_{i=0}^3 p_{ij}^{d_i} = 0.
\end{equation}
Das liefert uns den folgenden Fakt.

\begin{fact}
Eine projektive Gerade mit Plücker-Koordinaten $(p_{ij})$ liegt genau dann auf der Fermat-Fläche vom Grad $d$, wenn für alle $(d_0,\dots,d_3)$ mit $d_0 + \dots + d_3 = d$ und $\binom d{d_0,\dots,d_3} \neq 0$ die folgende Gleichung gilt:
\begin{equation}
\sum_{\substack{j=0 \\ d_j=0}}^3 \prod_{i=0}^3 p_{ij}^{d_i} = 0.
\end{equation}
\end{fact}

\noindent Seien Indizes $i,j,k,l$ gewählt mit $\{i,j,k,l\} = \{0,1,2,3\}$. Unabhängig von $d$ ist dann $\binom d{d,0,0,0} = 1 \neq 0$ (und analog für Permutationen), damit haben wir
\begin{equation} \label{eq:powers}
p_{ij}^d + p_{ik}^d + p_{il}^d = 0.
\end{equation}
Weiterhin ist $\binom d{d-1,1,0,0} = d \neq 0$, daher erhalten wir Gleichungen der Form
\begin{equation} \label{eq:ratios}
p_{jk}^{d-1} p_{ik} + p_{jl}^{d-1} p_{il} = 0 \qquad\overrel\Longleftrightarrow^{p_{il}, p_{jk} \neq 0}\qquad \frac{p_{ik}}{p_{il}} = -\left(\frac{p_{jl}}{p_{jk}}\right)^{d-1}.
\end{equation}

Inwiefern weitere Gleichungen erfüllt sein müssen, beantworten die folgenden Propositionen.
\begin{prop}
Sei $p \in \mathbb P$ und $d$ kein Vielfaches von $p$. Gilt weiterhin, dass $d-1$ keine $p$-Potenz ist, dann gibt es $d_0, d_1, d_2 > 0$ mit $p \nmid \binom d{d_0,d_1,d_2,0}$.
\end{prop}
\begin{proof}
Sei $d = k \cdot p^n + 1$ mit $n>1$ und $p \nmid k$. Weiterhin sei $m>0$ so gewählt, dass $p^m < k < p^{m+1}$. Betrachte
\begin{equation*}
\binom {kp^n+1}{(k-p^m)p^n,p^{n+m},1,0} = \frac{((k-p^m)p^n+1) \dots (kp^n-1)kp^n(kp^n+1)}{1 \dots (p^{n+m})}
\end{equation*}
Nenner und Zähler sind jeweils Produkte $p^{n+m}$ konsekutiver natürlicher Zahlen, von denen keine durch $p^{n+m+1}$ teilbar ist. Der Nenner ist daher genauso oft durch $p$ teilbar wie der Zähler. Also ist der Bruch nicht durch $p$ teilbar.
\end{proof}

\begin{fact}
Sei $K$ Körper der Charakteristik~$p \in \mathbb P \cup \{0\}$, $d \geq 3$ mit $p \nmid d$, und $d-1$ keine Potenz von~$p$. Dann liegen auf $F_d(K)$ genau die drei Familien von Geraden
\begin{equation} \label{eq:regular}
\begin{split}
\text{(I)}\qquad	&\langle (1,\zeta,0,0), (0,0,1,\eta)\rangle \\
\text{(II)}\qquad	&\langle (1,0,\zeta,0), (0,1,0,\eta)\rangle \\
\text{(III)}\qquad	&\langle (1,0,0,\zeta), (0,1,\eta,0)\rangle
\end{split} \qquad \zeta, \eta \in \mu_{2d} \setminus \mu_d,
\end{equation}
wobei $\mu_d \subset K$ die Menge der $d$-ten Einheitswurzeln ist.
\end{fact}
\begin{proof}
Nach voriger Proposition ist $\binom d{d_0,d_1,d_2,0} \neq 0$ für geeignete $d_0, d_1, d_2 > 0$. Damit haben wir
\begin{equation} \label{eq:products}
p_{03}^{d_0} p_{13}^{d_1} p_{23}^{d_2} = 0
\end{equation}
und Varianten. Das bedeutet: für jedes $i$ verschwindet mindestens eines der $p_{ij}$. Wir machen uns zunächst klar, dass nicht mehr verschwinden können: sei o.\,E. $p_{01} = p_{02} = 0$, dann ist wegen \eqref{eq:powers} auch $p_{03} = 0$. Mit \eqref{eq:ratios} folgt daraus, dass $p_{13}^{d-1}p_{23} = p_{12}^{d-1}p_{23} = p_{12}^{d-1}p_{13} = 0$, also ist verschwinden mindestens zwei von der drei Koordinaten $p_{12}$, $p_{13}$, $p_{23}$. Dass nur eine Koordinate nicht verschwindet, geht aber wegen \eqref{eq:powers} nicht.

Also verschwindet jeweils einer der Summanden in Gleichung \eqref{eq:powers} und diese bekommen die Form $X^d + Y^d = 0$. Das ist äquivalent zu $(X/Y)^d = -1$ bzw. $X/Y \in \mu_{2d} \setminus \mu_d$. Schreiben wir nun die Koeffizienten in einer Tabelle auf:

{\vskip 2ex\hfil
\begin{tabular}{|c|c|c|c|} \hline
0 & $p_{01}$ & $p_{02}$ & $p_{03}$ \\ \hline
$-p_{01}$ & 0 & $p_{12}$ & $p_{13}$ \\ \hline
$-p_{02}$ & $-p_{12}$ & 0 & $p_{23}$ \\ \hline
$-p_{03}$ & $-p_{13}$ & $-p_{23}$ & 0 \\ \hline
\end{tabular}
\hfil\vskip 2ex}

In jeder Spalte und Zeile steht zusätzlich eine Null, insgesamt hat die Tabelle also acht Nulleinträge. Von den vier Nulleinträgen, die nicht auf der Diagonale liegen, sind jeweils zwei oberhalb und zwei unterhalb, also verschwinden zwei der sechs $p_{ij}$, seien dies $p_{ij}$ und $p_{kl}$, dabei gilt $\{i,j,k,l\} = \{0,1,2,3\}$. Wir können also ohne Einschränkung annehmen, dass $p_{01}$ und $p_{23}$ verschwinden.

Setzen wir nun $p_{13} = 1$, dann ergibt sich $p_{03} = \lambda$ und $p_{12} = \eta$ mit $\lambda, \eta \in \mu_{2d}$, $\lambda^d = \eta^d = -1$. Mit \eqref{eq:grcond} folgt $p_{02} = \lambda\eta$. Die entsprechende projektive Gerade wird durch $(\lambda, 1, 0, 0)$ und $(0,0,\mu,1)$ aufgespannt. Man überzeugt sich leicht, dass diese tatsächlich auf der Fermat-Fläche liegt.
\end{proof}

Im generischen Fall liegen also $3d^2$ Geraden auf eine Fermat-Fläche vom Grad $d$. Nun untersuchen wir noch den Fall $p = d-1$. Ist $d-1$ Primzahl, dann ist das der einzige singuläre Fall.

\begin{prop}
Sei $d-1$ Primzahl und $d \geq 3$, dann liegen in Charakteristik $p=d-1$ auf $F_d$ zusätzlich die Geraden

\todo Hier fehlen noch die Gleichungen.
\end{prop}
\begin{proof}
Hier entfallen fast alle Gleichungen: $\binom d{d_0,d_1,d_2,d_3}$ ist nur dann nicht durch $d-1$ teilbar, wenn $d_i=d$ oder $d_i=d-1$ für ein $i$ gilt. Der erste Fall führt auf die Gleichungen \ref{eq:powers}, der zweite liefert Gleichungen der Form
\begin{equation}
p_{02}^{d-1} p_{12} + p_{03}^{d-1} p_{13} = 0 \qquad\Leftrightarrow\qquad \frac{p_{12}}{p_{13}} = -\left(\frac{p_{03}}{p_{02}}\right)^{d-1}
\end{equation}
und Permutationen. Iterieren liefert
\begin{equation*}
\frac{p_{12}}{p_{13}} = -\left(\frac{p_{03}}{p_{02}}\right)^{d-1} = \left(\frac{p_{12}}{p_{13}}\right)^{(d-1)^2},
\end{equation*}
also sind die Verhältnisse der Plücker-Koordinaten $k$-te Einheitswurzeln mit $k=(d-1)^2-1=d(d-2)$. \note Was ist, wenn Nenner verschwinden?

\todo Geraden ausrechnen.
\end{proof}
