\chapter{\textsc{Fermat}-Flächen} \label{chap:fermat}
Wir untersuchen nun einen wichtigen Spezialfall der allgemeinen Theorie, nämlich Geraden auf Fermat-Flächen. Im Jahre 1637 vermutete \textsc{Pierre de Fermat}, dass die Gleichung
\begin{equation*}
X^d + Y^d = Z^d
\end{equation*}
für $d \geq 3$ keine nichttrivialen ganzzahligen Lösungen hat. Dazu äquivalent ist, dass die Varietäten
\begin{equation*}
F_d^{(2)} = \{ X^d + Y^d = Z^d \} \subset \proj 2(\mathbb C)
\end{equation*}
keine rationalen Punkte haben. Diese nennen sich \emph{Fermat-Kurven}. Wir bezeichnen eine analoge Familie von Flächen im $\proj 3$ als \emph{Fermat-Flächen}:
\begin{equation}
F_d = F_d^{(3)} = \{ X_0^d + X_1^d + X_2^d + X_3^d = 0 \} \subset \proj 3(K),
\end{equation}
wobei $\Char K \nmid d$. (Im Fall $\Char K \mid d$ ist $X_0^d + X_1^d + X_2^d + X_3^d = (X_0 + X_1 + X_2 + X_3)^d$, also wäre $F_d$ eine Ebene.) Man beachte, dass wir dann primitive $d$-te Einheitswurzeln in $K$ haben.
\todo Falls wir es brauchen: Die Fläche ist irreduzibel und sogar regulär.

Wir wollen nun die Rechnungen aus dem Beweis von Fakt~\ref{fact:gammaproj} konkreter machen. Sei $W \in \grass 2/4$ ein Unterraum mit Basis $\{a = (a_0, \dots, a_3), b = (b_0, \dots, b_3)\}$. Dann sind die Plücker-Koordinaten $p_{ij} = a_i b_j - a_j b_i$, wobei $p_{ij} + p_{ji} = 0$. Wir rekonstruieren nun den Unterraum aus den Plücker-Koordinaten:
\begin{equation}
W = \{ \phi(a)b - \phi(b)a \colon \phi \in (K^4)^* \}.
\end{equation}
Ein $\phi \in (K^4)^*$ hat die Form $\phi(x) = \sum_{i=0}^3 \alpha_i \langle x, e_i \rangle$ mit geeigneten Koeffizienten $\alpha_i \in K$. Damit
\begin{align*}
\phi(x)y - \phi(y)x &= \sum_i \alpha_i \langle a, e_i \rangle b - \sum_i \alpha_i \langle b, e_i \rangle a \\
	&= \sum_i \alpha_i a_i \sum_j b_j e_j - \sum_i \alpha_i b_i \sum_j a_j e_j \\
	&= \sum_j \sum_i \left(\alpha_i a_i b_j - \alpha_i b_i a_j \right) e_j \\
\intertext{Dabei laufen die Summen jeweils über $\{0,\dots,3\}$. Für $i=j$ verschwinden die Summanden jeweils, also erhalten wir}
\phi(x)y - \phi(y)x &= \sum_j \left(\sum_{i \neq j} \alpha_i p_{ij} \right) e_j
\end{align*}

Das setzen wir nun in die Gleichung der Fermat-Fläche $F_d$ ein:
\begin{align*}
0 = \sum_{j=0}^3 X_j^d &= \sum_{j=0}^3 \left(\sum_{i \neq j} \alpha_i p_{ij} \right)^d \\
\text{(Multinomialtheorem)}\qquad &= \sum_{j=0}^3 \sum_{\substack{(d_0,\dots,d_3) \\ \sum d_i=d,\;d_j=0}} \binom d{d_0,\dots,d_3} \prod_{i=0}^3 \alpha_i^{d_i} p_{ij}^{d_i} \\
	&= \sum_{\substack{(d_0,\dots,d_3) \\ \sum d_i=d}} \binom d{d_0,\dots,d_3} \left(\sum_{\substack{j=0 \\ d_j=0}}^3 \prod_{i=0}^3 p_{ij}^{d_i} \right) \prod_{i=0}^3 \alpha_i^{d_i}
\end{align*}
Da die Gleichung in den $\alpha_i$ identisch gelten soll, können wir einen Koeffizientenvergleich machen. \note Nach \textsc{Hilbert}s Nullstellensatz? Ein Vergleich der Koeffizienten zu $\prod_{i=0}^3 \alpha_i^{d_i}$ für ein $(d_0,\dots,d_3)$ ergibt
\begin{equation}
\binom d{d_0,\dots,d_3} \sum_{\substack{j=0 \\ d_j=0}}^3 \prod_{i=0}^3 p_{ij}^{d_i} = 0.
\end{equation}
Das liefert uns den folgenden Fakt.

\begin{fact}
Eine projektive Gerade mit Plücker-Koordinaten $(p_{ij})$ liegt genau dann auf der Fermat-Fläche vom Grad $d$, wenn für alle $(d_0,\dots,d_3)$ mit $d_0 + \dots + d_3 = d$ und $\binom d{d_0,\dots,d_3} \neq 0$ die folgende Gleichung gilt:
\begin{equation}
\sum_{\substack{j=0 \\ d_j=0}}^3 \prod_{i=0}^3 p_{ij}^{d_i} = 0.
\end{equation}
\end{fact}

\noindent Seien Indizes $i,j,k,l$ gewählt mit $\{i,j,k,l\} = \{0,1,2,3\}$. Unabhängig von $d$ ist dann $\binom d{d,0,0,0} = 1 \neq 0$ (und analog für Permutationen), damit haben wir
\begin{equation} \label{eq:powers}
p_{ij}^d + p_{ik}^d + p_{il}^d = 0.
\end{equation}
Weiterhin ist $\binom d{d-1,1,0,0} = d \neq 0$, daher erhalten wir Gleichungen der Form
\begin{equation} \label{eq:ratios}
p_{jk}^{d-1} p_{ik} + p_{jl}^{d-1} p_{il} = 0 \qquad\overrel\Longleftrightarrow^{p_{il}, p_{jk} \neq 0}\qquad \frac{p_{ik}}{p_{il}} = -\left(\frac{p_{jl}}{p_{jk}}\right)^{d-1}.
\end{equation}

Inwiefern weitere Gleichungen erfüllt sein müssen, beantworten die folgenden Propositionen.
\begin{prop}
Sei $p \in \mathbb P$ und $d$ kein Vielfaches von $p$. Gilt weiterhin, dass $d-1$ keine $p$-Potenz ist, dann gibt es $d_0, d_1, d_2 > 0$ mit $p \nmid \binom d{d_0,d_1,d_2,0}$.
\end{prop}
\begin{proof}
Sei $d = k \cdot p^n + 1$ mit $n>1$ und $p \nmid k$. Weiterhin sei $m>0$ so gewählt, dass $p^m < k < p^{m+1}$. Betrachte
\begin{equation*}
\binom {kp^n+1}{(k-p^m)p^n,p^{n+m},1,0} = \frac{((k-p^m)p^n+1) \dots (kp^n-1)kp^n(kp^n+1)}{1 \dots (p^{n+m})}
\end{equation*}
Nenner und Zähler sind jeweils Produkte $p^{n+m}$ konsekutiver natürlicher Zahlen, von denen keine durch $p^{n+m+1}$ teilbar ist. Der Nenner ist daher genauso oft durch $p$ teilbar wie der Zähler. Also ist der Bruch nicht durch $p$ teilbar.
\end{proof}

\begin{fact}
Sei $K$ Körper der Charakteristik~$p \in \mathbb P \cup \{0\}$, $d \geq 3$ mit $p \nmid d$, und $d-1$ keine Potenz von~$p$. Dann liegen auf $F_d(K)$ genau die drei Familien von Geraden
\begin{equation} \label{eq:regular}
\begin{split}
\text{(I)}\qquad	&\langle (1,\zeta,0,0), (0,0,1,\eta)\rangle \\
\text{(II)}\qquad	&\langle (1,0,\zeta,0), (0,1,0,\eta)\rangle \\
\text{(III)}\qquad	&\langle (1,0,0,\zeta), (0,1,\eta,0)\rangle
\end{split} \qquad \zeta, \eta \in \mu_{2d} \setminus \mu_d,
\end{equation}
wobei $\mu_d \subset K$ die Menge der $d$-ten Einheitswurzeln ist.
\end{fact}
\begin{proof}
Nach voriger Proposition ist $\binom d{d_0,d_1,d_2,0} \neq 0$ für geeignete $d_0, d_1, d_2 > 0$. Damit haben wir
\begin{equation} \label{eq:products}
p_{03}^{d_0} p_{13}^{d_1} p_{23}^{d_2} = 0
\end{equation}
und Varianten. Das bedeutet: für jedes $i$ verschwindet mindestens eines der $p_{ij}$. Wir machen uns zunächst klar, dass nicht mehr verschwinden können: sei o.\,E. $p_{01} = p_{02} = 0$, dann ist wegen \eqref{eq:powers} auch $p_{03} = 0$. Mit \eqref{eq:ratios} folgt daraus, dass $p_{13}^{d-1}p_{23} = p_{12}^{d-1}p_{23} = p_{12}^{d-1}p_{13} = 0$, also ist verschwinden mindestens zwei von der drei Koordinaten $p_{12}$, $p_{13}$, $p_{23}$. Dass nur eine Koordinate nicht verschwindet, geht aber wegen \eqref{eq:powers} nicht.

Also verschwindet jeweils einer der Summanden in Gleichung \eqref{eq:powers} und diese bekommen die Form $X^d + Y^d = 0$. Das ist äquivalent zu $(X/Y)^d = -1$ bzw. $X/Y \in \mu_{2d} \setminus \mu_d$. Schreiben wir nun die Koeffizienten in einer Tabelle auf:

{\vskip 2ex\hfil
\begin{tabular}{|c|c|c|c|} \hline
0 & $p_{01}$ & $p_{02}$ & $p_{03}$ \\ \hline
$-p_{01}$ & 0 & $p_{12}$ & $p_{13}$ \\ \hline
$-p_{02}$ & $-p_{12}$ & 0 & $p_{23}$ \\ \hline
$-p_{03}$ & $-p_{13}$ & $-p_{23}$ & 0 \\ \hline
\end{tabular}
\hfil\vskip 2ex}

In jeder Spalte und Zeile steht zusätzlich eine Null, insgesamt hat die Tabelle also acht Nulleinträge. Von den vier Nulleinträgen, die nicht auf der Diagonale liegen, sind jeweils zwei oberhalb und zwei unterhalb, also verschwinden zwei der sechs $p_{ij}$, seien dies $p_{ij}$ und $p_{kl}$, dabei gilt $\{i,j,k,l\} = \{0,1,2,3\}$. Wir können also ohne Einschränkung annehmen, dass $p_{01}$ und $p_{23}$ verschwinden.

Setzen wir nun $p_{13} = 1$, dann ergibt sich $p_{03} = \lambda$ und $p_{12} = \eta$ mit $\lambda, \eta \in \mu_{2d}$, $\lambda^d = \eta^d = -1$. Mit \eqref{eq:grcond} folgt $p_{02} = \lambda\eta$. Die entsprechende projektive Gerade wird durch $(\lambda, 1, 0, 0)$ und $(0,0,\mu,1)$ aufgespannt. Man überzeugt sich leicht, dass diese tatsächlich auf der Fermat-Fläche liegt.
\end{proof}

Im generischen Fall liegen also $3d^2$ Geraden auf eine Fermat-Fläche vom Grad $d$. Im nächsten Kapitel werden wir ihre Konfiguration untersuchen. Betrachten wir aber zunächst noch die Spezialfälle.

\begin{prop}
Sei $p \in \mathbb P$ und $d = p^n+1$ mit $n \in \mathbb N$. Dann sind alle Multinomialkoeffizienten $\binom d{d_0,d_1,d_2,0}$ mit $d_0, d_1, d_2, d_3 \not\in \{d-1, d\}$ durch $p$ teilbar.
\end{prop}
\begin{proof}
(Das geht analog zu obiger Proposition.)
\end{proof}

\begin{lemma}
Sei $K$ ein Körper der Charakteristik $p$. Die Gleichung $X+Y+Z=0$ mit $X,Y,Z \in \mu_{p^n-1}$ hat dann $(p^n-1)(p^n-2)$ Lösungen.
\end{lemma}
\begin{proof}
Offenbar gilt $\mu_{p^n-1} = \mathbb F_{p^n}^* \subset K$. Es sind also alle Lösungen von $X+Y+Z=0$ in $\mathbb F_{p^n} \setminus \{0\}$ zu finden. Das Folgende ist nun Kombinatorik: für $X$ können wir $p^n-1$ verschiedene Werte wählen. Haben wir $Y$ gewählt, ergibt sich $Z$ als $Z=-X-Y$. Damit $Z \neq 0$ ist, muss $X+Y \neq 0$ sein, also dürfen wir nicht $Y = -X$ wählen. Folglich gibt es für $Y$ genau $p^n-2$ mögliche Wahlen.
\end{proof}

\begin{theorem}[Geistergeraden]
Für alle Charakteristiken $p \neq 2$ und Grade $d = p^n + 1$ mit $n \in \mathbb N$ liegen auf $F_d$ die Familien von Geraden (I)--(III). Ist weiterhin $p=3$, dann liegen zusätzlich die Geraden \note Wie sieht es allgemein aus?
\begin{equation} \label{eq:ghost}
\text{(IV)}\qquad \langle (1:0:\zeta:\eta),\; (0,\theta,\zeta,-\eta)\rangle, \qquad \theta, \zeta, \eta \in \mu_d,
\end{equation}
darauf. Diese sind verschieden, und weitere Geraden gibt es nicht.
\end{theorem}
\begin{remarks}
Die Geraden \eqref{eq:regular} nennen wir \emph{reguläre} Geraden, die in \eqref{eq:ghost} \emph{Geistergeraden}. Die Anzahl der regulären Geraden ist $3d^2$, die der Geistergeraden $(d-2)(d-3)d^3$. \note Wirklich? Für große $d$ gibt es also deutlich mehr davon.
\end{remarks}
\begin{proof}
Nach obiger Proposition haben wir neben \eqref{eq:grcond} genau die Gleichungen für $\binom{d}{d,0,0,0}=1$ und $\binom{d}{d-1,1,0,0}$ und Varianten. Der erste Fall führt auf die Gleichungen \eqref{eq:powers}, der zweite auf \eqref{eq:ratios}. Letztere in sich selbst eingesetzt liefern
\begin{equation*}
\frac{p_{12}}{p_{13}} = -\left(\frac{p_{03}}{p_{02}}\right)^{d-1} = -\left(-\left(\frac{p_{12}}{p_{13}}\right)^{d-1}\right)^{d-1} = \left(\frac{p_{12}}{p_{13}}\right)^{(d-1)^2}, \qquad\text{da $p$ ungerade,}
\end{equation*}
falls $p_{12}, p_{13} \neq 0$ und $p_{03} p_{02} \neq 0$. Verschwindet also keine der Plücker-Koordinaten, so sind ihre Verhältnisse $k$-te Einheitswurzeln mit $k=(d-1)^2-1=d(d-2)$. Da die Plücker-Koordinaten homogene Koordinaten sind, können wir annehmen, dass $p_{ij} \in \mu_{d(d-2)}$ für alle $i \neq j$.

Betrachten wir nun \eqref{eq:ratios} zur $d$-ten Potenz erhoben:
\begin{equation*}
\frac{p_{ik}^d}{p_{il}^d} = \left(\frac{p_{ik}}{p_{il}}\right)^d = \left(\frac{p_{jl}}{p_{jk}}\right)^{d(d-1)} = \left(\frac{p_{jl}}{p_{jk}}\right)^d = \frac{p_{jl}^d}{p_{jk}^d}.
\end{equation*}
Mit der Substitution $i \leftrightarrow k$, $j \leftrightarrow l$ erhalten wir
\begin{equation*}
\frac{p_{ik}^d}{p_{jk}^d} = \frac{p_{ki}^d}{p_{kj}^d} = \frac{p_{lj}^d}{p_{li}^d} = \frac{p_{jl}^d}{p_{il}^d}.
\end{equation*}
Durcheinander geteilt ergibt das
\begin{equation*}
\frac{p_{jk}^d}{p_{il}^d} = \frac{p_{il}^d}{p_{jk}^d} \qquad\text{bzw.}\qquad \frac{p_{il}^d}{p_{jk}^d} = \pm 1.
\end{equation*}

Setze $\mu = p_{01}^d$, $\eta = p_{02}^d$, $\nu = p_{03}^d$. Die Plückerkoordinaten in $d$-ter Potenz verhalten sich dann wie folgt:
{\vskip 2ex\hfil
\begin{tabular}{|c|c|c|c|} \hline
0 & $\mu$ & $\eta$ & $\nu$ \\ \hline
$\mu$ & 0 & $\pm \nu$ & $\pm \eta$ \\ \hline
$\eta$ & $\pm \nu$ & 0 & $\pm \mu$ \\ \hline
$\nu$ & $\pm \eta$ & $\pm \mu$ & 0 \\ \hline
\end{tabular}
\hfil\vskip 2ex}
Wegen \eqref{eq:powers} müssen die Summen über alle Zeilen und Spalten gleich sein. Eine leichte Überlegung ergibt dann, dass für alle $\pm$ nur $+$ infrage kommt. Steht an nur einer Stelle ein $-$, sei also beispielsweise $p_{12}=-\nu$, aber $p_{13}=\eta$. Dann ergibt eine Subtraktion der Gleichungen \eqref{eq:powers} für die ersten beiden Zeilen $\nu = -\nu$, also $\nu = 0$. Das ist ein Widerspruch. Steht an mindestens zwei Stellen ein $-$, sei also o.\,E. $p_{12}=-\nu$ und $p_{13}=-\eta$. Dann ergibt eine Addition der ersten beiden Zeilen, dass $2\mu = 0$, also $\mu = 0$. Auch das geht nicht.

Die Gleichung ist daher genau dann erfüllt, wenn $\mu+\eta+\nu = 0$ mit $\mu, \eta, \nu \in \mu_{d-2} = \mathbb F_{p^n}^*$. Ist $\zeta \in \mu_{d(d-2)}$ primitiv, so können wir $\mu = \zeta^{ad}$, $\eta = \zeta^{bd}$, $\nu = \zeta^{cd}$ schreiben. Nach dem vorigen Lemma gibt es genau $(p^n-1)(p^n-2)$ solche Tripel $(a,b,c) \in (\Zmod (d-2)Z)^3$. Wir schreiben $a_{01} = a_{23} = a, a_{02} = a_{13} = b$, $a_{03} = a_{12} = c$.

Damit haben die $p_{ij}$ die Form $\zeta^{a_{ij} + b_{ij}(d-2)}$ sind mit $b_{ij} \in \Zmod dZ$. Die Gleichungen \eqref{eq:ratios} werden dann zu
\begin{align*}
\frac{\zeta^{a_{ik} + b_{ik}(d-2)}}{\zeta^{a_{il} + b_{il}(d-2)}} = \frac{p_{ik}}{p_{il}} &= -\left(\frac{p_{jl}}{p_{jk}}\right)^{d-1} = -\left(\frac{\zeta^{a_{jl} + b_{jl}(d-2)}}{\zeta^{a_{jk} + b_{jk}(d-2)}}\right)^{d-1} \\
a_{ik} - a_{il} + (b_{ik} - b_{il})(d-2) &\equiv (a_{jl} - a_{jk} + (b_{jl} - b_{jk})(d-2))(d-1) + d(d-2)/2 &&\mod{d(d-2)} \\
(b_{ik} - b_{il} + b_{jl} - b_{jk})(d-2) &\equiv (a_{ik} - a_{il})(d-2) + d(d-2)/2 &&\mod{d(d-2)} \\
b_{ik} - b_{il} - b_{jk} + b_{jl} &\equiv a_{ik} - a_{il} + d/2 &&\mod d
\end{align*}
Man beachte dabei, dass $a_{ik} = a_{jl}$, $a_{il} = a_{jk}$. Weiterhin gilt $b_{ij} - b_{ji} \equiv d/2 \pmod d$ wegen $p_{ij} + p_{ji} = 0$. Wir müssen die Gleichung nicht für alle Permutationen testen, sondern wegen Symmetrie nur für die drei Quartupel $(i,j,k,l) = (0,1,2,3), (0,2,1,3), (0,3,1,2)$.

Das ergibt ein inhomogenes Gleichungssystem in den $b_{ij} \in \Zmod dZ$, wir lösen sie aber zunächst im größeren Ring $\frac 12 \Zmod dZ$. Dort können wir leicht eine Lösung angeben: für $i<j$ setze $b_{ij} = a_{ij}/2$ für $2 \mid i-j$ und $b_{ij} = a_{ij}/2 + d/4$ sonst. Es ist zu prüfen, ob für die drei Quartupel die Gleichung erfüllt ist:
\begin{align*}
&(0,1,2,3): &(a_{02}/2) - (a_{03}/2+\tfrac d4) - (a_{12}/2+\tfrac d4) + (a_{13}/2) &\overrel{\equiv}^! a_{02} - a_{03} + d/2 \\
&(0,2,1,3): &(a_{01}/2+\tfrac d4) - (a_{03}/2+\tfrac d4) - (a_{12}/2-\tfrac d4) + (a_{23}/2+\tfrac d4) &\overrel{\equiv}^! a_{01} - a_{03} + d/2 \\
&(0,3,1,2): &(a_{01}/2+\tfrac d4) - (a_{02}/2) - (a_{13}/2+\tfrac d2) + (a_{23}/2-\tfrac d4) &\overrel{\equiv}^! a_{01} - a_{02} + d/2
\end{align*}

Nun zur Lösung des homogenen Gleichungssystems. Wir setzen ohne Einschränkung $b_{01} + b_{23} = 0$, multipliziere gegebenenfalls die Plückerkoordinaten mit $\zeta^{-(b_{01} + b_{23})(d-2)/2}$. (Das geht wegen $2 \mid d-2$.) Das ergibt folgendes Gleichungssystem:
\begin{equation*}
\begin{pmatrix}
1 & 0 & -1 & -1 & 0 & 1 \\
0 & 1 & -1 & -1 & 1 & 0 \\
1 & -1 & 0 & 0 & -1 & 1 \\
1 & 0 & 0 & 0 & 0 & 1
\end{pmatrix}
\begin{pmatrix}
b_{01} \\ b_{02} \\ b_{03} \\ b_{12} \\ b_{13} \\ b_{23}
\end{pmatrix}
= \underline{0}
\end{equation*}
Das ist äquivalent zu $b_{01} + b_{23} = 0$, $b_{02} + b_{13} = 0$ und $b_{03} + b_{12} = 0$. Die allgemeine Lösung mit Parametern $\alpha, \beta, \gamma \in \Zmod 2dZ$ ist daher:
{\vskip 2ex\hfil
\begin{tabular}{|c|c|c|c|} \hline
0 & $(a+\alpha)/2+d/4$ & $(b+\beta)/2$ & $(c+\gamma)/2+d/4$ \\ \hline
$(a+\alpha)/2-d/4$ & 0 & $(c-\gamma)/2+d/4$ & $(b-\beta)/2$ \\ \hline
$(b+\beta)/2+d/2$ & $(c-\gamma)/2-d/4$ & 0 & $(a-\alpha)/2+d/4$ \\ \hline
$(c+\gamma)/2-d/4$ & $(b-\beta)/2+d/2$ & $(a-\alpha)/2-d/4$ & 0 \\ \hline
\end{tabular}
\hfil\vskip 2ex}
Um die Lösungen der Gleichung in $\Zmod dZ$ zu bekommen, schränken wir sie einfach ein. Das bedeutet $\alpha \equiv a+d/2,\; \beta \equiv b,\; \gamma \equiv c+d/2 \pmod 2$. Die endgültigen Exponenten $a_{ij} + (d-2)b_{ij}$ sind damit:
{\vskip 2ex\hfil
\begin{tabular}{|c|c|c|c|} \hline
0 & $\frac{ad}2+\alpha\frac{d-2}2+\frac{d(d-2)}4$ & $\frac{bd}2+\beta\frac{d-2}2$ & $\frac{cd}2+\gamma\frac{d-2}2+\frac{d(d-2)}4$ \\ \hline
$\frac{ad}2+\alpha\frac{d-2}2-\frac{d(d-2)}4$ & 0 & $\frac{cd}2-\gamma\frac{d-2}2+\frac{d(d-2)}4$ & $\frac{bd}2-\beta\frac{d-2}2$ \\ \hline
$\frac{bd}2+\beta\frac{d-2}2+\frac{d(d-2)}2$ & $\frac{cd}2-\gamma\frac{d-2}2-\frac{d(d-2)}4$ & 0 & $\frac{ad}2-\alpha\frac{d-2}2+\frac{d(d-2)}4$ \\ \hline
$\frac{cd}2+\gamma\frac{d-2}2-\frac{d(d-2)}4$ & $\frac{bd}2-\beta\frac{d-2}2+\frac{d(d-2)}2$ & $\frac{ad}2-\alpha\frac{d-2}2-\frac{d(d-2)}4$ & 0 \\ \hline
\end{tabular}
\hfil\vskip 2ex}

Nun überzeugen wir uns noch, dass die Bestimmungsgleichung der Grassmannschen \eqref{eq:grcond} erfüllt ist:
\begin{align*}
0 &\overrel{=}^! \zeta^{\frac{ad}2+\alpha\frac{d-2}2+\frac{d(d-2)}4} \zeta^{\frac{ad}2-\alpha\frac{d-2}2+\frac{d(d-2)}4} - \zeta^{\frac{bd}2+\beta\frac{d-2}2} \zeta^{\frac{bd}2-\beta\frac{d-2}2} \\
  &\qquad + \zeta^{\frac{cd}2+\gamma\frac{d-2}2+\frac{d(d-2)}4} \zeta^{\frac{cd}2-\gamma\frac{d-2}2+\frac{d(d-2)}4} \\
  &= \zeta^{ad + \frac{d(d-2)}2} - \zeta^{bd} + \zeta^{cd + \frac{d(d-2)}2} \\
  &= -(\zeta^{ad} + \zeta^{bd} + \zeta^{cd})
\end{align*}
Der Term in Klammern verschwindet, also ist die Gleichung automatisch erfüllt. \note Wollen wir noch die Geraden ausrechnen?

% -----------------

Nun zu dem Fall, dass Einträge verschwinden. Sei also $p_{ij}=0$, dann ist wegen \eqref{eq:ratios}
\begin{align*}
p_{ij}^{d-1}p_{kj} + p_{il}^{d-1}p_{kl} = 0 \qquad\Rightarrow p_{il} = 0 \text{ oder } p_{kl} = 0, \\
p_{ij}^{d-1}p_{lj} + p_{ik}^{d-1}p_{lk} = 0 \qquad\Rightarrow p_{ik} = 0 \text{ oder } p_{kl} = 0, \\
p_{ji}^{d-1}p_{ki} + p_{jl}^{d-1}p_{kl} = 0 \qquad\Rightarrow p_{jl} = 0 \text{ oder } p_{kl} = 0, \\
p_{ji}^{d-1}p_{li} + p_{jk}^{d-1}p_{lk} = 0 \qquad\Rightarrow p_{jk} = 0 \text{ oder } p_{kl} = 0.
\end{align*}
Also gilt $p_{kl} = 0$ oder $p_{il} = p_{ik} = p_{jl} = p_{jk} = 0$. Im zweiten Fall verschwinden dann alle Plücker-Koordinaten wegen \eqref{eq:powers}, der erste führt auf die regulären Geraden aus dem vorigen Fakt.
\end{proof}
