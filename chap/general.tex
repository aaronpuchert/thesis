\chapter{Allgemeine Untersuchungen} \label{chap:general}
\section{Die \textsc{Graßmann}-Mannigfaltigkeit} \label{sec:grassmann}
Die \textsc{Graßmann}-Mannigfaltigkeit $\grass r/n$ ist die Mannigfaltigkeit der $r$\-/dimensionalen Untervektorräume eines $n$\-/dimensionalen Vektorraums über einem Körper $K$. Ihre Struktur ergibt sich wie folgt: die Gruppe der allgemeinen linearen Transformationen $\GL nK$ operiert auf den $r$-dimensionalen Unterräumen von $K^n$ transitiv. Der Stabilisator jedes Unterraums ist isomorph zu $\GL rK \times K^{r(n-r)} \times \GL{n-r}K$, damit ergibt sich
\begin{equation}
\grass r/n = \GL nK / (\GL rK \times K^{r(n-r)} \times \GL{n-r}K)
\end{equation}
und wegen $\dim \GL nK = n^2$ ist $\dim \grass r/n = r(n-r)$: das folgt aus dem Satz über die Faserdimension für die kanonische Projektion $\GL nK \to \grass r/n$.

Man kann die Graßmann-Varietät als projektive Varietät im $r$\-/fachen äußeren Produkt $\mathbb P(\bigwedge^r K^n)$ auffassen (siehe \cite[S.~42]{Shafarevich}): wir haben eine Einbettung $\grass r/n \to \bigwedge^r K^n$, die die Basis eines Unterraums $v_1, \dots, v_r$ auf $v_1 \wedge
\dots \wedge v_r$ schickt. Man überlegt sich leicht, dass das Bild eines Unterraums unabhängig von der Wahl der Basis ist: ist $w_1, \dots, w_r$ eine andere Basis mit $w_j = \sum_i a_{ij} v_i$ mit einer geeigneten Matrix $\mat A = (a_{ij})$, dann ist
\begin{align*}
\bigwedge_j w_j &= \bigwedge_j \sum_i a_{ij} v_i = \sum_{\sigma: \{1,\dots,r\} \rightarrow \{1,\dots,r\}} \bigwedge_j a_{\sigma(j)j} v_{\sigma(j)} \\
	&= \sum_{\sigma: \{1,\dots,r\} \rightarrow \{1,\dots,r\}} \prod_j a_{\sigma(j)j} \wunderbrace{\bigwedge_j v_{\sigma(j)}}_{$= 0$, falls $\sigma$ nicht injektiv} = \sum_{\sigma \in S_r} \prod_j a_{\sigma(j)j} \bigwedge_j v_{\sigma(j)} \\
	&= \sum_{\sigma \in S_r} \sign \sigma \prod_j a_{\sigma(j)j} \bigwedge_j v_j = \det A \bigwedge_j v_j
\end{align*}
Die Abbildung ist auch injektiv, wie man sich leicht überlegt. \note Wie?
Sei $\{e_1, \dots, e_n\}$ die Standardbasis des $K^n$, dann ist $\{e_{i_1} \wedge \dots \wedge e_{i_r} \colon i_1 < \dots < i_r\}$ eine Basis von $\bigwedge^r K^n$. Die Koordinaten $(p_{i_1 \dots i_r})$ eines Punktes nennt man dann \textsc{Plücker}\-/Koordinaten.

Die Abbildung ist aber nicht surjektiv, nicht alle Elemente des äußeren Produkts entstehen durch eine Basis eines $r$\-/dimensionalen Teilraums. Das Bild ist vielmehr eine Untervarietät gegeben durch die Gleichungen
\begin{equation}
\sum_{t=1}^{r+1} (-1)^t p_{i_1 \dots i_{r-1} j_t} p_{j_1 \dots \hat{j_t} \dots j_{r+1}}, \quad\text{für } i_1 < \dots < i_{r-1}, j_1 < \dots < j_{r+1}.
\end{equation}
(siehe \cite[S.~42]{Shafarevich} \note Beweis?) Weiterhin ist $\grass r/n$ eine irreduzible projektive Varietät. Dazu zeigen wir zunächst ein Lemma.

\begin{prop}
Für jedes $n \in \mathbb N$ und jeden Körper $K$ ist die quasiprojektive Varietät $\GL nK$ glatt und irreduzibel.
\end{prop}
\begin{proof}
Glattheit ist offensichtlich, da $\GL nK$ algebraische Gruppe ist. Irreduzibilität zeigen wir mit Induktion über $n$. Für $n=1$ ist $\GL 1K = K^*$.

Für $n>1$ beobachten wir folgendes: Die Gruppe $\GL nK$ operiert transitiv auf $K^n$, sei $\vec{e_1} = (1,0,\dots,0)$ der erste Einheitsvektor. Der Stabilisator eines Punktes aus $K^n \setminus 0$ ist isomorph zum Stabilisator von $\vec{e_1}$, der ist
\begin{equation}
\Stab e_1 = \left\{ \begin{pmatrix}
1 & \vec v \\
\vec 0 & \mat A
\end{pmatrix} \colon \vec v \in K^{n-1}, \mat A \in \GL{n-1}K \right\} \cong K^{n-1} \times \GL{n-1}K.
\end{equation}
Dieser ist nach Induktionsvoraussetzung irreduzibel als Produkt irreduzibler Varietäten. Nun wenden wir ein Lemma aus \cite[S.~77]{Shafarevich} an. \note Auch beweisen?

Die reguläre Abbildung $\GL nK \to K^n \colon \mat A \mapsto \mat A \vec{e_1}$ ist surjektiv, das Bild ist irreduzibel und alle Fasern sind irreduzibel und haben dieselbe Dimension. Folglich ist auch $\GL nK$ irreduzibel.
\end{proof}

\begin{prop}
Die Graßmann-Mannigfaltigkeit ist glatt und irreduzibel.
\end{prop}
\begin{proof}
Wie oben erwähnt, operiert $\GL nK$ transitiv auf $\grass r/n$. Wäre $\grass r/n$ reduzibel, würde daher auch $\GL nK$ in mehrere Komponenten zerfallen---als Urbilder unter $\GL nK \to \grass r/n$.

Glattheit folgt ähnlich: da eine Gruppe transitiv operiert, muss $\grass r/n$ überall glatt sein.
\end{proof}

\section{Affine Unterräume auf projektiven Hyperflächen} \label{sec:linesproj}
Wir wollen nun untersuchen, wann auf Hyperflächen projektive Unterräume liegen. Dazu verallgemeinern wir die Vorgehensweise aus \cite[S.~78ff]{Shafarevich}. Als Hyperfläche vom Grad $d$ bezeichnen wir die Nullstellenmenge eines homogenen Polynoms vom Grad $d$. Alle Komponenten einer solchen Hyperfläche haben Codimension 1. \note Referenz?

Sei im Folgenden $n$ die Dimension des Raums, $r$ die Dimension der gesuchten Unterräume. Die homogenen Formen vom Grad $d$ in $n+1$ Variablen bilden einen projektiven Raum der Dimension $\nu_{n+1}^d = \binom{n+d}{n} - 1$: sind $X_0, \dots, X_n$ die Variablen, so bilden $X_0^{d_0} \dots X_n^{d_n}$ mit $d_0 + \dots + d_n = d$, $d_i \geq 0$ eine Basis. Die projektiven Hyperflächen vom Grad $d$ in einem projektiven Raum der Dimension $n$ bilden also eine Varietät isomorph zu $\proj{\nu_{n+1}^d}$.

Projektive Unterräume der Dimension $r$ in $\proj n$ korrespondieren zu Untervektorräumen der Dimension $r+1$ in $\aff{n+1}$. Also bilden sie eine Varietät isomorph zu $\grass r+1/{n+1}$.

Wir definieren nun eine Teilmenge $\Gamma_{r,n}^d$ im Produkt $\proj{\nu_{n+1}^d} \times \grass r+1/{n+1}$
\begin{equation}
\Gamma_{r,n}^d = \{(F,L) \in \mathbb P^{\nu_{n+1}^d} \times \grass r+1/{n+1} \colon \text{$L$ liegt auf der durch $F$ definierten Varietät} \}
\end{equation}

\begin{fact}
Die Menge $\Gamma_{r,n}^d$ ist eine projektive Varietät.
\end{fact}
\begin{proof}
Es sind also zunächst zur verbalen Beschreibung äquivalente algebraische Gleichungen zu finden. Betrachten wir dazu einen $(r+1)$-dimensionalen Unterraum~$W$ in~$\aff{n+1}$ mit Basis $(v_i)_{0 \leq i \leq r}$, die einzelnen Komponenten eines Basisvektors~$v_i$ mögen $v_{ij}$, $0 \leq j \leq n$, heißen. Dann gilt für die Plücker-Koordinaten des Unterraums:
\begin{equation}
p_{i_0 \dots i_r} = \sum_{\sigma \in S_{r+1}} \sign \sigma \cdot v_{0i_{\sigma(0)}} \dots v_{ri_{\sigma(r)}}
\end{equation}
wobei $S_{r+1}$ die Permutationsgruppe von $\{0,\dots,r\}$ ist. Wie lässt sich nun der Unterraum aus den Plücker-Koordinaten rekonstruieren? Zunächst macht man sich klar, dass sich $W$ schreiben lässt als
\begin{equation}
W = \left\{ \sum\limits_{\sigma \in S_{r+1}} \sign \sigma \cdot \phi(v_{\sigma(1)}, \dots, v_{\sigma(r)}) v_{\sigma(0)} \;\middle|\; \phi \colon (K^{n+1})^r \to K \text{ multilinear} \right\}.
\end{equation}
Eine solche Multilinearform $\phi \colon (K^{n+1})^r \to K$ wiederum hat die Form
\begin{equation}
\phi(x_1, x_2, \dots, x_r) = \sum_{\iota: \{1, \dots, r\} \to \{0, \dots, n\}} \alpha_{\iota(1) \dots \iota(r)} \langle x_1, e_{\iota(1)} \rangle \dots \langle x_r, e_{\iota(r)} \rangle
\end{equation}
mit Koeffizienten $\alpha_{i_1 \dots i_r} \in K$. Sei nun $w \in W$, Einsetzen liefert:
\begin{align*}
w &= \sum\limits_{\sigma \in S_{r+1}} \sign \sigma \sum_{\iota: \{1, \dots, r\} \to \{0, \dots, n\}} \alpha_{\iota(1) \dots \iota(r)} \langle v_{\sigma(1)}, e_{\iota(1)} \rangle \dots \langle v_{\sigma(r)}, e_{\iota(r)} \rangle v_{\sigma(0)} \\
	&= \sum\limits_{\sigma \in S_{r+1}} \sign \sigma \sum_{\iota: \{1, \dots, r\} \to \{0, \dots, n\}} \alpha_{\iota(1) \dots \iota(r)} v_{\sigma(1)\iota(1)} \dots v_{\sigma(r)\iota(r)} \sum_{j=0}^n v_{\sigma(0)j} e_j \\
	&= \sum_{j=0}^n \left(\sum\limits_{\sigma \in S_{r+1}} \sign \sigma \sum_{\iota: \{1, \dots, r\} \to \{0, \dots, n\}} \alpha_{\iota(1) \dots \iota(r)} v_{\sigma(1)\iota(1)} \dots v_{\sigma(r)\iota(r)} v_{\sigma(0)j}\right) e_j \\
	&= \sum_{j=0}^n \left(\sum\limits_{\sigma \in S_{r+1}} \sign \sigma \sum_{\substack{\iota: \{0, \dots, r\} \to \{0, \dots, n\} \\ \iota(0)=j }} \alpha_{\iota(1) \dots \iota(r)} v_{\sigma(0)\iota(0)} \dots v_{\sigma(r)\iota(r)}\right) e_j \\
	&= \sum_{j=0}^n \left(\sum_{\substack{\iota: \{0, \dots, r\} \to \{0, \dots, n\} \\ \iota(0)=j }} \alpha_{\iota(1) \dots \iota(r)} \sum\limits_{\sigma \in S_{r+1}} \sign \sigma \cdot v_{0\iota(\sigma^-1(0))} \dots v_{r\iota(\sigma^-1(r))}\right) e_j \\
\Rightarrow\quad w_j	&= \sum_{\substack{\iota: \{0, \dots, r\} \to \{0, \dots, n\} \\ \iota(0)=j }} \alpha_{\iota(1) \dots \iota(r)} \sum\limits_{\sigma \in S_{r+1}} \sign \sigma \cdot v_{0(\iota\circ\sigma)(0)} \dots v_{r(\iota\circ\sigma)(r)}
\end{align*}
Die innere Summe verschwindet für nicht injektive $\iota$: gilt $\iota(x) = \iota(y)$ für $x \neq y$, so ersetze man $\sigma$ durch $\sigma \circ \tau$ mit einer Transposition $\tau: x \leftrightarrow y$. Die Summanden für $\sigma$ und $\sigma \circ \tau$ heben sich dann genau auf.

Für die übrigen (also injektiven) $\iota$ ist diese Summe genau die Plücker-Koordinate $\sign \tau \cdot p_{(\iota\circ\tau)(0)\dots(\iota\circ\tau)(r)}$, wobei $\tau \in S_{r+1}$ die Permutation ist, die $\iota \circ \tau$ streng monoton macht.

Ist nun~$F$ eine homogene Form über $X_0, \dots, X_n$, so setzen wir $X_j = w_j$ und erhalten eine algebraische Gleichung in den~$\alpha_{i_1 \dots i_r}$ und den~$p_{i_0 \dots i_r}$. Der entsprechende Unterraum zu den Plücker-Koordinaten liegt genau dann auf der durch die Form beschriebene Fläche, wenn die Form in den~$\alpha_{i_1 \dots i_r}$ identisch erfüllt ist. Demnach erhalten wir die gesuchten Gleichungen durch Koeffizientenvergleich.
\end{proof}

Damit können wir nun untersuchen, auf welchen projektiven Varietäten eines vorgegebenen Grades Unterräume einer gewissen Dimension liegen. Insbesondere zeigt sich, dass es in bestimmten Fällen solche Unterräume immer gibt.

Wir haben Projektionen $\phi \colon \Gamma_{r,n}^d \to \proj{\nu_{n+1}^d}$ und $\psi: \Gamma_{r,n}^d \to \grass r+1/{n+1}$. Beide sind natürlich regulär.

\begin{prop}
Die Abbildung $\psi$ ist surjektiv mit Fasern isomorph zu $\proj{\nu_{n+1}^d - \nu_{n-r}^d - 1}$.
\end{prop}
\begin{proof}
Betrachte den Unterraum $L_0 = \{X_0 = \dots = X_r = 0\}$: dessen Urbild ist isomorph zu dem der anderen Unterräume, da auf $\grass r+1/{n+1}$ sowie $\proj{\nu_{n+1}^d}$ die Gruppe $\GL{n+1}K$ transitiv operiert.

Die Faser besteht aus allen Formen
\begin{equation*}
F = X_0 F_0 + \dots + X_r F_r
\end{equation*}
mit Formen $F_0, \dots, F_r$ vom Grad $d-1$. \note Stimmt das allgemein? Diese bilden einen Unterraum in $\proj{\nu_{n+1}^d}$ gegeben durch $a_{0 \dots 0 i_{r+1} \dots i_n} = 0$, wobei $a_{i_0 \dots i_n}$ der Koeffizient von $X_0^{i_0} \dots X_n^{i_n}$ ist, $i_0 + \dots + i_n = d$. Damit folgt
\begin{equation*}
\dim \psi^{-1}(L_0) = \nu_{n+1}^d - (\nu_{n-r}^d + 1).
\end{equation*}
Insbesondere ist ein Unterraum irreduzibel.
\end{proof}

\begin{coroll}
Die Varietät $\Gamma_{r,n}^d$ ist irreduzibel und hat Dimension $(r+1)(n-r) + \nu_{n+1}^d - \nu_{n-r}^d - 1$.
\end{coroll}
\begin{proof}
Das folgt mit dem Satz über irreduzible Fasern, den wir oben schon verwendet haben: die Projektion auf $\grass r+1/{n+1}$ hat Fasern isomorph zu einem projektiven Unterraum von $\proj{\nu_{n+1}^d}$, sind also irreduzibel.

Die Dimension ist $\dim \grass r+1/{n+1} + \dim \psi^{-1}(L) = (r+1)(n-r) + \nu_{n+1}^d - \nu_{n-r}^d - 1$ nach dem Satz über die Faserdimension.
\end{proof}

Betrachten wir nun die Projektion $\phi \colon \Gamma_{r,n}^d \to \proj{\nu_{n+1}^d}$: Surjektivität bedeutet, dass auf jeder Fläche vom Grad $d$ in $\proj n$ ein Unterraum der Dimension $r$ liegt. Dafür notwendig ist
\begin{equation}
\dim \Gamma_{r,n}^d = (r+1)(n-r) + \nu_{n+1}^d - \nu_{n-r}^d - 1 \geq \nu_{n+1}^d \qquad \Leftrightarrow \qquad (r+1)(n-r) \geq \nu_{n-r}^d + 1
\end{equation}
Andernfalls ist der Menge der Flächen, auf denen solche Unterräume liegen, eine echte abgeschlossene Teilmenge. \note Wann genau ist die Bedingung hinreichend?

In den folgenden Kapiteln wollen wir nur den Fall $n=3$, $r=1$ betrachten, also Geraden auf projektiven Flächen. Dort ergibt sich
\begin{equation}
(r+1)(n-r) \geq \nu_{n-r}^d + 1 \qquad \Leftrightarrow \qquad 4 \geq \binom{d+1}{1} \qquad \Leftrightarrow \qquad d \leq 3.
\end{equation}

\todo Konkretere Betrachtungen.
