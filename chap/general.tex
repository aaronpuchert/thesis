\chapter{Allgemeine Untersuchungen} \label{chap:general}
\section{Die \textsc{Graßmann}-Mannigfaltigkeit} \label{sec:grassmann}
Die \textsc{Graßmann}-Mannigfaltigkeit $\grass r/n$ ist die Mannigfaltigkeit der $r$\-/dimensionalen Untervektorräume eines $n$\-/dimensionalen Vektorraums über einem Körper $K$. Ihre Struktur ergibt sich wie folgt: die Gruppe der allgemeinen linearen Transformationen $\GL nK$ operiert auf den $r$-dimensionalen Unterräumen von $K^n$ transitiv. Der Stabilisator jedes Unterraums ist isomorph zu $\GL rK \times K^{r(n-r)} \times \GL{n-r}K$, damit ergibt sich
\begin{equation}
\grass r/n = \GL nK / (\GL rK \times K^{r(n-r)} \times \GL{n-r}K)
\end{equation}
und wegen $\dim \GL nK = n^2$ ist $\dim \grass r/n = r(n-r)$: das folgt aus dem Satz über die Faserdimension für die kanonische Projektion $\GL nK \to \grass r/n$.

Man kann die Graßmann-Varietät als projektive Varietät im $r$\-/fachen äußeren Produkt $\mathbb P(\bigwedge^r K^n)$ auffassen (siehe \cite[S.~42]{Shafarevich}): wir haben eine Einbettung $\grass r/n \to \bigwedge^r K^n$, die die Basis eines Unterraums $v_1, \dots, v_r$ auf $v_1 \wedge
\dots \wedge v_r$ schickt. Man überlegt sich leicht, dass das Bild eines Unterraums unabhängig von der Wahl der Basis ist: ist $w_1, \dots, w_r$ eine andere Basis mit $w_j = \sum_i a_{ij} v_i$ mit einer geeigneten Matrix $\mat A = (a_{ij})$, dann ist
\begin{align*}
\bigwedge_j w_j &= \bigwedge_j \sum_i a_{ij} v_i = \sum_{\sigma: \{1,\dots,r\} \rightarrow \{1,\dots,r\}} \bigwedge_j a_{\sigma(j)j} v_{\sigma(j)} \\
	&= \sum_{\sigma: \{1,\dots,r\} \rightarrow \{1,\dots,r\}} \prod_j a_{\sigma(j)j} \wunderbrace{\bigwedge_j v_{\sigma(j)}}_{$= 0$,falls $\sigma$ nicht injektiv} = \sum_{\sigma \in S_r} \prod_j a_{\sigma(j)j} \bigwedge_j v_{\sigma(j)} \\
	&= \sum_{\sigma \in S_r} \sign \sigma \prod_j a_{\sigma(j)j} \bigwedge_j v_j = \det A \bigwedge_j v_j
\end{align*}
Die Abbildung ist auch injektiv, wie man sich leicht überlegt. \note Wie?
Sei $\{e_1, \dots, e_n\}$ die Standardbasis des $K^n$, dann ist $\{e_{i_1} \wedge \dots \wedge e_{i_r} \colon i_1 < \dots < i_r\}$ eine Basis von $\bigwedge^r K^n$. Die Koordinaten $(p_{i_1 \dots i_r})$ eines Punktes nennt man dann \textsc{Plücker}\-/Koordinaten.

Die Abbildung ist aber nicht surjektiv, nicht alle Elemente des äußeren Produkts entstehen durch eine Basis eines $r$\-/dimensionalen Teilraums. Das Bild ist vielmehr eine Untervarietät gegeben durch die Gleichungen
\begin{equation}
\sum_{t=1}^{r+1} (-1)^t p_{i_1 \dots i_{r-1} j_t} p_{j_1 \dots \hat{j_t} \dots j_{r+1}}, \quad\text{für } i_1 < \dots < i_{r-1}, j_1 < \dots < j_{r+1}.
\end{equation}
(siehe \cite[S.~42]{Shafarevich} \note Beweis?) Damit ist $\grass r/n$ eine irreduzible projektive Varietät, denn obige Gleichungen lassen sich nicht faktorisieren. \note Kann man im Falle mehrerer Gleichungen auch so argumentieren?
