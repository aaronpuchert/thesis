\chapter{Einführung} \label{chap:intro}
Kühlturme von Kraftwerken haben oft die Gestalt eines einschaligen Hyperboloids, denn ein solches lässt sich disjunkt in Geraden zerlegen und damit durch eine Struktur aus geraden Stützen stabilisieren. Über den komplexen Zahlen im projektiven Raum zerfällt sogar jede Quadrik auf diese Weise: ist sie entartet, dann auf offensichtliche Weise, sonst kann man sie auf die Form $XY + ZW = 0$ bringen. Diese erhält man z.\,B. aus der Standardform $X_0^2+X_1^2+X_2^2+X_3^2=0$ durch die Substitution $X = X_0 + X_1i$, $Y = X_0 - X_1i$, $Z = X_2 + X_3i$, $W = X_2 - X_3i$. Dann sieht man leicht die Geraden definiert durch die Gleichungen $\lambda X + \mu Z = 0$, $\mu Y - \lambda W = 0$ für $(\lambda:\mu) \in \proj 1(\mathbb C)$.

Bereits Mitte des neunzehnten Jahrhunderts, also noch in der Frühphase der algebraischen Geometrie, erhielten \textsc{Cayley}\footcite{Cayley} und \textsc{Salmon}\footcite{Salmon} ein vergleichbares Resultat für Flächen dritten Grades: auf jeder regulären Kubik im $\proj 3(\mathbb C)$ liegen genau 27 Geraden. Deren Konfiguration ist in den folgenden Jahrzehnten bis ins letzte Detail ausgeleuchtet worden, siehe etwa die Monographie von \textsc{Henderson}.\footcite{Henderson}

Auf Flächen höheren Grades findet man im Allgemeinen keine Geraden mehr. Genauer gesagt bildet die Varietät der Flächen, auf denen Geraden liegen, eine Zariski-abgeschlossene echte Teilmenge des Modulraums aller Flächen eines Grades $d > 3$. Dazu untersuchen wir im ersten Teil allgemein, wann projektive Räume auf durch eine Gleichung definierten Hyperflächen liegen.

Daher werden wir uns im Rest der Arbeit auf eine spezielle Klasse von Flächen beschränken, die nach dem großen Satz von \textsc{Fermat} benannt wurden. Dort setzen sich einige Phänomene der 27 Geraden auf einer kubischen Fläche fort: so liegen auch auf diesen viele Geraden, deren Schnittkonfiguration hoch symmetrisch ist.

Berechnet man die Permutationen der 27 Geraden, die die Inzidenzrelationen zwischen den Geraden erhalten, so erkennt man einen hohen Grad an Symmetrie der Geradenkonfiguration. Die Gruppe dieser Permutationen enthält eine Untergruppe vom Index 2 isomorph zu der unitären Gruppe $\Unit 4{\mathbb F_4}$ mit der Konjugation $x \mapsto x^2$, dem \textsc{Frobenius}-Automorphismus von $\overline{\mathbb F_2}/\mathbb F_2$.\footcite[Aufg.~C--D, S.~180]{Mumford} Wie diese Gruppe operiert, sieht man auf der \textsc{Fermat}-Fläche vom Grad drei über dem Körper $\mathbb F_4$: denn die unitäre Gruppe operiert auf $\proj 3(\mathbb F_4)$ und lässt dabei die Norm $\langle v, v \rangle = \sum_{i=0}^3 v_i^3$ invariant. Der Index 2 lässt sich durch den \textsc{Frobenius}-Automorphismus selbst erklären, der auch auf der Fläche operiert.

Die Symmetrien der Geradenkonfiguration werden also in einer bestimmten Charakteristik von seminlinearen Symmetrien des Raumes induziert, die die Fläche invariant lassen. Im letzten Teil der Arbeit soll es darum gehen, ob für \textsc{Fermat}-Flächen höheren Grades ähnliches gilt.
