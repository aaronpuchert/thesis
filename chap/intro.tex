\chapter{Einführung} \label{chap:intro}
Kühltürme von Kraftwerken haben oft die Gestalt eines einschaligen Hyperboloids, denn ein solches lässt sich disjunkt in Geraden zerlegen und damit durch eine Struktur aus geraden Stützen stabilisieren. Über den komplexen Zahlen im projektiven Raum zerfällt sogar jede Quadrik auf diese Weise: ist sie entartet, dann in lauter parallele Geraden       , sonst kann man sie auf die Form $XY = ZW$ bringen. Diese erhält man z.\,B. aus der Standardform $X_0^2+X_1^2+X_2^2+X_3^2=0$ durch die Substitution $X = X_0 + X_1i$, $Y = X_0 - X_1i$, $Z = X_2 + X_3i$, $W = - X_2 + X_3i$. Dann sieht man leicht zwei Familien von Geraden $(\{\lambda X = \mu Z,\; \mu Y = \lambda W\})_{(\lambda:\mu) \in \proj 1}$ und $(\{\lambda X = \mu W,\; \mu Y = \lambda Z\})_{(\lambda:\mu) \in \proj 1}$. Durch jeden Punkt geht je eine Gerade beider Familien. Weil durch einen Punkt nicht mehr als zwei Geraden gehen können, sind das also bereits alle Geraden.

Bereits Mitte des neunzehnten Jahrhunderts, also noch in der Frühphase der algebraischen Geometrie, erhielten \textsc{Cayley}\footcite{Cayley} und \textsc{Salmon}\footcite{Salmon} ein vergleichbares Resultat für Flächen dritten Grades: auf jeder regulären Kubik im $\proj 3(\mathbb C)$ liegen genau 27 Geraden. Deren Konfiguration ist in den folgenden Jahrzehnten weitreichend untersucht worden.\footcite[siehe etwa][]{Henderson}

Auf Flächen höheren Grades findet man im Allgemeinen keine Geraden mehr. Genauer gesagt bildet die Varietät der Flächen, auf denen Geraden liegen, eine Zariski-abgeschlossene echte Teilmenge des Modulraums aller Flächen eines Grades $d > 3$. Dazu untersuchen wir im ersten Teil allgemein, wann projektive Räume auf Hyperflächen liegen.

Daher werden wir uns im Rest der Arbeit auf eine spezielle Klasse von Flächen beschränken, die nach dem großen Satz von \textsc{Fermat} benannt wurden. Dort setzen sich einige Phänomene der 27 Geraden auf einer kubischen Fläche fort: so liegen auch auf diesen viele Geraden, deren Schnittkonfiguration hoch symmetrisch ist.

Die Gruppe der Permutationen der 27 Geraden, die die Inzidenzrelationen zwischen den Geraden erhalten, enthält eine Untergruppe vom Index 2 isomorph zu der unitären Gruppe $\Unit 4{\mathbb F_4}$ mit der Konjugation $x \mapsto x^2$, dem \textsc{Frobenius}-Automorphismus von $\mathbb F_4/\mathbb F_2$.\footcite[Aufg.~C--D, S.~180]{Mumford} Wie diese Gruppe operiert, sieht man auf der \textsc{Fermat}-Fläche vom Grad drei über dem Körper $\mathbb F_4$, denn die unitäre Gruppe operiert auf $\proj 3(\mathbb F_4)$ und lässt dabei die Norm $\langle v, v \rangle = \sum_{i=0}^3 v_i^3$ invariant. Der Index 2 lässt sich durch den \textsc{Frobenius}-Automorphismus selbst erklären, der auch auf der Fläche operiert.

Die Symmetrien der Geradenkonfiguration werden also in einer bestimmten Charakteristik von semilinearen Symmetrien des Raumes induziert, die die Fläche invariant lassen. Im letzten Teil der Arbeit soll es darum gehen, ob für \textsc{Fermat}-Flächen höheren Grades Ähnliches gilt.
