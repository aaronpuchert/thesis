\begin{abstract}
Wir untersuchen in dieser Arbeit Konfiguration und Symmetrien der Geraden auf Fermat-Flächen. Dazu wird zunächst eine allgemeine Theorie entwickelt, mit der wir alle Geraden auf Fermat-Flächen berechnen können. Es stellt sich heraus, dass es in Charakteristik~$p > 0$ auf den Flächen vom Grad $p^n+1$ zusätzliche Geraden gibt.

Die Fermat-Flächen und damit auch die Konfiguration der auf ihnen liegenden Geraden haben eine hohe Symmetrie. Dabei gibt es im Fall der zusätzlichen Geraden auch zusätzliche Symmetrien. Daher untersuchen wir den Zusammenhang zwischen der Gruppe der Symmetrien der Fläche und der Gruppe der Permutationen der Geraden, die das Schnittverhalten respektieren. Es zeigt sich, dass in Charakteristik~$0$ und für ungeraden Grad beide Gruppen übereinstimmen.
\end{abstract}
